\section{Introduction}
One of quantum computing's most significant results is the development of Grover's
algorithm~\cite{bib-grover1996fast}. This algorithm performs an unstructured search
optimally~\cite{bib-zaika-grov-opt}. Various applications have
been proposed for such an algorithm, including the acceleration of solving NP-hard
problems~\cite{bib-williams-grover-np}.

Grover's algorithm uses a phase oracle, a quantum circuit that applies a phase of
$-1$ when the Boolean function it implements is true, to increase the probability
of returning a fulfilling assignment.

Phase oracles are interesting because a restricted class of them can be implemented
using only phase gates such as the T, T$^\dagger$, and CNOT gates (hereafter
referred to as CNOT+T). In fact, phase oracles of 3 or fewer variables can all be
implemented exactly using this set of gates, and these circuits can be calculated
exactly using the Boolean Fourier transform. Amy et al.~\cite{amy-meet-in-middle}
demonstrates that CNOT+T circuits can be optimized for both T-count and
T-depth~\cite{bib-amy-matroid}.

However, in general, it is impossible to implement Boolean functions of 4 or more
variables using only this gate set~\cite{bib-amy-rm}. There will inevitably be a need to use
a Hadamard (H) gate. The addition of this gate makes optimal solutions non-unique,
and several methods to optimize for T-count and T-depth when including the H gate
are only heuristic ~\cite{amy-meet-in-middle,bib-amy-matroid,bib-amy-rm}.

Phase oracles are straightforward to synthesize from an Exclusive Sum
of Products (ESOP) expression~\cite{bib-phaseoracle}. Since each cube maps to
a Boolean function, it is easy to see that any such function expressible with
cubes of at most 3 variables can be implemented as a CNOT+T circuit. We utilize
this observation in our proposal.

In this work, we propose a solution that utilizes CNOT+T circuits to reduce the
T-count and T-depth for phase oracles of a given Boolean function of arbitrary
literal count.

Our contributions include:

\begin{itemize}
\item A precomputed library of circuits that implement all Boolean functions of
  3 variables or fewer.
\item A synthesis method that uses ESOP minimization to limit the number of
  product terms involving 4 or more variables.
\end{itemize}

We find that our solution... [DCTODO: complete with results]

The rest of this paper is structured as follows: First Sec~\ref{Pre}
introduces some preliminary knowledge. In addition to the basics of
qubits and quantum circuits, this includes the basics of the
mathematics of phase polynomials and their mapping to phase oracles
and CNOT+T circuits. Then Sec.~\ref{Mot} goes through some
motivational examples that inform the general idea of our proposal.
Sec.~\ref{Pro} details the proposal itself, including how to
generate a library of up to 3-qubit phase oracles, and how to use
it to generate phase oracles of 4 or more qubits. In Sec.~\ref{Exp},
we test our results against Qiskit PhaseOracle and analyze our
findings. Finally, Sec.~\ref{Conc} concludes with our findings and
proposes some future research.

