\section{Introduction}
One of quantum computing's most significant results is the development of Grover's algorithm. This algorithm performs an unstructured search in the theoretical lower bound of such a search. Various applications have been proposed for such an algorithm, including the acceleration of solving NP-hard problems.

Grover's algorithm uses a phase oracle, a quantum circuit that applies a phase of $-1$ when the Boolean function it implements is true, to increase the probability of returning a fulfilling assignment.

Phase oracles are interesting because a restricted class of them can be implemented using only the T, T$^\dagger$, and CNOT gates (hereafter referred to as CNOT+T). In fact, phase oracles of 3 or fewer variables can all be implemented exactly using this set of gates, and these circuits can be calculated exactly using the Boolean Fourier transform. Amy et al.~\cite{amy-meet-in-middle} demonstrates that CNOT+T circuits can be optimized for both T-count and T-depth~\cite{bib-amy-matroid}.

However, in general, it is impossible to implement Boolean functions of 4 or more variables using only this gate set. There will inevitably be a need to use a Hadamard (H) gate. The addition of this gate makes optimal solutions non-unique, and several methods to optimize for T-count and T-depth when including the H gate are only heuristic.

Phase oracles are well known to be efficiently synthesizable from an Exclusive Sum of Products (ESOP) expression. Since each product maps to a Boolean function, it is easy to see that any such function expressible with product terms of at most 3 variables can be implemented as a CNOT+T circuit. We utilize this observation in our proposal.

In this work, we propose a solution that utilizes CNOT+T circuits to reduce the T-count and T-depth for general Boolean functions of arbitrary variable count. Our contributions include:

\begin{itemize}
    \item A precomputed library of circuits that implement all Boolean functions of 3 variables or fewer.
    \item A synthesis method that uses ESOP minimization to limit the number of product terms involving 4 or more variables.
\end{itemize}

We find that our solution... [DCTODO: complete with results]
