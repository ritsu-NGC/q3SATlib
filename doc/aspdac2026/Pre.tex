\begin{figure}[t]
  \hspace{-0.5cm}
  \begin{minipage}{0.45\linewidth}
    \centering
    \scalebox{1.0} {
      \Qcircuit @C=0.7em @R=0.7em @!R { \\
  \lstick{\ket{x}} &\qw & \gate{S}    & \qw & \rstick{e^{i\frac{\pi}{2}x}\ket{x}}  
}


    }
    \subcaption{S-gate}
    \sublabel{fig-sgate}
    \scalebox{1.0} {
      \Qcircuit @C=0.7em @R=0.7em @!R { \\
  \lstick{\ket{x}} & \qw & \gate{S^{\dagger}}    & \qw & \rstick{e^{-i\frac{\pi}{2}x}\ket{x}}  
}


    }
    \subcaption{S$^{\dagger}$-gate}
    \sublabel{fig-sdgate}
    \scalebox{1.0}{
      \Qcircuit @C=0.7em @R=0.7em @!R { \\
  \lstick{\ket{x}} & \qw & \ctrl{1} & \qw & \rstick{\ket{x}} \\
  \lstick{\ket{y}} & \qw & \targ    & \qw & \rstick{\ket{x \oplus y}}  \\
}


    }
    \subcaption{CNOT-gate}
    \sublabel{fig-cnot}
    \scalebox{1.0} {
      \Qcircuit @C=0.7em @R=0.7em @!R { \\
  \lstick{\ket{x}} &\qw & \gate{Z}    & \qw & \rstick{e^{i \pi x}\ket{x}}  
}


    }
    \subcaption{Z-gate}
    \sublabel{fig-zgate}
  \end{minipage}
  \begin{minipage}{0.45\linewidth}
  \centering
    \scalebox{1.0} {
      \Qcircuit @C=0.7em @R=0.7em @!R { \\
  \lstick{\ket{x}} & \qw & \gate{H}    & \qw & \rstick{\frac{1}{\sqrt{2}} (\ket{0} + {(-1)}^{x}\ket{1})}  
}


    }
    \subcaption{H-gate}
    \sublabel{fig-hgate}
    \scalebox{1.0} {
      \Qcircuit @C=0.7em @R=0.7em @!R { \\
  \lstick{\ket{x}} &\qw & \gate{T}    & \qw & \rstick{\omega^{x}\ket{x}}  
}


    }
    \subcaption{T-gate}
    \sublabel{fig-tgate}
    \scalebox{1.0} {
      \Qcircuit @C=0.7em @R=0.7em @!R { \\
  \lstick{\ket{x}} & \qw & \gate{T^{\dagger}}    & \qw & \rstick{\omega^{-x}\ket{x}}  
}


    }
    \subcaption{$T^{\dagger}$-gate}
    \sublabel{fig-tdgate}
    \scalebox{1.0} {
      \Qcircuit @C=0.7em @R=0.7em @!R { \\
  \lstick{\ket{x}} &\qw & \gate{X}    & \qw & \rstick{\ket{\bar{x}}}  
}


    }
    \subcaption{X-gate}
    \sublabel{fig-xgate}
  \end{minipage}
  \hspace{-0.5cm}
  \caption{Quantum gates}
  \label{fig-gates}
  \vspace{-0.5cm}
\end{figure}


\section{Preliminaries}
\label{Pre}
\subsection{Quantum Bits and Quantum Gates}
\label{Chap:Pre-qubits}
Quantum computers internally represent data as \emph{qubits}, which are quantum systems that can
take on the quantum states $\ket{0}$ and $\ket{1}$.

Additionally, qubits can also exist in quantum states that are a linear combination of $\ket{0}$ and
$\ket{1}$, which is called {\it superposition}:
\begin{equation}
\ket{\psi} = \alpha_0 \ket{0} + \alpha_1 \ket{1}, \alpha_0, \alpha_1 \in \mathbb{C}
\end{equation}

Because these coefficients $\alpha_0$ and $\alpha_1$ are complex numbers, they can also be expressed as
a complex exponential $a\cdot e^{i \theta}$, where $\theta$ is often referred to as a {\it phase}.

Qubits can be taken together as tensor products to create multiqubit states.
\begin{equation}
\label{eq-multi-qb}
\ket{\psi} = \ket{\psi}_0 \otimes \ket{\psi}_1 =(\alpha_{00} \ket{0}_0 + \alpha_{01} \ket{1}_0) \otimes
(\alpha_{10} \ket{0}_1 + \alpha_{11} \ket{1}_1)
\end{equation}
Note in Equation~\ref{eq-multi-qb} that, in particular, the tensor product of several basis states is known as a \emph{computational basis state},
and is written as a bit string consisting of the values in the component tensor multiplicands.
\begin{equation}
\ket{x_0}\otimes\ket{x_1}\otimes\cdots\otimes\ket{x_{n-1}}=\ket{x_0 x_1 \cdots x_{n-1}}=\ket{\mathbf{x}},
\mathbf{x} \in \{0,1\}^n
\end{equation}

These computational basis states can also be in linear superposition.
\begin{equation}
\ket{\psi} = \sum_{\mathbf{k} \in \{0,1\}^n} \alpha_{\mathbf{k}}\ket{\mathbf{k}},
\sum_{\mathbf{k} \in \{0,1\}^n} \alpha_{\mathbf{k}}^2 = 1
\end{equation}

{\it Quantum gates} describe transformations in qubits. The main ones concerning this work
are demonstrated in Fig.~\ref{fig-gates}. The $T/T^{\dagger}$, $S/S^{\dagger}$, and $Z$
gates will be referred to as {\it phase gates}.

All of these gates except $T/T^{\dagger}$ are known as {\it Clifford gates}. For
fault-tolerant quantum computers, Clifford gates are relatively inexpensive to implement.
However, without $T/T^{\dagger}$ gates, computations that display quantum advantage
cannot be implemented. Thus, {\it T-count}, the number of $T/T^{\dagger}$ gates, and
{\it T-depth}, the longest such serial chain of them in a circuit, are often used as a
cost metric~\cite{amy-meet-in-middle}. The set of gates in Fig.~\ref{fig-gates} is known
as the {\it Clifford+T} gates, which are universal for fault-tolerant quantum computing.

In addition, controlled versions of each of these gates exist, which means the phase is
applied when all the control bits are 1. One such example is the two-bit Control-Z gate,
which is 1 when its control bit and target bit are 1, implementing
$\ket{x}\ket{y} \rightarrow (-1)^{x \cdot y}\ket{x}\ket{y}$. We will see how this is
constructed from the gates in Fig.~\ref{fig-gates} later.

All of the gates in Fig.~\ref{fig-gates} can be composed into
{\it quantum circuits}.

\subsection{Phase Oracles and Phase Polynomials}
\label{Pre:oracle}
A {\it phase oracle} is one such quantum circuit that implements a Boolean function as a
quantum circuit, implementing a mapping of a computational basis state
$\ket{\mathbf{x}} \mapsto e^{i \pi f(\mathbf{x})}\ket{\mathbf{x}}$, where $f(\mathbf{x})$
is a Boolean function $\{0,1\}^n \rightarrow \{0,1\}$. Phase oracles are key elements in
algorithms such as Grover's Algorithm.

Previous research has demonstrated that some phase oracles can be implemented using only
CNOT and phase gates
(CNOT+T)~\cite{bib-amy-cnot}. When done in such a manner, $f(\mathbf{x})$ can be expressed
as a {\it phase polynomial},
defined in Eq.~\ref{eq-boolean-fourier}.

\begin{equation}
  \begin{aligned}
    \label{eq-boolean-fourier}
    f(\mathbf{x}) = \sum_{\mathbf{k} \neq 0} \hat{f}(\mathbf{k}) \cdot \chi_k(\mathbf{x}),
    \hat{f}(\mathbf{k}) \in \mathbb{R}
  \end{aligned}
\end{equation}

Where $\chi_k(\mathbf{x}) = ( k_0 x_0 \oplus k_1 x_1 \oplus \cdots \oplus k_{n-1} x_{n-1})$
for some $\mathbf{k} \in \{0,1\}^{n}$.

Eq.~\ref{eq-toff-bool} demonstrates one such phase polynomial. It can be verified from
Table~\ref{table-pseudo-toff} that this is indeed true.

\begin{equation}
  \label{eq-toff-bool}
  \begin{aligned}
    &x_a \cdot x_b \cdot y = \frac{1}{4}x_a + \frac{1}{4}x_b + \frac{1}{4}y - \frac{1}{4}(x_a \oplus x_b) \\
    &\qquad -\frac{1}{4}(x_a \oplus y) - \frac{1}{4}(x_b \oplus y) + \frac{1}{4}(x_a \oplus x_b \oplus y)
  \end{aligned}
\end{equation}

\def\arraystretch{1.3}
\begin{table*}[h]
  \begin{minipage}{\textwidth}
    \begin{center}
      \scalebox{1.0} {
        \begin{tabular}{c|c|c|c|c|c|c|c|c|c|c}\hline
  $x_a$ & $x_b$ & $y$ & $T_{x_a}$         & $T_{x_b}$         & $T_{y}$         & $T_{x_a \oplus x_b}$ & $T_{x_a \oplus y}$ & $T_{x_b \oplus y}$ & $T_{x_a \oplus x_b \oplus y}$ & $x_a \cdot x_b \cdot y$\\\hline
  0     & 0     & 0   & 0                 & 0                 & 0               & 0                    & 0                  & 0                  & 0                             & 0            \\\hline
  0     & 0     & 1   & 0                 & 0                 & $\frac{1}{4}$   & 0                    & $-\frac{1}{4}$     & $-\frac{1}{4}$     & $\frac{1}{4}$                 & 0            \\\hline
  0     & 1     & 0   & 0                 & $\frac{1}{4}$     & 0               & $-\frac{1}{4}$       & 0                  & $-\frac{1}{4}$     & $\frac{1}{4}$                 & 0            \\\hline
  0     & 1     & 1   & 0                 & $\frac{1}{4}$     & $\frac{1}{4}$   & $-\frac{1}{4}$       & $-\frac{1}{4}$     & 0                  & 0                             & 0            \\\hline
  1     & 0     & 0   & $\frac{1}{4}$     & 0                 & 0               & $-\frac{1}{4}$       & $-\frac{1}{4}$     & 0                  & $\frac{1}{4}$                 & 0            \\\hline
  1     & 0     & 1   & $\frac{1}{4}$     & 0                 & $\frac{1}{4}$   & $-\frac{1}{4}$       & 0                  & $-\frac{1}{4}$     & 0                             & 0            \\\hline
  1     & 1     & 0   & $\frac{1}{4}$     & $\frac{1}{4}$     & 0               & 0                    & $-\frac{1}{4}$     & $-\frac{1}{4}$     & 0                             & 0            \\\hline
  1     & 1     & 1   & $\frac{1}{4}$     & $\frac{1}{4}$     & $\frac{1}{4}$   & 0                    & 0                  & 0                  & $\frac{1}{4}$                 & 1            \\\hline
\end{tabular}

      }
      \caption{Truth table-like values of the phase polynomial of $x_a \cdot x_b \cdot y$}
      \label{table-pseudo-toff}
    \end{center}
  \end{minipage}
  \vspace{-0.5cm}
\end{table*}
\def\arraystretch{1.0}

Such a phase polynomial can be implemented as a phase oracle using the following
method~\cite{amy-meet-in-middle}.

\begin{itemize}
\item Coefficients of the phase polynomial describe the phase that needs to be driven
  as multiples of $i\pi$ (i.e. $3/4$ will be driven as an $S$-gate followed by $T$-gate).
\item The corresponding $\chi_k(\mathbf{x})$ functions can be implemented as a network
  of CNOT and X gates.
\item The phase gates are scheduled according to their corresponding $\chi_k(\mathbf{x})$
  functions.
\item CNOT and X networks are synthesized to drive each phase gate's corresponding
  $\chi_k(\mathbf{x})$ function
\item additional CNOT and X logic is added to return the state to the input state.
\end{itemize}

Scheduling these T-gates can be done using the matroid partitioning algorithm described in
\cite{bib-amy-matroid}. This guarantees a minimal T-depth for CNOT+T circuits.
The terms in Eq.~\ref{eq-toff-bool} are synthesized and scheduled using that algorithm as
demonstrated in Fig.~\ref{fig-toff-mark-matroid}.

\begin{figure}[t]
  \centering
  \scalebox{0.7} {
    %% \Qcircuit @C=0.5em @R=0.2em @!R { \\
%%                              &            &     &          &  & &                           &          &                                &          &                                &          &                             &          & \push{x_a}            &          &     & \\
%%           \lstick{\ket{x_a}} & \ctrl{3}   & \qw &          &  & & \qw                       & \ctrl{4} & \qw                            & \qw      & \qw                            & \ctrl{4} & \qw                         & \qw      & \gate{\mathrm{T}}     & \ctrl{2} & \qw & \\
%%                              &            &     &          &  & &                           &          &                                &          & \push{x_b}                     &          &                             &          & \push{x_a \oplus x_b} &          &     & \\
%%           \lstick{\ket{x_b}} & \ctrl{2}   & \qw & \push{=} &  & & \qw                       & \qw      & \qw                            & \ctrl{2} & \gate{\mathrm{T^{\dagger}}}    & \qw      & \qw                         & \ctrl{2} & \gate{\mathrm{T}}     & \targ    & \qw & \\
%%                              &            &     &          &  & & \push{y}                  &          & \push{x_a \oplus y}            &          & \push{x_a \oplus x_b \oplus y} &          & \push{x_b \oplus y}         &          &                       &          &     & \\
%%           \lstick{\ket{y}}   & \ctrl{-1}  & \qw &          &  & & \gate{\mathrm{T}}         & \targ    & \gate{\mathrm{T^{\dagger}}}    & \targ    & \gate{\mathrm{T}}              & \targ    & \gate{\mathrm{T^{\dagger}}} & \targ    & \qw                   & \qw      & \qw & \\
%% }

\Qcircuit @C=0.5em @R=0.2em @!R {
                       &           & \push{x_a}          &           &          &          & \push{x_a \oplus y}    &           &          &           &                                &           &          &       \\
\lstick{\ket{x_a}}     & \qw       & \gate{T}            & \targ     & \qw      & \ctrl{2} & \gate{T^\dagger}       & \targ     & \ctrl{4} & \qw       & \qw                            & \qw       & \ctrl{2} & \qw \\
                       &           & \push{x_b \oplus y} &           &          &          & \push{x_a \oplus x_b } &           &          &           & \push{x_a \oplus x_b \oplus y} &           &          &       \\
\lstick{\ket{x_b}}     & \targ     & \gate{T^\dagger}    & \qw       & \ctrl{2} & \targ    & \gate{T^\dagger}       & \ctrl{-2} & \qw      & \targ     & \gate{T}                       & \targ     & \targ    & \qw \\
                       &           & \push{y}            &           &          &          & \push{x_b}             &           &          &           &                                &           &          &       \\
\lstick{\ket{y}}       & \ctrl{-2} & \gate{T}            & \ctrl{-4} & \targ    & \qw      & \gate{T}               & \qw       & \targ    & \ctrl{-2} & \qw                            & \ctrl{-2} & \qw      & \qw
}

  }
  \caption{Implementing $x_a \cdot x_b \cdot y$ using CNOT and T gates, sequenced using~\cite{bib-amy-matroid}}
  \label{fig-toff-mark-matroid}
  \vspace{-0.5cm}
\end{figure}

The next section shows how $\hat{f}(\mathbf{k})$ may be obtained for the three or fewer qubit case using the
Boolean Fourier transform.

\subsection{Boolean Fourier Transform}
\label{Pre:Four}
Observe that the $\chi_k(\mathbf{x})$ from Eq.~\ref{eq-boolean-fourier} form an orthonormal basis with respect
to the below inner product.

\begin{equation}
  \label{eq-inner-prod}
  \langle f(\mathbf{x}) , g(\mathbf{x}) \rangle = \frac{1}{2^n} \sum_{\mathbf{x}} (-1)^{f(\mathbf{x})} (-1)^{g(\mathbf{x})}, \quad \mathbf{x} \in \{0,1\}^n
\end{equation}

for $n = |\mathbf{x}| \leq 3$.

That means, we can calculate $\hat{f}(\mathbf{k})$ using the below inner product

\begin{equation}
  \label{eq-fhat-prod}
  \hat{f}(\mathbf{k}) = \langle f(\mathbf{x}), \chi(\mathbf{x})_{\mathbf{k}} \rangle \forall \mathbf{k} 
\end{equation}

$\hat{f}(\mathbf{k})$ is also known as the Boolean Fourier transform of $f(\mathbf{x})$~\cite{bib-odonnell}
\footnote{All equations that are taken from~\cite{bib-odonnell} are written here
in terms of the $\{0,1\}$ basis and $\chi_{\mathbf{k}}(\mathbf{x})$ functions from ~\cite{bib-amy-matroid}}.

\begin{example}
As an example, let's derive the coefficients for Eq.~\ref{eq-toff-bool} with $\mathbf{k} = (k_{x_a}, k_{x_b}, k_y)$.
For $\mathbf{k} = 100$, $\chi_{100} = x_a$. Table~\ref{table-ex-100} enumerates the values and the term $(-1)^{f(\mathbf{x})} (-1)^{\chi_{100}(\mathbf{x})}$ for the inner product.
\begin{table}[h]
  \begin{center}
    \begin{tabular}{c|c|c|c|c|c}
      \hline
      $x_a$ & $x_b$ & $y$ & $f(\mathbf{x})$ & $\chi_{100}$ & $(-1)^{\chi_{100}(\mathbf{x})} (-1)^{f(\mathbf{x})}$ \\\hline
      0 & 0 & 0 & 0 & 0 & 1\\\hline
      0 & 0 & 1 & 0 & 0 & 1\\\hline
      0 & 1 & 0 & 0 & 0 & 1\\\hline
      0 & 1 & 1 & 0 & 0 & 1\\\hline
      1 & 0 & 0 & 0 & 1 & -1\\\hline
      1 & 0 & 1 & 0 & 1 & -1\\\hline
      1 & 1 & 0 & 0 & 1 & -1\\\hline      
      1 & 1 & 1 & 1 & 1 & 1\\\hline
    \end{tabular}
    \caption{The value of $\chi_{100}$ and the inner product $\langle f(\mathbf{x}), \chi_{100} \rangle$}
    \label{table-ex-100}
  \end{center}
\end{table}

Summing the rightmost column, as in Eq.~\ref{eq-inner-prod}:
\begin{align}
  &\hat{f}(100) &= \langle f(\mathbf{x}), \chi_{100} \rangle &= \frac{1}{8} ( 1 + 1 + 1 + 1 - 1 - 1 - 1 + 1 ) \\
  &&= \frac{1}{4}\nonumber
\end{align}
Thus,
\begin{equation}
  f(\mathbf{x}) = \frac{1}{4}x_a + \sum_{\mathbf{k} \neq 000,100} \hat{f}(\mathbf{k}) \cdot \chi_{\mathbf{k}}
\end{equation}
The rest of the calculation is omitted for brevity, but readers can verify that it results in Eq.~\ref{eq-toff-bool}.
\end{example}

Synthesizing the phase polynomial of a Boolean function using the Boolean Fourier
transform for 3 variables or fewer has an interesting consequence. First, the
number of odd multiples of $i\frac{\pi}{4}$ of $\hat{f}(\mathbf{k})$ corresponds to
the T-count needed to implement $f(\mathbf{x})$~\cite{bib-amy-rm}. This is easy to
check, since any even multiple of $i\frac{\pi}{4}$ can be implemented using only
S and Z gates, which do not take any T-cost, and any odd multiple can be realized
using S and Z gates, plus one T-gate. Since each $\hat{f}(\mathbf{k})$
is unique to each $f(\mathbf{x})$ when $|\mathbf{x}| \leq 3$,
$\hat{f}(\mathbf{k})$ has the optimal T-count needed to implement
$f(\mathbf{x})$ as a phase oracle when $|\mathbf{x}| \leq 3$.

\subsection{Synthesizing Phase Oracles from ESOP}
\label{Pre:OracleEsop}
A transformation $\ket{x}\ket{y} \rightarrow (-1)^{x \oplus y}\ket{x}\ket{y}$ can be
realized using the below circuit
\begin{figure}[h]
  \centering
  \scalebox{1.0} {
    \Qcircuit @C=0.5em @R=0.2em @!R {
      \lstick{\ket{x}} & \qw & \gate{Z} & \qw  \\
      \lstick{\ket{y}} & \qw & \gate{Z} & \qw
    }
  }
\end{figure}

Thus, given an Exclusive Sum of Products expression of a Boolean function
$f(\mathbf{x}) = x_a x_b \cdots x_g \oplus x_i x_j \cdots x_m \oplus \cdots$, we
can implement its phase oracle by inserting controlled phase gates that are controlled
by the cubes as Boolean functions

This method is used by Qiskit's Phase Oracle \cite{bib-phaseoracle}. Our proposal revolves around
the effective use of such an Exclusive Sum of Products to implement such a phase oracle.

\begin{example}
  \label{ex-esop-synth}
  We attempt to synthesize the Boolean function
  $x_2 x_3 \oplus x_3 x_4 \oplus x_4 x_5 \oplus x_2 x_4 x_5 x_1$ into a quantum circuit.
  The first term $x_2x_3$ is merely a two input Controlled-Z gate so we insert such a gate
  between the qubits containing $x_2$ and $x_3$. We do the same for $x_3x_4$ and $x_4x_5$.
  $x_2 x_4 x_5 x_1$ is a 4 input quantum Controlled-Z, which is not implementable
  using only CNOT+T. Such a gate has a T-count of at least 15. This means that the
  resulting circuit has T-count of 15. The resulting circuit is depicted in
  Fig.~\ref{fig-esop-synth}.
\end{example}

\begin{figure}[h]
 \centering
  \scalebox{1.0} {
    \Qcircuit @C=1em @R=1.2em {
  \lstick{\ket{x_1}} & \qw      & \qw      & \qw      & \ctrl{4} & \qw \\
  \lstick{\ket{x_2}} & \ctrl{1} & \qw      & \qw      & \ctrl{3} & \qw \\
  \lstick{\ket{x_3}} & \gate{Z} & \ctrl{1} & \qw      & \qw      & \qw \\
  \lstick{\ket{x_4}} & \qw      & \gate{Z} & \ctrl{1} & \ctrl{1} & \qw \\
  \lstick{\ket{x_5}} & \qw      & \qw      & \gate{Z} & \gate{Z} & \qw
}

  }
  \caption{A phase oracle for $x_2x_3 \oplus x_3x_4 \oplus x_4x_5 \oplus x_2x_4 x_5x_1$}
  \label{fig-esop-synth}
  \vspace{-0.5cm}
\end{figure}


%% DCTODO The first term is merely a 3-input AND, which is already seen in
%%   Fig.~\ref{fig-toff-mark-matroid}. Thus we merely insert this circuit to implement
%%   the first term. The second term has a negated input, so we insert X on the $x_c$ qubit
%%   after the first circuit to negate $x_c$, again insert Fig.~\ref{fig-toff-mark-matroid}
%%   to implement the phase, and finally place another X on $x_c$ to return its value to
%%   the original.
