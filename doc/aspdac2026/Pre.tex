\begin{figure}[t]
  \begin{minipage}{0.45\linewidth}
    \centering
    \scalebox{1.0} {
      \Qcircuit @C=0.7em @R=0.7em @!R { \\
  \lstick{\ket{x}} &\qw & \gate{S}    & \qw & \rstick{e^{i\frac{\pi}{2}x}\ket{x}}  
}


    }
    \subcaption{S-gate}
    \sublabel{fig-sgate}
    \scalebox{1.0} {
      \Qcircuit @C=0.7em @R=0.7em @!R { \\
  \lstick{\ket{x}} & \qw & \gate{S^{\dagger}}    & \qw & \rstick{e^{-i\frac{\pi}{2}x}\ket{x}}  
}


    }
    \subcaption{S$^{\dagger}$-gate}
    \sublabel{fig-sdgate}
    \scalebox{1.0}{
      \Qcircuit @C=0.7em @R=0.7em @!R { \\
  \lstick{\ket{x}} & \qw & \ctrl{1} & \qw & \rstick{\ket{x}} \\
  \lstick{\ket{y}} & \qw & \targ    & \qw & \rstick{\ket{x \oplus y}}  \\
}


    }
    \subcaption{CNOT-gate}
    \sublabel{fig-cnot}
    \scalebox{1.0} {
      \Qcircuit @C=0.7em @R=0.7em @!R { \\
  \lstick{\ket{x}} &\qw & \gate{Z}    & \qw & \rstick{e^{i \pi x}\ket{x}}  
}


    }
    \subcaption{Z-gate}
    \sublabel{fig-zgate}
  \end{minipage}
  \begin{minipage}{0.45\linewidth}
  \centering
    \scalebox{1.0} {
      \Qcircuit @C=0.7em @R=0.7em @!R { \\
  \lstick{\ket{x}} &\qw & \gate{T}    & \qw & \rstick{\omega^{x}\ket{x}}  
}


    }
    \subcaption{T-gate}
    \sublabel{fig-tgate}
    \scalebox{1.0} {
      \Qcircuit @C=0.7em @R=0.7em @!R { \\
  \lstick{\ket{x}} & \qw & \gate{T^{\dagger}}    & \qw & \rstick{\omega^{-x}\ket{x}}  
}


    }
    \subcaption{$T^{\dagger}$-gate}
    \sublabel{fig-tdgate}
    \scalebox{1.0} {
      \Qcircuit @C=0.7em @R=0.7em @!R { \\
  \lstick{\ket{x}} & \qw & \gate{H}    & \qw & \rstick{\frac{1}{\sqrt{2}} (\ket{0} + {(-1)}^{x}\ket{1})}  
}


    }
    \subcaption{H-gate}
    \sublabel{fig-hgate}
    \scalebox{1.0} {
      \Qcircuit @C=0.7em @R=0.7em @!R { \\
  \lstick{\ket{x}} &\qw & \gate{X}    & \qw & \rstick{\ket{\bar{x}}}  
}


    }
    \subcaption{X-gate}
    \sublabel{fig-xgate}
  \end{minipage}
  \caption{Quantum gates}
  \label{fig-gates}
\end{figure}


\section{Preliminaries}

\subsection{Quantum Bits and Quantum Gates}
\label{Chap:Pre-qubits}
Quantum computers internally represent data as \emph{qubits}, which are quantum systems that can
take on the quantum states $\ket{0}$ and $\ket{1}$,

Additionally, qubits can also exist in quantum states that are a linear combination of $\ket{0}$ and
$\ket{1}$, which is called {\it superposition}:
\begin{equation}
\ket{\psi} = \alpha_0 \ket{0} + \alpha_1 \ket{1}, \alpha_0, \alpha_1 \in \mathbb{C}
\end{equation}

Because these coefficients $\alpha_0$ and $\alpha_b$ are complex numbers, they can also be expressed as
a complex exponential $a\cdot e^{i \theta}$.
$\theta$ in this exponential is often referred to as a {\it phase}.

Qubits can be taken together as tensor products to create multiqubit states.
\begin{equation}
\label{eq-multi-qb}
\ket{\psi} = \ket{\psi}_0 \otimes \ket{\psi}_1 =(\alpha_{00} \ket{0}_0 + \alpha_{01} \ket{1}_0) \otimes
(\alpha_{10} \ket{0}_1 + \alpha_{11} \ket{1}_1)
\end{equation}
Note in Equation~\ref{eq-multi-qb} that In particular, the tensor product of several basis states is known as a \emph{computational basis state},
and is written as a bit string consisting of the values in the component tensor multiplicands. 
\begin{equation}
\ket{x_0}\otimes\ket{x_1}\otimes\cdots\otimes\ket{x_{n-1}}=\ket{x_0 x_1 \cdots x_{n-1}}=\ket{\mathbf{x}},
\mathbf{x} \in \{0,1\}^n
\end{equation}

These computational basis states can similarly be in linear superposition.
\begin{equation}
\ket{\psi} = \sum_{\mathbf{k} \in \{0,1\}^n} \alpha_{\mathbf{k}}\ket{\mathbf{k}},
\sum_{\mathbf{k} \in \{0,1\}^n} \alpha_{\mathbf{k}} = 1
\end{equation}

{\it Quantum gates} describe transformations in qubits. The main ones that concern this work are demonstrated
in Fig.~\ref{fig-gates}. The $T/T^{\dagger}$, $S/S^{\dagger}$, and $Z$ gates will be referred to as {\it phase gates}.
All of the gates in Fig.~\ref{fig-gates} can in turn be composed into {\it quantum circuits}. 

\subsection{Phase Oracles and Phase Polynomials}
A {\it phase oracle} is one such quantum circuit that implements a Boolean function as a quantum circuit, implementing
a mapping of a computational basis state $\ket{\mathbf{x}} \mapsto (-1)^{f(\mathbf{x})}\ket{\mathbf{x}}$, where $f(\mathbf{x})$
is a Boolean function $\{0,1\}^n \rightarrow \{0,1\}$. Phase oracles are key elements in algorithms such as
Grover's Algorithm.

Previous research has demonstrated that phase oracles can be implemented using only CNOT and phase gates
(CNOT+T)~\cite{bib-amy-cnot}. When done in such a manner, $f(\mathbf{x})$ can be expressed as a {\it phase polynomial},
defined in Eq.~\ref{eq-boolean-fourier}.

\begin{equation}
  \begin{aligned}
    \label{eq-boolean-fourier}
    f(\mathbf{x}) = \sum_{\mathbf{k} \neq 0} \hat{f}(\mathbf{k}) \cdot \chi_k(\mathbf{x}), \\\nonumber
    \hat{f}(\mathbf{k}) \in \mathbb{R} \quad \mathbf{k} \in \{0,1\}^{n}
  \end{aligned}
\end{equation}

Where $( k_0 x_0 \oplus k_1 x_1 \oplus \cdots \oplus k_{n-1} x_{n-1}) = \chi_k(\mathbf{x})$

Eq.~\ref{eq-toff-bool} demonstrates one such phase polynomial. It can be verified from Table~\ref{table-pseudo-toff} that
this is indeed true.

\begin{equation}
  \label{eq-toff-bool}
  \begin{aligned}
    &x_a \cdot x_b \cdot y = \frac{1}{4}x_a + \frac{1}{4}x_b + \frac{1}{4}y - \frac{1}{4}(x_a \oplus x_b) \\
    &\qquad -\frac{1}{4}(x_a \oplus y) - \frac{1}{4}(x_b \oplus y) + \frac{1}{4}(x_a \oplus x_b \oplus y)
  \end{aligned}
\end{equation}

\def\arraystretch{1.3}
\begin{table*}[h]
  \begin{minipage}{\textwidth}
    \begin{center}
      \scalebox{1.0} {
        \begin{tabular}{c|c|c|c|c|c|c|c|c|c|c}\hline
          $x_a$ & $x_b$ & $y$ & $T_{x_a}$         & $T_{x_b}$         & $T_{y}$         & $T_{x_a \oplus x_b}$ & $T_{x_a \oplus y}$ & $T_{x_b \oplus y}$ & $T_{x_a \oplus x_b \oplus y}$ & $x_a \cdot x_b \cdot y$\\\hline
          0     & 0     & 0   & 0                 & 0                 & 0               & 0                    & 0                  & 0                  & 0                             & 0            \\\hline
          0     & 0     & 1   & 0                 & 0                 & $\frac{1}{4}$   & 0                    & $-\frac{1}{4}$     & $-\frac{1}{4}$     & $\frac{1}{4}$                 & 0            \\\hline
          0     & 1     & 0   & 0                 & $\frac{1}{4}$     & 0               & $-\frac{1}{4}$       & 0                  & $-\frac{1}{4}$     & $\frac{1}{4}$                 & 0            \\\hline
          0     & 1     & 1   & 0                 & $\frac{1}{4}$     & $\frac{1}{4}$   & $-\frac{1}{4}$       & $-\frac{1}{4}$     & 0                  & 0                             & 0            \\\hline
          1     & 0     & 0   & $\frac{1}{4}$     & 0                 & 0               & $-\frac{1}{4}$       & $-\frac{1}{4}$     & 0                  & $\frac{1}{4}$                 & 0            \\\hline
          1     & 0     & 1   & $\frac{1}{4}$     & 0                 & $\frac{1}{4}$   & $-\frac{1}{4}$       & 0                  & $-\frac{1}{4}$     & 0                             & 0            \\\hline
          1     & 1     & 0   & $\frac{1}{4}$     & $\frac{1}{4}$     & 0               & 0                    & $-\frac{1}{4}$     & $-\frac{1}{4}$     & 0                             & 0            \\\hline
          1     & 1     & 1   & $\frac{1}{4}$     & $\frac{1}{4}$     & $\frac{1}{4}$   & 0                    & 0                  & 0                  & $\frac{1}{4}$                 & 1            \\\hline
        \end{tabular}
      }
      \caption{Truth table-like values of the pseudo-Boolean representation of $x_a \cdot x_b \cdot y$}
      \label{table-pseudo-toff}
    \end{center}
  \end{minipage}
\end{table*}
\def\arraystretch{1.1}

Such a phase polynomial can be implemented as a phase oracle using the following
method.

\begin{itemize}
\item Coefficients of the phase polynomial describe the phase that needs to be driven
  as multiples of $i\pi$ (i.e. $3/4$ will be driven as an $S$-gate followed by $T$-gate).
\item The corresponding $\chi_k(\mathbf{x})$ functions can be implemented as a network
  of CNOT and X gates.
\item The phase gates are sequenced by their corresponding $\chi_k(\mathbf{x})$ functions.
\item CNOT and X networks are synthesized to drive each phase gate's corresponding
  $\chi_k(\mathbf{x})$ function
\item additional CNOT and X logic is added to return the state to the input state.
\end{itemize}


\begin{figure}[t]
  \centering
  \scalebox{0.7} {
    \Qcircuit @C=0.5em @R=0.2em @!R { \\
                             &            &     &          &  & &                           &          &                                &          &                                &          &                             &          & \push{x_a}            &          &     & \\
          \lstick{\ket{x_a}} & \ctrl{3}   & \qw &          &  & & \qw                       & \ctrl{4} & \qw                            & \qw      & \qw                            & \ctrl{4} & \qw                         & \qw      & \gate{\mathrm{T}}     & \ctrl{2} & \qw & \\
                             &            &     &          &  & &                           &          &                                &          & \push{x_b}                     &          &                             &          & \push{x_a \oplus x_b} &          &     & \\
          \lstick{\ket{x_b}} & \ctrl{2}   & \qw & \push{=} &  & & \qw                       & \qw      & \qw                            & \ctrl{2} & \gate{\mathrm{T^{\dagger}}}    & \qw      & \qw                         & \ctrl{2} & \gate{\mathrm{T}}     & \targ    & \qw & \\
                             &            &     &          &  & & \push{y}                  &          & \push{x_a \oplus y}            &          & \push{x_a \oplus x_b \oplus y} &          & \push{x_b \oplus y}         &          &                       &          &     & \\
          \lstick{\ket{y}}   & \ctrl{-1}  & \qw &          &  & & \gate{\mathrm{T}}         & \targ    & \gate{\mathrm{T^{\dagger}}}    & \targ    & \gate{\mathrm{T}}              & \targ    & \gate{\mathrm{T^{\dagger}}} & \targ    & \qw                   & \qw      & \qw & \\
}

  }
  \caption{Implementing $x_a \cdot x_b \cdot y$ using CNOT and T gates}
  \label{fig-toff-mark}
\end{figure}


The next section shows how $\hat{f}(\mathbf{k})$ may be obtained for the three or fewer qubit case using the
Boolean Fourier transform.

\section{Boolean Fourier Transform}


In plain terms, it is a sum of products of Boolean functions with real coefficients, resulting in binary values~\cite{bib-barenco-elementary}. This representation is also known as a {\it phase polynomial}~\cite{bib-amy-cnot}.

\vspace{0.5cm}

Phase polynomials are important because they can be calculated using the {\it Boolean Fourier transform}:
\begin{equation}
  \begin{aligned}
    \label{eq-boolean-fourier}
    f(\mathbf{x}) = \sum_{\mathbf{k} \neq 0} \hat{f}(\mathbf{k}) \cdot ( k_0 x_0 \oplus k_1 x_1 \oplus \cdots \oplus k_{n-1} x_{n-1}), \\\nonumber
    \hat{f}(\mathbf{k}) \in \mathbb{R} \quad \mathbf{k} \in \{0,1\}^{n}
  \end{aligned}
\end{equation}

\vspace{0.3cm}

%DCTODO The coefficients $\hat{f}(\mathbf{k})$ (for all nontrivial $\mathbf{k}$) specify the phase polynomial, and the linear functions  form a basis.

Such a phase polynomial can be realized using a CNOT+T network, as in Fig.~\ref{fig-toff-mark}. Minimizing the number of odd multiples of $\frac{\pi}{4}$ in $\hat{f}(\mathbf{k})$ also minimizes the T-count~\cite{bib-amy-rm}. Optimizing T-depth is done as in~\cite{bib-amy-matroid}.

\vspace{0.4cm}

\section{Phase Polynomial Calculation for the Three Qubit Case}
\label{Chap:Bool-pbool3q}

\vspace{0.3cm}

This section shows how to calculate the phase polynomial for a three-input Boolean function, using the Boolean Fourier transform~\cite{bib-odonnell}. The notation is chosen to clarify how the equation may be implemented using quantum gates in the CNOT+T basis. 

\vspace{0.3cm}

The inner product for the Boolean Fourier transform is
\begin{equation}
  \label{eq-inner-prod}
  \langle f(\mathbf{x}) , g(\mathbf{x}) \rangle = \frac{1}{2^n} \sum_{\mathbf{x}} (-1)^{f(\mathbf{x})} (-1)^{g(\mathbf{x})}, \quad \mathbf{x} \in \{0,1\}^n
\end{equation}

The {\it parity functions}, i.e., exclusive sums of input variables $\chi(x_0,\ldots,x_n)$ as in Eq.~\ref{eq-bool-basis}, form an orthogonal basis with respect to this product:
\begin{equation}
  \label{eq-bool-basis}
  \chi(x_0,\ldots,x_n)_{\mathbf{k}} = k_0 x_0 \oplus k_1 x_1 \oplus \cdots \oplus k_{n-1} x_{n-1},\quad k_i \in \{0,1\}
\end{equation}

\vspace{0.2cm}

Since $\langle \chi_{\mathbf{k}}, \chi_{\mathbf{b}} \rangle = 0$ for $\mathbf{k} \neq \mathbf{b}$, the coefficient $\hat{f}(\mathbf{k})$ is computed as
\begin{equation}
  \label{eq-fhat-prod}
  \hat{f}(\mathbf{k}) = \langle f(\mathbf{x}), \chi(\mathbf{x})_{\mathbf{k}} \rangle
\end{equation}

\vspace{0.2cm}

\begin{example}
As an example, let's derive the coefficients for Eq.~\ref{eq-toff-bool} with $\mathbf{k} = (k_{x_a}, k_{x_b}, k_y)$.
For $\mathbf{k} = 100$, $\chi_{100} = x_a$. Table~\ref{table-ex-100} enumerates the values and the term $(-1)^{f(\mathbf{x})} (-1)^{\chi_{100}(\mathbf{x})}$ for the inner product.
\begin{table}[h]
  \begin{center}
    \begin{tabular}{c|c|c|c|c|c}
      \hline
      $x_a$ & $x_b$ & $y$ & $f(\mathbf{x})$ & $\chi_{100}$ & $(-1)^{\chi_{100}(\mathbf{x})} (-1)^{f(\mathbf{x})}$ \\\hline
      0 & 0 & 0 & 0 & 0 & 1\\\hline
      0 & 0 & 1 & 0 & 0 & 1\\\hline
      0 & 1 & 0 & 0 & 0 & 1\\\hline
      0 & 1 & 1 & 0 & 0 & 1\\\hline
      1 & 0 & 0 & 0 & 1 & -1\\\hline
      1 & 0 & 1 & 0 & 1 & -1\\\hline
      1 & 1 & 0 & 0 & 1 & -1\\\hline      
      1 & 1 & 1 & 1 & 1 & 1\\\hline
    \end{tabular}
    \caption{The value of $\chi_{100}$ and the inner product $\langle f(\mathbf{x}), \chi_{100} \rangle$}
    \label{table-ex-100}
  \end{center}
\end{table}

Summing the rightmost column, as in Eq.~\ref{eq-inner-prod}:
\begin{align}
    &\hat{f}(100) = \langle f(\mathbf{x}), \chi_{100} \rangle = \frac{1}{8} ( 1 + 1 + 1 + 1 - 1 - 1 - 1 + 1 ) = \frac{1}{4}
\end{align}
Thus,
\begin{equation}
  f(\mathbf{x}) = \frac{1}{4}x_a + \sum_{\mathbf{k} \neq 000,100} \hat{f}(\mathbf{k}) \cdot \chi_{\mathbf{k}}
\end{equation}
The rest of the calculation is omitted for brevity, but readers can verify that it results in Eq.~\ref{eq-toff-bool}.
\end{example}

These circuits are notable because, for the 3-input case, it is straightforward to optimize the {\it T-count}, i.e., the number of T-gates and $T^{\dagger}$-gates~\cite{amy-meet-in-middle}. Optimizing T-count is crucial for fault-tolerant circuit implementation, as T-gates are expensive in most encodings~\cite{bib-herr-lattice,bib-fowler-bridge}. The {\it T-depth} is the longest sequence of T-gates in a circuit, which is also important since many modern encodings can execute T-gates in parallel for cost savings~\cite{bib-google-ecc}.

The 3-input case is interesting because it is the largest class of Boolean functions representable by a CNOT+T circuit. Since there are only 256 such Boolean functions, there are potential uses for a library of such oracles.
