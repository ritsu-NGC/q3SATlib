\section{Preliminaries}

\section{Quantum Bits}
\label{Chap:Pre-qubits}
Quantum computers internally represent data as \emph{qubits}, which are quantum systems that can
take on the quantum states $\ket{0}$ and $\ket{1}$,

Additionally, qubits can also exist in quantum states that are a linear combination of $\ket{0}$ and
$\ket{1}$, which is called {\it superposition}:
\begin{equation}
\ket{\psi} = \alpha_0 \ket{0} + \alpha_1 \ket{1}, \alpha_0, \alpha_1 \in \mathbb{C}
\end{equation}

Because these coefficients $\alpha_0$ and $\alpha_b$ are complex numbers, they can also be expressed as
a complex exponential $a\cdot e^{i \theta}$.
$\theta$ in this exponential is often referred to as a {\it phase}.

Qubits can be taken together as tensor products to create multiqubit states.
\begin{equation}
\label{eq-multi-qb}
\ket{\psi} = \ket{\psi}_0 \otimes \ket{\psi}_1 =(\alpha_{00} \ket{0}_0 + \alpha_{01} \ket{1}_0) \otimes
(\alpha_{10} \ket{0}_1 + \alpha_{11} \ket{1}_1)
\end{equation}
Note in Equation~\ref{eq-multi-qb} that In particular, the tensor product of several basis states is known as a \emph{computational basis state},
and is written as a bit string consisting of the values in the component tensor multiplicands. 
\begin{equation}
\ket{x_0}\otimes\ket{x_1}\otimes\cdots\otimes\ket{x_{n-1}}=\ket{x_0 x_1 \cdots x_{n-1}}=\ket{\mathbf{x}},
\mathbf{x} \in \{0,1\}^n
\end{equation}

These computational basis states can similarly be in linear superposition.
\begin{equation}
\ket{\psi} = \sum_{\mathbf{a} \in \{0,1\}^n} \alpha_{\mathbf{a}}\ket{\mathbf{a}},
\sum_{\mathbf{a} \in \{0,1\}^n} \alpha_{\mathbf{a}} = 1
\end{equation}

Multiqubit states can also have equivalence up to a global phase, as well as a relative phase.

Because computational basis states are analogous to classical bit strings, quantum operations that map
between basis states are analogous to Boolean functions. This work focuses mainly on operations
that map between computational basis states.

One additional property is when two qubits $q_0$ and $q_1$ are in a two-qubit state $\ket{\psi}$,
then they are said to be {\it entangled} if there exists no $\ket{\psi_a}_{0}$ and $\ket{\psi_b}_{1}$
such that
\begin{equation}
        \ket{\psi} = \ket{\psi_a}_{0} \otimes \ket{\psi_b}_{1}
\end{equation}

An example of such an entangled state is a {\it Bell pair}
\begin{equation}
        \ket{\psi} = \frac{1}{\sqrt{2}} (\ket{01} + \ket{10})
\end{equation}


\subsection{Quantum Bits and Quantum Gates}
\section{Quantum Bits}
\label{Chap:Pre-qubits}
Quantum computers internally represent data as \emph{qubits}, which are quantum systems that can
take on the quantum states $\ket{0}$ and $\ket{1}$, where $\ket{}$ is known as the Dirac
notation for indicating a quantum state~\cite{bib-mike-and-ike}. Translating these quantum states to classically understandable values like $0$ and $1$ requires an operation called a
{\it measurement}. Measurement, for the purposes of this dissertation, measures a $\ket{0}$ state and returns
the value $0$, and the $\ket{1}$ state and returns the value $1$.

Additionally, qubits can also exist in quantum states that are a linear combination of $\ket{0}$ and
$\ket{1}$, which is called {\it superposition}:
\begin{equation}
\ket{\psi} = \alpha_0 \ket{0} + \alpha_1 \ket{1}, \alpha_0, \alpha_1 \in \mathbb{C}
\end{equation}

Where the coefficients $\alpha_0$ and $\alpha_1$ are constrained by the equation $\alpha_0^2 + \alpha_1^2 = 1$.
As such, $\ket{0}$ and $\ket{1}$ represent a {\it basis} for this quantum state. The coefficients on these basis
states are important because for a qubit in linear superposition, $\alpha_0^2$ is the probability of measuring
$\ket{\psi}$ as $0$, and $\alpha_1^2$ is the probability of measuring $\ket{\psi}$ as $1$. Because these
coefficients are complex numbers, they can also be expressed as a complex exponential $a\cdot e^{i \theta}$.
$\theta$ in this exponential is often referred to as a {\it phase}.

When comparing two quantum states, if $\ket{\psi_0}$ is related to $\ket{\psi_1}$ by the following relation:
\begin{equation}
        \ket{\psi_1} = e^{i \theta_g} \ket{\psi_0}
\end{equation}
then $\ket{\psi_0}$ and $\ket{\psi_1}$ are equivalent up to a {\it global phase} $\theta_g$.

In contrast, if instead the relation between $\ket{\psi_0}$ and $\ket{\psi_1}$ can only be expressed as a
function $\theta(k)$ in the way shown in the equation below:
\begin{equation}
        \ket{\psi_0} = \sum_{k \in \{0,1\}} a_k e^{i \theta_0} \ket{k}
        \ket{\psi_1} = \sum_{k \in \{0,1\}} a_k e^{i \theta_0 + \theta(k)} \ket{k}        
\end{equation}
then $\ket{psi_0}$ and $\ket{\psi_1}$ are equivalent up to {\it relative phase}.

The distinction between global phase and relative phase is important. In general, global phases are
invariant under most quantum operations, including measurement. However, the same cannot be said
for those equivalent up to a relative phase. In general, states that are equivalent up to a relative
phase cannot be considered equivalent in most cases. This dissertation attempts to find cases which
can tolerate this relative phase to effect the same quantum operations.

\section{Multiqubit States}
Qubits can be taken together as tensor products to create multiqubit states.
\begin{equation}
\label{eq-multi-qb}
\ket{\psi} = \ket{\psi}_0 \otimes \ket{\psi}_1 =(\alpha_{00} \ket{0}_0 + \alpha_{01} \ket{1}_0) \otimes
(\alpha_{10} \ket{0}_1 + \alpha_{11} \ket{1}_1)
\end{equation}
Note in Equation~\ref{eq-multi-qb} that In particular, the tensor product of several basis states is known as a \emph{computational basis state},
and is written as a bit string consisting of the values in the component tensor multiplicands. 
\begin{equation}
\ket{x_0}\otimes\ket{x_1}\otimes\cdots\otimes\ket{x_{n-1}}=\ket{x_0 x_1 \cdots x_{n-1}}=\ket{\mathbf{x}},
\mathbf{x} \in \{0,1\}^n
\end{equation}

Additionally, many of the properties in Section~\ref{Chap:Pre-qubits} generalize to the multiqubit case.
Observe that computational basis states can similarly be in linear superposition.
\begin{equation}
\ket{\psi} = \sum_{\mathbf{a} \in \{0,1\}^n} \alpha_{\mathbf{a}}\ket{\mathbf{a}},
\sum_{\mathbf{a} \in \{0,1\}^n} \alpha_{\mathbf{a}} = 1
\end{equation}

Multiqubit states can also have equivalence up to a global phase, as well as a relative phase.

Because computational basis states are analogous to classical bit strings, quantum operations that map
between basis states are analogous to Boolean functions. This work focuses mainly on operations
that map between computational basis states.

One additional property is when two qubits $q_0$ and $q_1$ are in a two-qubit state $\ket{\psi}$,
then they are said to be {\it entangled} if there exists no $\ket{\psi_a}_{0}$ and $\ket{\psi_b}_{1}$
such that
\begin{equation}
        \ket{\psi} = \ket{\psi_a}_{0} \otimes \ket{\psi_b}_{1}
\end{equation}

An example of such an entangled state is a {\it Bell pair}
\begin{equation}
        \ket{\psi} = \frac{1}{\sqrt{2}} (\ket{01} + \ket{10})
\end{equation}


\section{Quantum Gates}

In general, quantum operations are depicted using {\it quantum gates}, which effect the mapping
shown below:

\begin{figure}[h]
  \begin{center}
  \scalebox{1.25} {
    \Qcircuit @C=0.5em @R=0.5em @!R { \\                         
      \lstick{\ket{\mathbf{x}}} & \qw   & \gate{U}   & \qw & \push{\sum_{\mathbf{k} \in \{0,1\}^n} \alpha_{\mathbf{k}}(\mathbf{x}) e^{i \theta_{\mathbf{k}}(\mathbf{x})} \ket{\mathbf{k}}}
    }
  }
  \end{center}
\end{figure}

Because quantum gates are unitary operations as dictated by the laws of quantum mechanics, each quantum gate
also has corresponding {\it inverse} that effects the reverse mapping, marked by the symbol $^{\dagger}$.
Replacing a quantum gate with its inverse is called {\it conjugation}:
\begin{figure}[h]
  \begin{center}
  \hspace{4cm}
  \scalebox{1.25} {
    \Qcircuit @C=0.5em @R=0.5em @!R { \\                         
      \lstick{\sum_{\mathbf{k} \in \{0,1\}^n} \alpha_{\mathbf{k}}(\mathbf{x}) e^{i \theta_{\mathbf{k}}(\mathbf{x})}
      \ket{\mathbf{k}}} & \qw   & \gate{U^{\dagger}}   & \qw & \push{\ket{\mathbf{x}}}
    }
  }
  \end{center}
\end{figure}
\begin{figure}[t]
  \begin{minipage}[t]{0.45\textwidth}
    \centering
    \scalebox{1.0}{
      \Qcircuit @C=0.7em @R=0.7em @!R { \\
  \lstick{\ket{x}} & \qw & \ctrl{1} & \qw & \rstick{\ket{x}} \\
  \lstick{\ket{y}} & \qw & \targ    & \qw & \rstick{\ket{x \oplus y}}  \\
}


    }
    \subcaption{CNOT-gate}
    \sublabel{fig-cnot}
    \centering
    \scalebox{1.0} {
      \Qcircuit @C=0.7em @R=0.7em @!R { \\
  \lstick{\ket{x}} &\qw & \gate{T}    & \qw & \rstick{\omega^{x}\ket{x}}  
}


    }
    \subcaption{T-gate}
    \sublabel{fig-tgate}
    \centering
    \scalebox{1.0} {
      \Qcircuit @C=0.7em @R=0.7em @!R { \\
  \lstick{\ket{x}} &\qw & \gate{S}    & \qw & \rstick{e^{i\frac{\pi}{2}x}\ket{x}}  
}


    }
    \subcaption{S-gate}
    \sublabel{fig-sgate}
  \end{minipage}
  \begin{minipage}[t]{.45\textwidth}
    \vspace{5mm}
    \centering
    \hspace{5mm}
    \centering
    \scalebox{1.0} {
      \Qcircuit @C=0.7em @R=0.7em @!R { \\
  \lstick{\ket{x}} & \qw & \gate{H}    & \qw & \rstick{\frac{1}{\sqrt{2}} (\ket{0} + {(-1)}^{x}\ket{1})}  
}


    }
    \subcaption{H-gate}
    \sublabel{fig-hgate}
    \centering
    \scalebox{1.0} {
      \Qcircuit @C=0.7em @R=0.7em @!R { \\
  \lstick{\ket{x}} & \qw & \gate{T^{\dagger}}    & \qw & \rstick{\omega^{-x}\ket{x}}  
}


    }
    \subcaption{$T^{\dagger}$-gate}
    \sublabel{fig-tdgate}
    \centering
    \scalebox{1.0} {
      \Qcircuit @C=0.7em @R=0.7em @!R { \\
  \lstick{\ket{x}} & \qw & \gate{S^{\dagger}}    & \qw & \rstick{e^{-i\frac{\pi}{2}x}\ket{x}}  
}


    }
    \subcaption{$S^{\dagger}$-gate}
    \sublabel{fig-sdgate}
  \end{minipage}
  \caption{The H, CNOT, T, S (Clifford+T) basis gate set}
  \label{fig-basis}
\end{figure}

Figure~\ref{fig-basis} enumerates the elementary gate set consisting of the $H$-gate, $\mathit{CNOT}$,
$T$-gate, $T^{\dagger}$-gate (the inverse of the $T$-gate), $S$-gate, and $S^{\dagger}$-gate (the
inverse of the $S$-gate), known as the \emph{Clifford+T basis gate set}. As mentioned in
Chapter~\ref{Chap:Intro}, this gate set is enough to implement universal quantum computing,
including operations that have a quantum advantage~\cite{bib-solovay-kitaev}.
The implication, therefore, is that a finite set of gates is enough to express any quantum operation up
to a finite error with minimal overhead. This means the Clifford+T gate set can be used as a basis set for
fault-tolerant quantum computing, provided each of the gates can be implemented fault-tolerantly.

In the fault-tolerant paradigm, $T$-gates and $T^{\dagger}$-gates are much more expensive to implement than
$H$-gate,$CNOT$, $S$-gate, and $S^{\dagger}$-gates~\cite{bib-bravyi-kitaev-magic-distillation}. This means
that the number of T-gates is key to lowering the cost of the implementation. Hereafter, 
the set of the T-gate and the T$^{\dagger}$-gate are referred to as ``T-gates'', and the set of the S-gate and the
S$^{\dagger}$-gate ``S-gates'', unless otherwise noted.

When this set of gates is assembled into a network, the result is called a \emph{quantum circuit}. In a
quantum circuit, inputs come in from the left, gates are applied in order from left to right, and outputs
go out from the right. Quantum circuits also have \emph{output qubits}, which are the qubits that output
quantum states to be measured or used by other quantum circuits. 

In general, quantum operations are depicted using {\it quantum gates}, which effect the mapping
shown below:

\begin{figure}[h]
  \begin{center}
  \scalebox{1.25} {
    \Qcircuit @C=0.5em @R=0.5em @!R { \\                         
      \lstick{\ket{\mathbf{x}}} & \qw   & \gate{U}   & \qw & \push{\sum_{\mathbf{k} \in \{0,1\}^n} \alpha_{\mathbf{k}}(\mathbf{x}) e^{i \theta_{\mathbf{k}}(\mathbf{x})} \ket{\mathbf{k}}}
    }
  }
  \end{center}
\end{figure}

Because quantum gates are unitary operations as dictated by the laws of quantum mechanics, each quantum gate
also has corresponding {\it inverse} that effects the reverse mapping, marked by the symbol $^{\dagger}$.
Replacing a quantum gate with its inverse is called {\it conjugation}:
\begin{figure}[h]
  \begin{center}
  \hspace{4cm}
  \scalebox{1.25} {
    \Qcircuit @C=0.5em @R=0.5em @!R { \\                         
      \lstick{\sum_{\mathbf{k} \in \{0,1\}^n} \alpha_{\mathbf{k}}(\mathbf{x}) e^{i \theta_{\mathbf{k}}(\mathbf{x})}
      \ket{\mathbf{k}}} & \qw   & \gate{U^{\dagger}}   & \qw & \push{\ket{\mathbf{x}}}
    }
  }
  \end{center}
\end{figure}
\begin{figure}[t]
  \begin{minipage}[t]{0.45\textwidth}
    \centering
    \scalebox{1.0}{
      \Qcircuit @C=0.7em @R=0.7em @!R { \\
  \lstick{\ket{x}} & \qw & \ctrl{1} & \qw & \rstick{\ket{x}} \\
  \lstick{\ket{y}} & \qw & \targ    & \qw & \rstick{\ket{x \oplus y}}  \\
}


    }
    \subcaption{CNOT-gate}
    \sublabel{fig-cnot}
    \centering
    \scalebox{1.0} {
      \Qcircuit @C=0.7em @R=0.7em @!R { \\
  \lstick{\ket{x}} &\qw & \gate{T}    & \qw & \rstick{\omega^{x}\ket{x}}  
}


    }
    \subcaption{T-gate}
    \sublabel{fig-tgate}
    \centering
    \scalebox{1.0} {
      \Qcircuit @C=0.7em @R=0.7em @!R { \\
  \lstick{\ket{x}} &\qw & \gate{S}    & \qw & \rstick{e^{i\frac{\pi}{2}x}\ket{x}}  
}


    }
    \subcaption{S-gate}
    \sublabel{fig-sgate}
  \end{minipage}
  \begin{minipage}[t]{.45\textwidth}
    \vspace{5mm}
    \centering
    \hspace{5mm}
    \centering
    \scalebox{1.0} {
      \Qcircuit @C=0.7em @R=0.7em @!R { \\
  \lstick{\ket{x}} & \qw & \gate{H}    & \qw & \rstick{\frac{1}{\sqrt{2}} (\ket{0} + {(-1)}^{x}\ket{1})}  
}


    }
    \subcaption{H-gate}
    \sublabel{fig-hgate}
    \centering
    \scalebox{1.0} {
      \Qcircuit @C=0.7em @R=0.7em @!R { \\
  \lstick{\ket{x}} & \qw & \gate{T^{\dagger}}    & \qw & \rstick{\omega^{-x}\ket{x}}  
}


    }
    \subcaption{$T^{\dagger}$-gate}
    \sublabel{fig-tdgate}
    \centering
    \scalebox{1.0} {
      \Qcircuit @C=0.7em @R=0.7em @!R { \\
  \lstick{\ket{x}} & \qw & \gate{S^{\dagger}}    & \qw & \rstick{e^{-i\frac{\pi}{2}x}\ket{x}}  
}


    }
    \subcaption{$S^{\dagger}$-gate}
    \sublabel{fig-sdgate}
  \end{minipage}
  \caption{The H, CNOT, T, S (Clifford+T) basis gate set}
  \label{fig-basis}
\end{figure}

Figure~\ref{fig-basis} enumerates the elementary gate set consisting of the $H$-gate, $\mathit{CNOT}$,
$T$-gate, $T^{\dagger}$-gate (the inverse of the $T$-gate), $S$-gate, and $S^{\dagger}$-gate (the
inverse of the $S$-gate), known as the \emph{Clifford+T basis gate set}. As mentioned in
Chapter~\ref{Chap:Intro}, this gate set is enough to implement universal quantum computing,
including operations that have a quantum advantage~\cite{bib-solovay-kitaev}.
The implication, therefore, is that a finite set of gates is enough to express any quantum operation up
to a finite error with minimal overhead. This means the Clifford+T gate set can be used as a basis set for
fault-tolerant quantum computing, provided each of the gates can be implemented fault-tolerantly.

In the fault-tolerant paradigm, $T$-gates and $T^{\dagger}$-gates are much more expensive to implement than
$H$-gate,$CNOT$, $S$-gate, and $S^{\dagger}$-gates~\cite{bib-bravyi-kitaev-magic-distillation}. This means
that the number of T-gates is key to lowering the cost of the implementation. Hereafter, 
the set of the T-gate and the T$^{\dagger}$-gate are referred to as ``T-gates'', and the set of the S-gate and the
S$^{\dagger}$-gate ``S-gates'', unless otherwise noted.

When this set of gates is assembled into a network, the result is called a \emph{quantum circuit}. In a
quantum circuit, inputs come in from the left, gates are applied in order from left to right, and outputs
go out from the right. Quantum circuits also have \emph{output qubits}, which are the qubits that output
quantum states to be measured or used by other quantum circuits. 


\subsection{Quantum Gates and Circuits}
CNOT+T circuits consist only of the CNOT, T, and T$^\dagger$ gates. Such circuits can be represented by a path-sum, which captures the algebraic structure of the circuit. Two circuits sharing the same path-sum are functionally equivalent~\cite{pathsum2018}. This property enables a resynthesis process where T gates can be merged.

One complication in this process is the treatment of negated variables. As demonstrated in Fig.~[DCTODO], certain T-gates can be merged more easily using the Boolean Fourier transform of a function rather than by explicitly negating its inputs. A library of such decompositions aids this combination.

\subsection{Phase Oracles}
A phase oracle applies a phase $(-1)$ to any input that satisfies a Boolean function $f$. A common method of synthesizing phase oracles is via an ESOP representation, expressing the function as an XOR of product terms (ANDs) of variables. Given the mapping $|x\rangle|y\rangle \mapsto (-1)^{x \oplus y}|x\rangle|y\rangle$, such circuits can be synthesized using known quantum subroutines (see Fig.~[DCTODO]).


\section{Pseudo-Boolean Functions and the Boolean Fourier Transform}

One way to realize a Boolean AND is to implement what's called a {\it pseudo-Boolean}
representation~\cite{bib-barenco-elementary,bib-amy-cnot}, which uses a mix of
Boolean values, integer coefficients, and arithmetic to create a mapping $\{0,1\}^n \mapsto \{0,1\}$~\cite{bib-amy-rm}.

Before defining the concept mathematically, first observe Equation~\ref{eq-toff-bool} to see an example of a
pseudo-Boolean function. Here, real-valued coefficients are multiplied with Boolean functions of $x_a$, $x_b$,
and $y$, creating a real number, and those real numbers are then summed together using arithmetic operations
$+$ and $-$. Table~\ref{table-pseudo-toff} shows the values of $x_a \cdot x_b \cdot y$ with respect to
its components, where $T(f)$ is as defined in Equation~\ref{eq-toff-bool}. It can be proven by inspection of
Table~\ref{table-pseudo-toff} that the pseudo-Boolean expression in~\ref{eq-toff-bool} indeed implements
$x_a \cdot x_b \cdot y$. 
\begin{equation}
  \label{eq-toff-bool}
  \begin{aligned}
    &x_a \cdot x_b \cdot y &= &x_a + x_b + y \\\nonumber
    &&&+ 7(x_a \oplus x_b) + 7(x_a \oplus y) + 7(x_b \oplus y) \\\nonumber
    &&&+ (x_a \oplus x_b \oplus y)\\\nonumber
    &&= & (T_{x_a} + T_{x_b} + T_{y} + T_{x_a \oplus x_b} \\\nonumber
    &&&T_{x_a \oplus y} T_{x_b \oplus y} + T_{x_a \oplus x_b \oplus y})
  \end{aligned}
\end{equation}
\begin{table*}[t]
  \begin{minipage}{\textwidth}
    \begin{center}
      \scalebox{1.0} {
        \begin{tabular}{c|c|c|c|c|c|c|c|c|c|c}
          $x_a$ & $x_b$ & $y$ & $T_{x_a}$ & $T_{x_b}$ & $T_{y}$ & $T_{x_a \oplus x_b}$ & $T_{x_a \oplus y}$ & $T_{x_b \oplus y}$ & $T_{x_a \oplus x_b \oplus y}$ & $4x_a \cdot x_b \cdot y$\\\hline
          0     & 0     & 0   & 0         & 0         & 0       & 0                    & 0                  & 0                  & 0                             & 0            \\\hline
          0     & 0     & 1   & 0         & 0         & $1$   & 0                    & $7$             & $7$
                       & $1$                         & 0            \\\hline
          0     & 1     & 0   & 0         & $1$     & 0       & $7$               & $7$             & 0                  & $1$                         & 0            \\\hline
          0     & 1     & 1   & 0         & $1$     & $1$   & $7$               & 0                  & $7$             & 0                             & 0            \\\hline
          1     & 0     & 0   & $1$     & 0         & 0       & $7$               & 0                  & $7$             & $1$                         & 0            \\\hline
          1     & 0     & 1   & $1$     & 0         & $1$   & $7$               & $7$             & 0                  & 0                             & 0            \\\hline
          1     & 1     & 0   & $1$     & $1$     & 0       & 0                    & $7$             & $7$             & 0                             & 0            \\\hline
          1     & 1     & 1   & $1$     & $1$     & $1$   & 0                    & 0                  & 0                  & $1$                         & 4            \\\hline
        \end{tabular}
      }
      \caption{Truth table-like values of the pseudo-Boolean representation of $x_a \cdot x_b \cdot y$}
      \label{table-pseudo-toff}
    \end{center}
  \end{minipage}
  
\end{table*}

\begin{figure*}[t]
  \centering
  \scalebox{0.8} {
    \Qcircuit @C=0.5em @R=0.2em @!R { \\
                             &            &     &          &  & &                           &          &                                &          &                                &          &                             &          & \push{x_a}            &          &     & \\
          \lstick{\ket{x_a}} & \ctrl{3}   & \qw &          &  & & \qw                       & \ctrl{4} & \qw                            & \qw      & \qw                            & \ctrl{4} & \qw                         & \qw      & \gate{\mathrm{T}}     & \ctrl{2} & \qw & \\
                             &            &     &          &  & &                           &          &                                &          & \push{x_b}                     &          &                             &          & \push{x_a \oplus x_b} &          &     & \\
          \lstick{\ket{x_b}} & \ctrl{2}   & \qw & \push{=} &  & & \qw                       & \qw      & \qw                            & \ctrl{2} & \gate{\mathrm{T^{\dagger}}}    & \qw      & \qw                         & \ctrl{2} & \gate{\mathrm{T}}     & \targ    & \qw & \\
                             &            &     &          &  & & \push{y}                  &          & \push{x_a \oplus y}            &          & \push{x_a \oplus x_b \oplus y} &          & \push{x_b \oplus y}         &          &                       &          &     & \\
          \lstick{\ket{y}}   & \ctrl{-1}  & \qw &          &  & & \gate{\mathrm{T}}         & \targ    & \gate{\mathrm{T^{\dagger}}}    & \targ    & \gate{\mathrm{T}}              & \targ    & \gate{\mathrm{T^{\dagger}}} & \targ    & \qw                   & \qw      & \qw & \\
}

  }
  \caption{Implementing $x_a \cdot x_b \cdot y$ using CNOTs and T gates}
  \label{fig-toff-mark}
\end{figure*}


In general, a pseudo-Boolean function is a function such that
\begin{equation}
  \label{eq-pseudo-boolean}
  F(\mathbf{x}) = \sum_{\mathbf{k}} c_a \cdot f_k(\mathbf{x}), c_a \in \mathbb{R}, f_k : \{0,1\}^n \mapsto \{0,1\}
  F(\mathbf{x}) \in \{0,1\}
\end{equation}

In plain language, it is the arithmetic sum of the products of Boolean functions with real valued coefficients, such that
the values of the sum are in the set of binary values $\{0,1\}$~\cite{bib-barenco-elementary}. This
representation is also often referred to in the context of the phases of quantum circuits as the
{\it phase polynomial}~\cite{bib-amy-cnot}.

Phase polynomials and pseudo-Boolean functions important because phase polynomials can be calculated
using the {\it Boolean Fourier transform}, which is shown in Equation~\ref{eq-boolean-fourier}.
\begin{equation}
  \label{eq-boolean-fourier}
  \begin{aligned}
    f(\mathbf{x}) = \sum_{\mathbf{a} \neq 0} \hat{f}(\mathbf{a}) \cdot ( a_0 \cdot x_0 \oplus a_1 \cdot x_1 \cdots \oplus a_{n-1} \cdot x_{n-1}),\\\nonumber
    \hat{f}(\mathbf{a}) \in \mathbb{R} , a_0,a_1,\cdots a_{n-1} \in [0,1]
  \end{aligned}
\end{equation}

$\hat{f}(\mathbf{a})$ is a set of $2^{n}-1$ coefficients (one for each $\mathbf{a}$, except for the trivial
case $\mathbf{a} = 0$) associated with the pseudo-Boolean function, and
$( a_0 \cdot x_0 \oplus a_1 \cdot x_1 \cdots \oplus a_{n-1} \cdot x_{n-1}) $ are linear functions
that form a basis for the pseudo-Boolean function, and are maps $[0,1]^n \mapsto [0,1]$. $\hat{f}$ is referred to
as the {\it phase polynomial} of the quantum Boolean circuit.

Observe that a T-gate adds to the phase no matter which qubit it acts on.
For example, a T-gate acting on bit 0 creates the mapping $\ket{1}_0\ket{1}_1 \mapsto \ket{1}_0\ket{1}_1$ while
a T-gate acting on bit 1 also creates the same mapping. Therefore, $\hat{f}(\mathbf{a})$ sets the number of
phase gates, such as T-gates and T$^{\dagger}$ gates, that are used in the circuit. A Boolean function thus can
have a T-count up to $2^{n}-1$.

It is easy to see then that the $\hat{f}(\mathbf{a})$ that has the minimal number of odd entries for each
$\mathbf{a}$ also minimizes the T-count~\cite{bib-amy-rm}. Observe also, that the linear functions
$( a_0 \cdot x_0 \oplus a_1 \cdot x_1 \cdots \oplus a_{n-1} \cdot x_{n-1}) $ can be implemented
entirely by CNOT gates. This means that Equation~\ref{eq-boolean-fourier} can be realized entirely by
CNOT gates and phase gates, known as a {\it CNOT+T network}~\cite{bib-amy-cnot}. 

Optimizing the T-depth of such a CNOT+T network is done using the method in~\cite{bib-amy-matroid}.

\section{Phase Polynomial Calculation for the Three Qubit Case}
\label{Chap:Bool-pbool3q}

This section demonstrates how to calculate the phase polynomial for a three input Boolean function, using the
Boolean Fourier transform~\cite{bib-odonnell}. This section uses only a simplified version
of the Boolean Fourier transform introduced in~\cite{bib-odonnell}, but with notation taken from
\cite{bib-barenco-elementary} and~\cite{bib-amy-cnot}. The notation is chosen to more clearly demonstrate how
the equation may be implemented using quantum gates defined in the CNOT+T basis. 

First, the inner product to be used for the Boolean Fourier transform is defined below

\begin{equation}
  \label{eq-inner-prod}
  \langle f(\mathbf{x}) , g(\mathbf{x}) \rangle = \sum_{\mathbf{x}} \text{INNPROD}(f(\mathbf{x}),g(\mathbf{x})), \mathbf{x} \in \{0,1\}^n
\end{equation}
\begin{equation}  
  \text{INNPROD}(f(\mathbf{x}),g(\mathbf{x})) =
  \begin{cases}
    7 & \text{if $f(\mathbf{x})\neq g(\mathbf{x})$} \\
    1 & \text{if  $f(\mathbf{x})= g(\mathbf{x})$} 
  \end{cases}
\end{equation}

\cite{bib-odonnell} states that the {\it parity functions}, the set of exclusive sums of the input variables
$\chi(x_0,x_1,\cdots,x_n)$ shown in Equation~\ref{eq-bool-basis}, form an orthogonal basis with respect to the
inner product defined in Equation~\ref{eq-inner-prod}. In this equation, there is a factor of $\frac{1}{2^n}$ so
that $\langle f(\mathbf{x}), f(\mathbf{x}) \rangle = 1$, making the basis orthonormal.
\begin{equation}
  \label{eq-bool-basis}
  \begin{split}
  \chi(x_0,x_1,\cdots,x_n)_{\mathbf{a}} = a_0 \cdot x_0 \oplus a_1 \cdot x_1 \cdots \oplus a_{n-1} \cdot x_{n-1},\\
  a_0,a_1,\cdots a_{n-1} \in [0,1]
  \end{split}
\end{equation}

Because for each $\chi_{\mathbf{a}}$ and $\chi_{\mathbf{b}}$, there is
$\langle \chi_{\mathbf{a}}, \chi_{\mathbf{b}} \rangle = 0$, calculating $\hat{f}(\mathbf{a})$ for a specific
value of $\mathbf{a}$ simply involves taking the inner product of the full function $f(\mathbf{x})$ and the
basis state $\chi(\mathbf{x})_{\mathbf{a}}$ as shown in Equation~\ref{eq-fhat-prod}. This multiplication is
repeated for every value of $\mathbf{a}$ until all of $\hat{f}(\mathbf{a})$ is calculated.
\begin{equation}
  \label{eq-fhat-prod}
  \hat{f}(\mathbf{a}) = \langle f(\mathbf{x}), \chi(\mathbf{x})_{\mathbf{a}} \rangle
\end{equation}

\begin{example}
  As an example, let's try to derive the coefficients from Equation~\ref{eq-toff-bool}. Here
  $\mathbf{a} = (a_{x_a}, a_{x_b}, a_y)$.

  Because the trivial case ($\mathbf{a}= 000$) is not included, let's start with $\mathbf{a} = 100$ with
  $\chi_{100} = x_a$.
  Table~\ref{table-ex-100} enumerates its values in truth table form for convenience, as well as the
  value of each of the $(-1)^{f(\mathbf{x})} (-1)^{\chi_{001}(\mathbf{x})}$ terms in the inner product
  $\langle f(\mathbf{x}), \chi_{001} \rangle$

  \begin{table}[h]
    \begin{center}
      \begin{tabular}{c|c|c|c|c|c}
        \hline
        $x_a$ & $x_b$ & $y$ & $f(\mathbf{x})$ & $\chi_{100}$ & INNPROD($\chi_{100},f(\mathbf{x}$) \\\hline
        0 & 0 & 0 & 0 & 0 & 1\\\hline
        0 & 0 & 1 & 0 & 0 & 1\\\hline
        0 & 1 & 0 & 0 & 0 & 1\\\hline
        0 & 1 & 1 & 0 & 0 & 1\\\hline
        1 & 0 & 0 & 0 & 1 & 7\\\hline
        1 & 0 & 1 & 0 & 1 & 7\\\hline
        1 & 1 & 0 & 0 & 1 & 7\\\hline      
        1 & 1 & 1 & 1 & 1 & 1\\\hline
      \end{tabular}
      \caption{The value of $\chi_{100}$ and the inner product $\langle f(\mathbf{x}), \chi_{100} \rangle$}
      \label{table-ex-100}
    \end{center}
  \end{table}

  
  The terms in the right most column can then be summed just like in Equation~\ref{eq-inner-prod}:
  \begin{align}
      &\hat{f}(100) = &\langle f(\mathbf{x}), \chi_{100} \rangle = &( 1 + 1 + 1 + 1 - 1 - 1 - 1 + 1) \\\nonumber
      &&& = ( 2 ) = 1\nonumber\\\nonumber
  \end{align}
  This then gives:
  \begin{equation}
    f(\mathbf{x}) = x_a + \sum_{\mathbf{a} \neq 000,100} \hat{f}(\mathbf{a}) \cdot \chi_{\mathbf{a}}
  \end{equation}

  In the interest of conciseness, the rest of the calculation is omitted. But the reader can easily perform the
  rest to verify its equivalence with Equation~\ref{eq-toff-bool}.
\end{example}

The pseudo-Boolean function from Eq.~\ref{eq-toff-bool} is implemented as a circuit in Fig~\ref{fig-toff-mark}.
