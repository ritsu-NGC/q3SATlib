
\section{Proposal}
\label{Pro}
\subsection{Overview}

In this section, we present the overall flow of our proposal, depicted in Fig.~\ref{fig-flow}. The blue boxes represent precomputation, while tan boxes represent runtime.

\begin{itemize}
\item \textbf{(Generate 3bit phase polynomials)} : Precompute library of phase polynomials of Boolean functions of
  3 bits or fewer using the Boolean Fourier transform.
\item \textbf{(Synthesize 3bit phase subcircuits)} : The phase polynomials are then sequenced using matroid
  partitioning from~\cite{bib-amy-matroid} and synthesized as subcircuits. The results are stored in hard disk,
  to be loaded into memory at runtime.
\item \textbf{(Decompose $f$ into ESOP expression)} : At runtime, a Boolean function is read in and converted to
  an ESOP expression.
\item \textbf{(Minimize the number of $n \geq 4$ cubes in ESOP)} : The ESOP is optimized by adapting
  EXORCISM-4~\cite{bib-exorcism} to reduce the number of cubes of more than 4 literals. This method is explained
  in Sec.~\ref{Pro:Minimize}.
\item \textbf{(Map the $n \leq 3$ cubes to the 3 bit library)} : Takes all cubes of $n \leq 3$ and maps them to 
  appropriate functions from the 3 bit library, inserting the corresponding subcircuit into the phase oracle.
\item \textbf{(Synthesize $n \geq 4$ cubes)} : Synthesizes all cubes of $n \geq 4$ using the method described in
  Sec.~\ref{Pro:n4}.
\item \textbf{(Combine $n \geq 4$ and $n \leq 3$)} : The results of the above two steps are combined into one
  phase oracle.
\item \textbf{(Optimize for T-count and T-depth)} : The combined circuit is optimized for T-count and T-depth using
  the T-par algorithm~\cite{bib-amy-matroid}. Further detail of this step is in Sec.~\ref{Pro:Tpar}.
\end{itemize}
\begin{figure}[t]
  \centering
  \scalebox{1.0} {
    %% \begin{tikzpicture}[
%%     box/.style={draw, rounded corners, fill=orange!20, align=center, font=\small, minimum width=4.2cm, minimum height=1.1cm, rotate=90, text width=3.6cm, inner sep=6pt},
%%     bigbox/.style={draw, rounded corners, fill=orange!20, align=center, font=\small, minimum width=5.3cm, minimum height=1.3cm, inner sep=8pt, rotate=90, text width=4.6cm},
%%     bluebox/.style={draw, rounded corners, fill=blue!20, align=center, font=\small, minimum width=2.9cm, minimum height=1.6cm, rotate=90, text width=2.5cm},
%%     arrow/.style={-Stealth, thick, color=teal!80!black},
%%     bendarrow/.style={-Stealth, thick, color=teal!80!black, line width=5pt, rounded corners=15pt}
%% ]

%% % Place nodes in a straight horizontal (east-west) line with rotated text
%% \node[box] (decompose) at (0,0) {Decompose $f$ into\\ESOP expression};
%% \node[box] (minimize) at (3,0) {Minimize the \\ number of $n \geq 4$ cubes in ESOP};
%% \node[box] (map3qubit) at (6.2,0) {Map the $n \leq 3$ cubes to the \textcolor{blue}{3 qubit} library};
%% \node[box] (optimize) at (9.7,0) {Optimize for T-count and T-depth};
%% \node[bigbox] (synthesize) at (13.4,0) {Synthesize $n \geq 4$};

%% % Blue box above "map3qubit"
%% \node[bluebox] (oracle) at (6.2,2.6) {Generate 3bit\\Boolean Oracle\\Library};

%% % Arrows (horizontal)
%% \draw[arrow] (decompose) -- (minimize);
%% \draw[arrow] (minimize) -- (map3qubit);
%% \draw[arrow] (map3qubit) -- (optimize);
%% \draw[arrow] (optimize) -- (synthesize);

%% % Bent arrow from blue box to map3qubit
%% \draw[bendarrow] (oracle.south) -- ++(0,-0.7) -- (map3qubit.north);

%% \end{tikzpicture}
\begin{tikzpicture}[
    box/.style={draw, rounded corners, fill=orange!20, align=center, font=\scriptsize, minimum width=3cm, minimum height=0.8cm, text width=2.8cm, inner sep=2pt},
    bigbox/.style={draw, rounded corners, fill=orange!20, align=center, font=\scriptsize, minimum width=3cm, minimum height=1.0cm, inner sep=2.8pt, text width=5.4cm},
    bluebox/.style={draw, rounded corners, fill=blue!20, align=center, font=\scriptsize, minimum width=3cm, minimum height=1.0cm, text width=2.8cm, inner sep=2pt},
    arrow/.style={-Stealth, thick, color=teal!80!black},
    bendarrow/.style={-Stealth, thick, color=teal!80!black}
]

% Place main vertical nodes
\node[box] (decompose) at (0,0) {Decompose $f$ into ESOP expression};
\node[box, below=0.8cm of decompose] (minimize) {Minimize the number of $n \geq 4$ cubes in ESOP};
\node[box, below=0.8cm of minimize] (synthesize) {Synthesize $n \geq 4$ cubes};

% Synthesize to the left of map3qubit, aligned with minimize
\node[box, left=1cm of synthesize] (map3qubit) at ($(synthesize)+( -1,0 )$) {Map the $n \leq 3$ cubes to the \textcolor{blue}{3 qubit} library};

% 'Combine' below the midpoint of synthesize and map3qubit
\path let \p1 = (map3qubit), \p2 = (synthesize) in
    coordinate (combinecoord) at ({0.5*(\x1+\x2)}, {\y2-1.2cm});
\node[box] (combine) at (combinecoord) {Combine $n \geq 4$ and $n \leq 3$};

% Optimize below combine
\node[box, below=0.8cm of combine] (optimize) {Optimize for T-count and T-depth};

% Blue box left of decompose
\node[bluebox, left=0.5cm of decompose] (oracle) {Generate 3bit phase polynomials};
\node[bluebox, below=0.8cm of oracle] (oraclecirc) {Synthesize 3bit phase subcircuits};

% Arrows
\draw[arrow] (decompose) -- (minimize);
\draw[arrow] (minimize) -- (map3qubit);
\draw[arrow] (minimize) -- (synthesize);
\draw[arrow] (synthesize) -- (combine);
\draw[arrow] (map3qubit) -- (combine);
\draw[arrow] (combine) -- (optimize);
\draw[arrow] (oracle) -- (oraclecirc);
\draw[arrow] (oraclecirc) -- (map3qubit);
% Bent arrow from blue box to map3qubit
% \draw[arrow] (oracle.south) |- (map3qubit.north);

\end{tikzpicture}
%% \begin{tikzpicture}[
%%     box/.style={draw, rounded corners, fill=orange!20, align=center, font=\scriptsize, minimum width=3cm, minimum height=0.8cm, text width=2.8cm, inner sep=2pt},
%%     bigbox/.style={draw, rounded corners, fill=orange!20, align=center, font=\scriptsize, minimum width=3cm, minimum height=1.0cm, inner sep=2.8pt, text width=5.4cm},
%%     bluebox/.style={draw, rounded corners, fill=blue!20, align=center, font=\scriptsize, minimum width=3cm, minimum height=1.0cm, text width=2.8cm, inner sep=2pt},
%%     arrow/.style={-Stealth, thick, color=teal!80!black},
%%     bendarrow/.style={-Stealth, thick, color=teal!80!black}
%% ]

%% % Place nodes in a straight vertical line
%% \node[box] (decompose) at (0,0) {Decompose $f$ into ESOP expression};
%% \node[box, below=0.8cm of decompose] (minimize) {Minimize the number of $n \geq 4$ cubes in ESOP};
%% \node[box, below=0.8cm of minimize] (map3qubit) {Map the $n \leq 3$ cubes to the \textcolor{blue}{3 qubit} library};
%% \node[box, below=0.8cm of minimize, left=1cm of map3qubit] (synthesize) {Synthesize $n \geq 4$};
%% \node[box, below=0.8cm of synthesize, left=0.5cm of synthesize] (combine) {Combine $n \geq 4$ and $n \leq 3$};
%% \node[box, below=0.8cm of combine] (optimize) {Optimize for T-count and T-depth};


%% % Blue box left of "minimize"
%% \node[bluebox, left=0.5cm of decompose] (oracle) {Generate 3bit\\Boolean Oracle\\Library};

%% % Arrows (vertical)
%% \draw[arrow] (decompose) -- (minimize);
%% \draw[arrow] (minimize) -- (map3qubit);
%% \draw[arrow] (minimize) -- (synthesize);
%% \draw[arrow] (synthesize) -- (combine);
%% \draw[arrow] (map3qubit) -- (combine);
%% \draw[arrow] (combine) -- (synthesize);

%% % Bent arrow from blue box to map3qubit
%% \draw[bendarrow] (oracle.south) |- (map3qubit.west);

%% \end{tikzpicture}

  }
  \caption{The overall flow of the proposed method}
  \label{fig-flow}
\end{figure}

\subsection{Minimize the number of $n \geq 4$ cubes in ESOP}
\label{Pro:Minimize}

To minimize the number of $n \geq 4$ cubes in the ESOP, the proposal adapts EXORCISM-4~\cite{bib-exorcism}.
EXORCISM-4 is a heuristic method of minimizing the number of cubes in a given ESOP expression. Extended
discussion of its details is defered to the cited paper. 

We modify it for our purposes by changing
the cost function \texttt{CountCubesInExactPseudoKro} that EXORCISM-4 uses for its heuristic, which uses a plain sum
of cubes to assess the cost of the arrangement that its heuristic encounters. We introduce a cost function
that instead uses a weighted sum of the cubes, where each cube is weighted by its T-count (i.e. a 2 literal cube
has weight 0) for $n \leq 3$ and $2^n$ for $n \geq 4$ (its T-count when synthesized according to the method in
Sec.~\ref{Pro:n4}). This penalizes terms of $n \geq 4$, while terms of $n < 3$ are free.


\subsection{Synthesize $n \geq 4$ cubes}
\label{Pro:n4}

To synthesize cubes of $n \geq 4$ variables, we utilize the construction in Fig.~\ref{fig-phase-n4}. Here,
$U_F$ and $U_F^{\dagger}$ are $F$-controlled-NOT gates, meaning that they apply X to the target bit when
the Boolean function $F$ is true. We add an ancilla bit and use the relative-phase Boolean decomposition
method from~\cite{bib-clarino-lut} to enable the implementation of such $F$-controlled-NOT gates with a
reduced number of T-gates. We defer detailed discussion of that method to the cited paper. This resulting
phase oracle has T-count $2^{n}$, and T-depth $2^{n+1}$. Note that this means that larger cubes can
dominate the T-depth of a given circuit, so it is imperative that the minimum of them be found for
any given Boolean function.

\begin{figure}[t]
  \centering
  \scalebox{1.0} {
    \Qcircuit @C=1em @R=1em {
  \lstick{\ket{x_1}} & \qw               & \multigate{3}{U_F} & \qw           & \multigate{3}{U_F^{\dagger}} & \qw \\
  \lstick{\ket{x_2}} & \qw               & \ghost{U_F}        & \qw           & \ghost{U_F^{\dagger}}        & \qw \\
                     & \push{\vdots}     &                    & \push{\vdots} &                              & \push{\vdots} \\
  \lstick{\ket{x_n}} & \qw               & \ghost{U_F}       & \qw            & \ghost{U_F^{\dagger}}        & \qw \\
  \lstick{\ket{y}}   & \qw               & \targ\qwx[-1]     & \gate{Z}       & \targ\qwx[-1]                & \qw
}

  }
  \caption{Synthesize a phase oracle for a cube of $n \geq 4$}
  \label{fig-phase-n4}
  \vspace{-0.5cm}
\end{figure}

\subsection{Optimize for T-count and T-depth}
\label{Pro:Tpar}

To optimize for T-count and T-depth, we utilize the T-par algorithm from~\cite{bib-amy-matroid}. This is a
variant of the matroid partitioning algorithm that deals with optimization of Clifford+T circuits by heuristically
separating the circuit into CNOT+T partitions along the H-gates in the circuit. This deals with the circuits
synthesized in Sec.~\ref{Pro:n4} as well as the CNOT+T circuits the $n \leq 3$ processing step produces. We defer
more detailed discussion to the cited paper.



