\documentclass[conference]{IEEEtran}

\usepackage{float}
\usepackage{cite}
\usepackage{caption}
\usepackage{subcaption}
\usepackage{amsmath,amssymb,amsfonts,amstext}
% \usepackage{algorithmic}
\usepackage[dvipdfmx]{graphicx}
\usepackage{textcomp}
\usepackage{xcolor}
\usepackage{colortbl}
\usepackage[T1]{fontenc}
\usepackage{mdframed}
\usepackage{algorithm, algpseudocode} % texlive-science
\usepackage[braket, qm]{qcircuit}
\usepackage{newtxtext}
\usepackage[normalem]{ulem}

\usepackage{nameref}
\usepackage[braket, qm]{qcircuit}
\usepackage{tikz}
%
\newcommand*\rtofstarg{\tikz[baseline=(char.base)]{
            \node[shape=circle,double,draw,inner sep=1pt,minimum width=16pt] (char) {S};}}
\newcommand*\rtofsdgtarg{\tikz[baseline=(char.base)]{
            \node[shape=circle,double,draw,inner sep=1pt] (char) {S$^{\dagger}$};}}
\newcommand*\rtoftarg{\tikz[baseline=(char.base)]{
            \node[shape=circle,draw,inner sep=-0.5pt,fill=none] (char) {$\oplus$};}}
\newcommand*\rtofsctrl{\tikz[baseline=(char.base)]{
            \node[shape=circle,double,draw,inner sep=1pt,fill=black, minimum width=3pt] (char) {};}}
\newcommand*\rtofsctrlo{\tikz[baseline=(char.base)]{
            \node[shape=circle,double,draw,inner sep=1pt,fill=white, minimum width=3pt] (char) {};}}
%\newcommand*\rtofsctrlo{\tikz[baseline=(char.base)]{
%            \node[shape=circle,double,draw,inner sep=1pt,fill=white, minimum width=3pt] (char) {};}}
\newcommand*\dashedwire{\tikz{
            \draw[dashed] (0,0) -- (0.5,0); }}

\newcommand*\ctrlrtofs[1]{\push{\rtofsctrl}\qwx[#1]\qw}
\newcommand*\ctrlortofs[1]{\push{\rtofsctrlo}\qwx[#1]\qw}
\newcommand*\targrtofs{\push{\rtofstarg}\qw}
\newcommand*\targrtof{\push{\rtoftarg}\qw}


\begin{document}

\title{Synthesis of Phase Oracles Using CNOT+T Circuits and ESOP Minimization}

\author{
    \IEEEauthorblockN{Your Name}
    \IEEEauthorblockA{Your Institution \\
    Email: your@email.com}
}

\maketitle

\begin{abstract}
Phase oracles are essential components of quantum algorithms such as Grover's algorithm. These circuits apply a phase of $-1$ to an input state when a particular Boolean function $f$ is true. For functions of fewer than 3 variables, these are well known to be exactly implementable using only the CNOT, T, and T$^\dagger$ gates. Amy~\cite{amy2018} established that this class of circuits can be provably optimized for T-count and T-depth. We identify a class of Boolean functions of 4 or more variables which can be synthesized using such a circuit, by way of a library of circuits that implement phase oracles of 3 or fewer variables. We then use this to synthesize Boolean functions of 4 or more variables, by minimizing the number of ESOP terms of more than 3 variables. We then synthesize the rest using a modification of a previously proposed method to synthesize arbitrary Boolean functions as single-target Boolean gates. We compare the results against Qiskit's \texttt{PhaseOracle} method.
\end{abstract}

\section{Introduction}
One of quantum computing's most significant results is the development of Grover's algorithm. This algorithm performs an unstructured search in the theoretical lower bound of such a search. Various applications have been proposed for such an algorithm, including the acceleration of solving NP-hard problems.

Grover's algorithm uses a phase oracle, a quantum circuit that applies a phase of $-1$ when the Boolean function it implements is true, to increase the probability of returning a fulfilling assignment.

Phase oracles are interesting because a restricted class of them can be implemented using only the T, T$^\dagger$, and CNOT gates (hereafter referred to as CNOT+T). In fact, phase oracles of 3 or fewer variables can all be implemented exactly using this set of gates, and these circuits can be calculated exactly using the Boolean Fourier transform. Amy~\cite{amy2018} demonstrates that CNOT+T circuits can be optimized for both T-count and T-depth~\cite{amy_matroid}.

However, in general, it is impossible to implement Boolean functions of 4 or more variables using only this gate set. There will inevitably be a need to use a Hadamard (H) gate. The addition of this gate makes optimal solutions non-unique, and several methods to optimize for T-count and T-depth when including the H gate are only heuristic.

Phase oracles are well known to be efficiently synthesizable from an Exclusive Sum of Products (ESOP) expression. Since each product maps to a Boolean function, it is easy to see that any such function expressible with product terms of at most 3 variables can be implemented as a CNOT+T circuit. We utilize this observation in our proposal.

In this work, we propose a solution that utilizes CNOT+T circuits to reduce the T-count and T-depth for general Boolean functions of arbitrary variable count. Our contributions include:

\begin{itemize}
    \item A precomputed library of circuits that implement all Boolean functions of 3 variables or fewer.
    \item A synthesis method that uses ESOP minimization to limit the number of product terms involving 4 or more variables.
\end{itemize}

We find that our solution... [DCTODO: complete with results]

\section{Preliminaries}

\subsection{Quantum Gates and Circuits}
CNOT+T circuits consist only of the CNOT, T, and T$^\dagger$ gates. Such circuits can be represented by a path-sum, which captures the algebraic structure of the circuit. Two circuits sharing the same path-sum are functionally equivalent~\cite{pathsum2018}. This property enables a resynthesis process where T gates can be merged.

One complication in this process is the treatment of negated variables. As demonstrated in Fig.~[DCTODO], certain T-gates can be merged more easily using the Boolean Fourier transform of a function rather than by explicitly negating its inputs. A library of such decompositions aids this combination.

\subsection{Phase Oracles}
A phase oracle applies a phase $(-1)$ to any input that satisfies a Boolean function $f$. A common method of synthesizing phase oracles is via an ESOP representation, expressing the function as an XOR of product terms (ANDs) of variables. Given the mapping $|x\rangle|y\rangle \mapsto (-1)^{x \oplus y}|x\rangle|y\rangle$, such circuits can be synthesized using known quantum subroutines (see Fig.~[DCTODO]).

\section{Proposal}

A sum-of-paths representation of a Boolean function can be derived from the Boolean Fourier decomposition~\cite{odonnell}. However, this decomposition forms a complete basis for functions with 3 variables or fewer when using the CNOT+T gate set. We demonstrate that composing such decompositions enables the exact implementation of a subset of $k$-variable Boolean functions.

We use a modified Boolean Fourier decomposition approach, based on~\cite{amy_duality}, to construct our 3-qubit-or-less library. Our synthesis is ancilla-free and schedules T gates following~\cite{amy2018} to minimize T-depth.

We then apply a modified form of the relative phase function synthesis described in~\cite{amy_relphase} to complete our circuit synthesis.

% DCTODO: Add Evaluation, Results, and Conclusion Sections

\bibliographystyle{IEEEtran}
\bibliography{ref}

\end{document}
