\documentclass[conference]{IEEEtran}

\usepackage{float}
\usepackage{cite}
\usepackage{caption}
\usepackage{subcaption}
\usepackage{amsmath,amssymb,amsfonts,amstext}
% \usepackage{algorithmic}
\usepackage[dvipdfmx]{graphicx}
\usepackage{textcomp}
\usepackage{xcolor}
\usepackage{colortbl}
\usepackage[T1]{fontenc}
\usepackage{mdframed}
\usepackage{algorithm, algpseudocode} % texlive-science
\usepackage[braket, qm]{qcircuit}
\usepackage{newtxtext}
\usepackage[normalem]{ulem}

\usepackage{nameref}
\usepackage[braket, qm]{qcircuit}
\usepackage{tikz}
%
\newcommand*\rtofstarg{\tikz[baseline=(char.base)]{
            \node[shape=circle,double,draw,inner sep=1pt,minimum width=16pt] (char) {S};}}
\newcommand*\rtofsdgtarg{\tikz[baseline=(char.base)]{
            \node[shape=circle,double,draw,inner sep=1pt] (char) {S$^{\dagger}$};}}
\newcommand*\rtoftarg{\tikz[baseline=(char.base)]{
            \node[shape=circle,draw,inner sep=-0.5pt,fill=none] (char) {$\oplus$};}}
\newcommand*\rtofsctrl{\tikz[baseline=(char.base)]{
            \node[shape=circle,double,draw,inner sep=1pt,fill=black, minimum width=3pt] (char) {};}}
\newcommand*\rtofsctrlo{\tikz[baseline=(char.base)]{
            \node[shape=circle,double,draw,inner sep=1pt,fill=white, minimum width=3pt] (char) {};}}
%\newcommand*\rtofsctrlo{\tikz[baseline=(char.base)]{
%            \node[shape=circle,double,draw,inner sep=1pt,fill=white, minimum width=3pt] (char) {};}}
\newcommand*\dashedwire{\tikz{
            \draw[dashed] (0,0) -- (0.5,0); }}

\newcommand*\ctrlrtofs[1]{\push{\rtofsctrl}\qwx[#1]\qw}
\newcommand*\ctrlortofs[1]{\push{\rtofsctrlo}\qwx[#1]\qw}
\newcommand*\targrtofs{\push{\rtofstarg}\qw}
\newcommand*\targrtof{\push{\rtoftarg}\qw}


\begin{document}

\title{Synthesis of Phase Oracles Using CNOT+T Circuits and ESOP Minimization}

\author{
    \IEEEauthorblockN{Your Name}
    \IEEEauthorblockA{Your Institution \\
    Email: your@email.com}
}

\maketitle

\begin{abstract}
Phase oracles are essential components of quantum algorithms such as Grover's algorithm. These circuits apply a phase of $-1$ to an input state when a particular Boolean function $f$ is true. For functions of fewer than 3 variables, these are well known to be exactly implementable using only the CNOT, T, and T$^\dagger$ gates. Amy~\cite{amy2018} established that this class of circuits can be provably optimized for T-count and T-depth. We identify a class of Boolean functions of 4 or more variables which can be synthesized using such a circuit, by way of a library of circuits that implement phase oracles of 3 or fewer variables. We then use this to synthesize Boolean functions of 4 or more variables, by minimizing the number of ESOP terms of more than 3 variables. We then synthesize the rest using a modification of a previously proposed method to synthesize arbitrary Boolean functions as single-target Boolean gates. We compare the results against Qiskit's \texttt{PhaseOracle} method.
\end{abstract}

\section{Introduction}
One of quantum computing's most significant results is the development of Grover's algorithm. This algorithm performs an unstructured search in the theoretical lower bound of such a search. Various applications have been proposed for such an algorithm, including the acceleration of solving NP-hard problems.

Grover's algorithm uses a phase oracle, a quantum circuit that applies a phase of $-1$ when the Boolean function it implements is true, to increase the probability of returning a fulfilling assignment.

Phase oracles are interesting because a restricted class of them can be implemented using only the T, T$^\dagger$, and CNOT gates (hereafter referred to as CNOT+T). In fact, phase oracles of 3 or fewer variables can all be implemented exactly using this set of gates, and these circuits can be calculated exactly using the Boolean Fourier transform. Amy et al.~\cite{amy-meet-in-middle} demonstrates that CNOT+T circuits can be optimized for both T-count and T-depth~\cite{bib-amy-matroid}.

However, in general, it is impossible to implement Boolean functions of 4 or more variables using only this gate set. There will inevitably be a need to use a Hadamard (H) gate. The addition of this gate makes optimal solutions non-unique, and several methods to optimize for T-count and T-depth when including the H gate are only heuristic.

Phase oracles are well known to be efficiently synthesizable from an Exclusive Sum of Products (ESOP) expression. Since each product maps to a Boolean function, it is easy to see that any such function expressible with product terms of at most 3 variables can be implemented as a CNOT+T circuit. We utilize this observation in our proposal.

In this work, we propose a solution that utilizes CNOT+T circuits to reduce the T-count and T-depth for general Boolean functions of arbitrary variable count. Our contributions include:

\begin{itemize}
    \item A precomputed library of circuits that implement all Boolean functions of 3 variables or fewer.
    \item A synthesis method that uses ESOP minimization to limit the number of product terms involving 4 or more variables.
\end{itemize}

We find that our solution... [DCTODO: complete with results]


\begin{figure}[t]
  \begin{minipage}{0.45\linewidth}
    \centering
    \scalebox{1.0} {
      \input{img/Sgate}
    }
    \subcaption{S-gate}
    \sublabel{fig-sgate}
    \scalebox{1.0} {
      \Qcircuit @C=0.7em @R=0.7em @!R { \\
  \lstick{\ket{x}} & \qw & \gate{S^{\dagger}}    & \qw & \rstick{e^{-i\frac{\pi}{2}x}\ket{x}}  
}


    }
    \subcaption{S$^{\dagger}$-gate}
    \sublabel{fig-sdgate}
    \scalebox{1.0}{
      \Qcircuit @C=0.7em @R=0.7em @!R { \\
  \lstick{\ket{x}} & \qw & \ctrl{1} & \qw & \rstick{\ket{x}} \\
  \lstick{\ket{y}} & \qw & \targ    & \qw & \rstick{\ket{x \oplus y}}  \\
}


    }
    \subcaption{CNOT-gate}
    \sublabel{fig-cnot}
    \scalebox{1.0} {
      \Qcircuit @C=0.7em @R=0.7em @!R { \\
  \lstick{\ket{x}} &\qw & \gate{Z}    & \qw & \rstick{e^{i \pi x}\ket{x}}  
}


    }
    \subcaption{Z-gate}
    \sublabel{fig-zgate}
  \end{minipage}
  \begin{minipage}{0.45\linewidth}
  \centering
    \scalebox{1.0} {
      \Qcircuit @C=0.7em @R=0.7em @!R { \\
  \lstick{\ket{x}} &\qw & \gate{T}    & \qw & \rstick{\omega^{x}\ket{x}}  
}


    }
    \subcaption{T-gate}
    \sublabel{fig-tgate}
    \scalebox{1.0} {
      \Qcircuit @C=0.7em @R=0.7em @!R { \\
  \lstick{\ket{x}} & \qw & \gate{T^{\dagger}}    & \qw & \rstick{\omega^{-x}\ket{x}}  
}


    }
    \subcaption{$T^{\dagger}$-gate}
    \sublabel{fig-tdgate}
    \scalebox{1.0} {
      \Qcircuit @C=0.7em @R=0.7em @!R { \\
  \lstick{\ket{x}} & \qw & \gate{H}    & \qw & \rstick{\frac{1}{\sqrt{2}} (\ket{0} + {(-1)}^{x}\ket{1})}  
}


    }
    \subcaption{H-gate}
    \sublabel{fig-hgate}
    \scalebox{1.0} {
      \Qcircuit @C=0.7em @R=0.7em @!R { \\
  \lstick{\ket{x}} &\qw & \gate{X}    & \qw & \rstick{\ket{\bar{x}}}  
}


    }
    \subcaption{X-gate}
    \sublabel{fig-xgate}
  \end{minipage}
  \caption{Quantum gates}
  \label{fig-gates}
\end{figure}


\section{Preliminaries}

\subsection{Quantum Bits and Quantum Gates}
\label{Chap:Pre-qubits}
Quantum computers internally represent data as \emph{qubits}, which are quantum systems that can
take on the quantum states $\ket{0}$ and $\ket{1}$,

Additionally, qubits can also exist in quantum states that are a linear combination of $\ket{0}$ and
$\ket{1}$, which is called {\it superposition}:
\begin{equation}
\ket{\psi} = \alpha_0 \ket{0} + \alpha_1 \ket{1}, \alpha_0, \alpha_1 \in \mathbb{C}
\end{equation}

Because these coefficients $\alpha_0$ and $\alpha_b$ are complex numbers, they can also be expressed as
a complex exponential $a\cdot e^{i \theta}$.
$\theta$ in this exponential is often referred to as a {\it phase}.

Qubits can be taken together as tensor products to create multiqubit states.
\begin{equation}
\label{eq-multi-qb}
\ket{\psi} = \ket{\psi}_0 \otimes \ket{\psi}_1 =(\alpha_{00} \ket{0}_0 + \alpha_{01} \ket{1}_0) \otimes
(\alpha_{10} \ket{0}_1 + \alpha_{11} \ket{1}_1)
\end{equation}
Note in Equation~\ref{eq-multi-qb} that In particular, the tensor product of several basis states is known as a \emph{computational basis state},
and is written as a bit string consisting of the values in the component tensor multiplicands. 
\begin{equation}
\ket{x_0}\otimes\ket{x_1}\otimes\cdots\otimes\ket{x_{n-1}}=\ket{x_0 x_1 \cdots x_{n-1}}=\ket{\mathbf{x}},
\mathbf{x} \in \{0,1\}^n
\end{equation}

These computational basis states can similarly be in linear superposition.
\begin{equation}
\ket{\psi} = \sum_{\mathbf{k} \in \{0,1\}^n} \alpha_{\mathbf{k}}\ket{\mathbf{k}},
\sum_{\mathbf{k} \in \{0,1\}^n} \alpha_{\mathbf{k}} = 1
\end{equation}

{\it Quantum gates} describe transformations in qubits. The main ones that concern this work are demonstrated
in Fig.~\ref{fig-gates}. The $T/T^{\dagger}$, $S/S^{\dagger}$, and $Z$ gates will be referred to as {\it phase gates}.
All of the gates in Fig.~\ref{fig-gates} can in turn be composed into {\it quantum circuits}. 

\subsection{Phase Oracles and Phase Polynomials}
A {\it phase oracle} is one such quantum circuit that implements a Boolean function as a quantum circuit, implementing
a mapping of a computational basis state $\ket{\mathbf{x}} \mapsto e^{i \pi f(\mathbf{x})}\ket{\mathbf{x}}$, where $f(\mathbf{x})$
is a Boolean function $\{0,1\}^n \rightarrow \{0,1\}$. Phase oracles are key elements in algorithms such as
Grover's Algorithm.

Previous research has demonstrated that some phase oracles can be implemented using only CNOT and phase gates
(CNOT+T)~\cite{bib-amy-cnot}. When done in such a manner, $f(\mathbf{x})$ can be expressed as a {\it phase polynomial},
defined in Eq.~\ref{eq-boolean-fourier}.

\begin{equation}
  \begin{aligned}
    \label{eq-boolean-fourier}
    f(\mathbf{x}) = \sum_{\mathbf{k} \neq 0} \hat{f}(\mathbf{k}) \cdot \chi_k(\mathbf{x}), \\\nonumber
    \hat{f}(\mathbf{k}) \in \mathbb{R}
  \end{aligned}
\end{equation}

Where $\chi_k(\mathbf{x}) = ( k_0 x_0 \oplus k_1 x_1 \oplus \cdots \oplus k_{n-1} x_{n-1})$ for some
$\mathbf{k} \in \{0,1\}^{n}$.

Eq.~\ref{eq-toff-bool} demonstrates one such phase polynomial. It can be verified from Table~\ref{table-pseudo-toff} that
this is indeed true.

\begin{equation}
  \label{eq-toff-bool}
  \begin{aligned}
    &x_a \cdot x_b \cdot y = \frac{1}{4}x_a + \frac{1}{4}x_b + \frac{1}{4}y - \frac{1}{4}(x_a \oplus x_b) \\
    &\qquad -\frac{1}{4}(x_a \oplus y) - \frac{1}{4}(x_b \oplus y) + \frac{1}{4}(x_a \oplus x_b \oplus y)
  \end{aligned}
\end{equation}

\def\arraystretch{1.3}
\begin{table*}[h]
  \begin{minipage}{\textwidth}
    \begin{center}
      \scalebox{1.0} {
        \begin{tabular}{c|c|c|c|c|c|c|c|c|c|c}\hline
          $x_a$ & $x_b$ & $y$ & $T_{x_a}$         & $T_{x_b}$         & $T_{y}$         & $T_{x_a \oplus x_b}$ & $T_{x_a \oplus y}$ & $T_{x_b \oplus y}$ & $T_{x_a \oplus x_b \oplus y}$ & $x_a \cdot x_b \cdot y$\\\hline
          0     & 0     & 0   & 0                 & 0                 & 0               & 0                    & 0                  & 0                  & 0                             & 0            \\\hline
          0     & 0     & 1   & 0                 & 0                 & $\frac{1}{4}$   & 0                    & $-\frac{1}{4}$     & $-\frac{1}{4}$     & $\frac{1}{4}$                 & 0            \\\hline
          0     & 1     & 0   & 0                 & $\frac{1}{4}$     & 0               & $-\frac{1}{4}$       & 0                  & $-\frac{1}{4}$     & $\frac{1}{4}$                 & 0            \\\hline
          0     & 1     & 1   & 0                 & $\frac{1}{4}$     & $\frac{1}{4}$   & $-\frac{1}{4}$       & $-\frac{1}{4}$     & 0                  & 0                             & 0            \\\hline
          1     & 0     & 0   & $\frac{1}{4}$     & 0                 & 0               & $-\frac{1}{4}$       & $-\frac{1}{4}$     & 0                  & $\frac{1}{4}$                 & 0            \\\hline
          1     & 0     & 1   & $\frac{1}{4}$     & 0                 & $\frac{1}{4}$   & $-\frac{1}{4}$       & 0                  & $-\frac{1}{4}$     & 0                             & 0            \\\hline
          1     & 1     & 0   & $\frac{1}{4}$     & $\frac{1}{4}$     & 0               & 0                    & $-\frac{1}{4}$     & $-\frac{1}{4}$     & 0                             & 0            \\\hline
          1     & 1     & 1   & $\frac{1}{4}$     & $\frac{1}{4}$     & $\frac{1}{4}$   & 0                    & 0                  & 0                  & $\frac{1}{4}$                 & 1            \\\hline
        \end{tabular}
      }
      \caption{Truth table-like values of the pseudo-Boolean representation of $x_a \cdot x_b \cdot y$}
      \label{table-pseudo-toff}
    \end{center}
  \end{minipage}
\end{table*}
\def\arraystretch{1.1}

Such a phase polynomial can be implemented as a phase oracle using the following
method.

\begin{itemize}
\item Coefficients of the phase polynomial describe the phase that needs to be driven
  as multiples of $i\pi$ (i.e. $3/4$ will be driven as an $S$-gate followed by $T$-gate).
\item The corresponding $\chi_k(\mathbf{x})$ functions can be implemented as a network
  of CNOT and X gates.
\item The phase gates are scheduled according to their corresponding $\chi_k(\mathbf{x})$
  functions.
\item CNOT and X networks are synthesized to drive each phase gate's corresponding
  $\chi_k(\mathbf{x})$ function
\item additional CNOT and X logic is added to return the state to the input state.
\end{itemize}

<<<<<<< Updated upstream
Scheduling these T-gates can be done using the matroid partitioning algorithm described in
~\ref{bib-amy-matroid}. This guarantees a minimal T-depth for CNOT+T circuits.
The terms in Eq.~\ref{eq-toff-bool} are synthesized and scheduled using that algorithm as
demonstrated in Fig.~\ref{fig-toff-mark-matroid}.
=======
Fig.~\ref{fig-toff-mark}
>>>>>>> Stashed changes
\begin{figure}[t]
  \centering
  \scalebox{0.7} {
    %% \Qcircuit @C=0.5em @R=0.2em @!R { \\
%%                              &            &     &          &  & &                           &          &                                &          &                                &          &                             &          & \push{x_a}            &          &     & \\
%%           \lstick{\ket{x_a}} & \ctrl{3}   & \qw &          &  & & \qw                       & \ctrl{4} & \qw                            & \qw      & \qw                            & \ctrl{4} & \qw                         & \qw      & \gate{\mathrm{T}}     & \ctrl{2} & \qw & \\
%%                              &            &     &          &  & &                           &          &                                &          & \push{x_b}                     &          &                             &          & \push{x_a \oplus x_b} &          &     & \\
%%           \lstick{\ket{x_b}} & \ctrl{2}   & \qw & \push{=} &  & & \qw                       & \qw      & \qw                            & \ctrl{2} & \gate{\mathrm{T^{\dagger}}}    & \qw      & \qw                         & \ctrl{2} & \gate{\mathrm{T}}     & \targ    & \qw & \\
%%                              &            &     &          &  & & \push{y}                  &          & \push{x_a \oplus y}            &          & \push{x_a \oplus x_b \oplus y} &          & \push{x_b \oplus y}         &          &                       &          &     & \\
%%           \lstick{\ket{y}}   & \ctrl{-1}  & \qw &          &  & & \gate{\mathrm{T}}         & \targ    & \gate{\mathrm{T^{\dagger}}}    & \targ    & \gate{\mathrm{T}}              & \targ    & \gate{\mathrm{T^{\dagger}}} & \targ    & \qw                   & \qw      & \qw & \\
%% }

\Qcircuit @C=0.5em @R=0.2em @!R {
                       &           & \push{x_a}          &           &          &          & \push{x_a \oplus y}    &           &          &           &                                &           &          &       \\
\lstick{\ket{x_a}}     & \qw       & \gate{T}            & \targ     & \qw      & \ctrl{2} & \gate{T^\dagger}       & \targ     & \ctrl{4} & \qw       & \qw                            & \qw       & \ctrl{2} & \qw \\
                       &           & \push{x_b \oplus y} &           &          &          & \push{x_a \oplus x_b } &           &          &           & \push{x_a \oplus x_b \oplus y} &           &          &       \\
\lstick{\ket{x_b}}     & \targ     & \gate{T^\dagger}    & \qw       & \ctrl{2} & \targ    & \gate{T^\dagger}       & \ctrl{-2} & \qw      & \targ     & \gate{T}                       & \targ     & \targ    & \qw \\
                       &           & \push{y}            &           &          &          & \push{x_b}             &           &          &           &                                &           &          &       \\
\lstick{\ket{y}}       & \ctrl{-2} & \gate{T}            & \ctrl{-4} & \targ    & \qw      & \gate{T}               & \qw       & \targ    & \ctrl{-2} & \qw                            & \ctrl{-2} & \qw      & \qw
}

  }
  \caption{Implementing $x_a \cdot x_b \cdot y$ using CNOT and T gates, sequenced using~\cite{bib-amy-matroid}}
  \label{fig-toff-mark-matroid}
\end{figure}

The next section shows how $\hat{f}(\mathbf{k})$ may be obtained for the three or fewer qubit case using the
Boolean Fourier transform.

\section{Boolean Fourier Transform}

Observe that the $\chi_k(\mathbf{x})$ from Eq.~\ref{eq-boolean-fourier} form an orthonormal basis with respect
to the below inner product.

\begin{equation}
  \label{eq-inner-prod}
  \langle f(\mathbf{x}) , g(\mathbf{x}) \rangle = \frac{1}{2^n} \sum_{\mathbf{x}} (-1)^{f(\mathbf{x})} (-1)^{g(\mathbf{x})}, \quad \mathbf{x} \in \{0,1\}^n
\end{equation}

for $n = |\mathbf{x}| \leq 3$.

That means, we can calculate $\hat{f}(\mathbf{k})$ using the below inner product

\begin{equation}
  \label{eq-fhat-prod}
  \hat{f}(\mathbf{k}) = \langle f(\mathbf{x}), \chi(\mathbf{x})_{\mathbf{k}} \rangle \forall \mathbf{k} 
\end{equation}

$\hat{f}(\mathbf{k})$ is also known as the Boolean Fourier transform of $f(\mathbf{x})$~\cite{bib-odonnell}
\footnote{All equations that are taken from ~\cite{bib-odonnell} are written here
in terms of the $\{0,1\}$ basis and $\chi(\mathbf{x})_{\mathbf{k}}$ functions from ~\cite{bib-amy-matroid}}.

\begin{example}
As an example, let's derive the coefficients for Eq.~\ref{eq-toff-bool} with $\mathbf{k} = (k_{x_a}, k_{x_b}, k_y)$.
For $\mathbf{k} = 100$, $\chi_{100} = x_a$. Table~\ref{table-ex-100} enumerates the values and the term $(-1)^{f(\mathbf{x})} (-1)^{\chi_{100}(\mathbf{x})}$ for the inner product.
\begin{table}[h]
  \begin{center}
    \begin{tabular}{c|c|c|c|c|c}
      \hline
      $x_a$ & $x_b$ & $y$ & $f(\mathbf{x})$ & $\chi_{100}$ & $(-1)^{\chi_{100}(\mathbf{x})} (-1)^{f(\mathbf{x})}$ \\\hline
      0 & 0 & 0 & 0 & 0 & 1\\\hline
      0 & 0 & 1 & 0 & 0 & 1\\\hline
      0 & 1 & 0 & 0 & 0 & 1\\\hline
      0 & 1 & 1 & 0 & 0 & 1\\\hline
      1 & 0 & 0 & 0 & 1 & -1\\\hline
      1 & 0 & 1 & 0 & 1 & -1\\\hline
      1 & 1 & 0 & 0 & 1 & -1\\\hline      
      1 & 1 & 1 & 1 & 1 & 1\\\hline
    \end{tabular}
    \caption{The value of $\chi_{100}$ and the inner product $\langle f(\mathbf{x}), \chi_{100} \rangle$}
    \label{table-ex-100}
  \end{center}
\end{table}

Summing the rightmost column, as in Eq.~\ref{eq-inner-prod}:
\begin{align}
    &\hat{f}(100) = \langle f(\mathbf{x}), \chi_{100} \rangle = \frac{1}{8} ( 1 + 1 + 1 + 1 - 1 - 1 - 1 + 1 ) = \frac{1}{4}
\end{align}
Thus,
\begin{equation}
  f(\mathbf{x}) = \frac{1}{4}x_a + \sum_{\mathbf{k} \neq 000,100} \hat{f}(\mathbf{k}) \cdot \chi_{\mathbf{k}}
\end{equation}
The rest of the calculation is omitted for brevity, but readers can verify that it results in Eq.~\ref{eq-toff-bool}.
\end{example}

Synthesizing the phase polynomial of a Boolean function using the Boolean Fourier
transform for 3 variables or fewer has an interesting consequence. First, the
number of odd multiples of $i\frac{\pi}{4}$ of $\hat{f}(\mathbf{k})$ corresponds to
the T-count needed to implement $f(\mathbf{x})$~\cite{bib-amy-rm}. This is easy to
check, since any even multiple of $i\frac{\pi}{4}$ can be implemented using only
S and Z gates, which do not take any T-cost, and any odd multiple can be realized
using S and Z gates, plus one T-gate. Since each $\hat{f}(\mathbf{k})$
is unique to each $f(\mathbf{x})$ when $|mathbf{x}| \leq 3$,
$\hat{f}(\mathbf{k})$ has the optimal T-count needed to implement
$f(\mathbf{x})$ as a phase oracle.

\subsection{Synthesizing Phase Oracles from ESOP}

A transformation $\ket{x}\ket{y} \rightarrow (-1)^{x \oplus y}\ket{x}\ket{y}$ can be realized using
the below circuit
\begin{figure}[h]
  \scalebox{1.0} {
    \Qcircuit @C=0.5em @R=0.2em @!R {
      \lstick{\ket{x}} & \qw & \gate{Z} & \qw  \\
      \lstick{\ket{y}} & \qw & \gate{Z} & \qw
    }
  }
\end{figure}

Thus, given an Exclusive Sum of Products expression of a Boolean function
$f(\mathbf{x}) = x_a x_b \cdots x_g \oplus x_i x_j \cdots x_m \oplus \cdots$, we
can implement its phase oracle as controlled phase gates of the following form
[DCTODO diagram]

This method is used by Qiskit's Phase Oracle [DCTODO ref]. Our proposal revolves around
the effective use of such an Exclusive Sum of Products to implement such a phase oracle.




\section{Proposal}
\label{Pro}
\subsection{Overview}

In this section, we go over the overall flow of our proposal, depicted in Fig.~\ref{fig-flow}. The blue boxes
represent precomputation, while tan boxes represent runtime.

\begin{itemize}
\item \textbf{(Generate 3bit phase polynomials)} : Precompute library of phase polynomials of Boolean functions of
  3 bits or fewer using the Boolean Fourier transform.
\item \textbf{(Synthesize 3bit phase subcircuits)} : The phase polynomials are then sequenced using matroid
  partitioning from~\cite{bib-amy-matroid} and synthesized as subcircuits. The results are stored in hard disk,
  to be loaded into memory at runtime.
\item \textbf{(Decompose $f$ into ESOP expression)} : At runtime, a Boolean function is read in and converted to
  an ESOP expression.
\item \textbf{(Minimize the number of $n \geq 4$ cubes in ESOP)} : The ESOP is optimized by adapting
  EXORCISM-4~\cite{bib-exorcism} to reduce the number of cubes of more than 4 literals. This method is explained
  in Sec.~\ref{Pro:Minimize}.
\item \textbf{(Map the $n \leq 3$ cubes to the 3 bit library)} : Takes all cubes of $n \leq 3$ and maps them to 
  appropriate functions from the 3 bit library, inserting the corresponding subcircuit into the phase oracle.
\item \textbf{(Synthesize $n \geq 4$ cubes)} : Synthesizes all cubes of $n \geq 4$ using the method labeled in
  Sec.~\ref{Pro:n4}.
\item \textbf{(Combine $n \geq 4$ and $n \leq 3$)} : The results of the above two steps are combined into one
  phase oracle.
\item \textbf{(Optimize for T-count and T-depth)} : The combined circuit is optimized for T-count and T-depth using
  the T-par algorithm~\cite{bib-amy-matroid}. Further detail of this step is in Sec.~\ref{Pro:Tpar}.
\end{itemize}
\begin{figure}[t]
  \centering
  \scalebox{1.0} {
    %% \begin{tikzpicture}[
%%     box/.style={draw, rounded corners, fill=orange!20, align=center, font=\small, minimum width=4.2cm, minimum height=1.1cm, rotate=90, text width=3.6cm, inner sep=6pt},
%%     bigbox/.style={draw, rounded corners, fill=orange!20, align=center, font=\small, minimum width=5.3cm, minimum height=1.3cm, inner sep=8pt, rotate=90, text width=4.6cm},
%%     bluebox/.style={draw, rounded corners, fill=blue!20, align=center, font=\small, minimum width=2.9cm, minimum height=1.6cm, rotate=90, text width=2.5cm},
%%     arrow/.style={-Stealth, thick, color=teal!80!black},
%%     bendarrow/.style={-Stealth, thick, color=teal!80!black, line width=5pt, rounded corners=15pt}
%% ]

%% % Place nodes in a straight horizontal (east-west) line with rotated text
%% \node[box] (decompose) at (0,0) {Decompose $f$ into\\ESOP expression};
%% \node[box] (minimize) at (3,0) {Minimize the \\ number of $n \geq 4$ cubes in ESOP};
%% \node[box] (map3qubit) at (6.2,0) {Map the $n \leq 3$ cubes to the \textcolor{blue}{3 qubit} library};
%% \node[box] (optimize) at (9.7,0) {Optimize for T-count and T-depth};
%% \node[bigbox] (synthesize) at (13.4,0) {Synthesize $n \geq 4$};

%% % Blue box above "map3qubit"
%% \node[bluebox] (oracle) at (6.2,2.6) {Generate 3bit\\Boolean Oracle\\Library};

%% % Arrows (horizontal)
%% \draw[arrow] (decompose) -- (minimize);
%% \draw[arrow] (minimize) -- (map3qubit);
%% \draw[arrow] (map3qubit) -- (optimize);
%% \draw[arrow] (optimize) -- (synthesize);

%% % Bent arrow from blue box to map3qubit
%% \draw[bendarrow] (oracle.south) -- ++(0,-0.7) -- (map3qubit.north);

%% \end{tikzpicture}
\begin{tikzpicture}[
    box/.style={draw, rounded corners, fill=orange!20, align=center, font=\scriptsize, minimum width=3cm, minimum height=0.8cm, text width=2.8cm, inner sep=2pt},
    bigbox/.style={draw, rounded corners, fill=orange!20, align=center, font=\scriptsize, minimum width=3cm, minimum height=1.0cm, inner sep=2.8pt, text width=5.4cm},
    bluebox/.style={draw, rounded corners, fill=blue!20, align=center, font=\scriptsize, minimum width=3cm, minimum height=1.0cm, text width=2.8cm, inner sep=2pt},
    arrow/.style={-Stealth, thick, color=teal!80!black},
    bendarrow/.style={-Stealth, thick, color=teal!80!black}
]

% Place main vertical nodes
\node[box] (decompose) at (0,0) {Decompose $f$ into ESOP expression};
\node[box, below=0.8cm of decompose] (minimize) {Minimize the number of $n \geq 4$ cubes in ESOP};
\node[box, below=0.8cm of minimize] (synthesize) {Synthesize $n \geq 4$ cubes};

% Synthesize to the left of map3qubit, aligned with minimize
\node[box, left=1cm of synthesize] (map3qubit) at ($(synthesize)+( -1,0 )$) {Map the $n \leq 3$ cubes to the 3 qubit library};

% 'Combine' below the midpoint of synthesize and map3qubit
\path let \p1 = (map3qubit), \p2 = (synthesize) in
    coordinate (combinecoord) at ({0.5*(\x1+\x2)}, {\y2-1.2cm});
\node[box] (combine) at (combinecoord) {Combine $n \geq 4$ and $n \leq 3$};

% Optimize below combine
\node[box, below=0.8cm of combine] (optimize) {Optimize for T-count and T-depth};

% Blue box left of decompose
\node[bluebox, left=0.5cm of decompose] (oracle) {Generate 3bit phase polynomials};
\node[bluebox, below=0.8cm of oracle] (oraclecirc) {Synthesize 3bit phase subcircuits};

% Arrows
\draw[arrow] (decompose) -- (minimize);
\draw[arrow] (minimize) -- (map3qubit);
\draw[arrow] (minimize) -- (synthesize);
\draw[arrow] (synthesize) -- (combine);
\draw[arrow] (map3qubit) -- (combine);
\draw[arrow] (combine) -- (optimize);
\draw[arrow] (oracle) -- (oraclecirc);
\draw[arrow] (oraclecirc) -- (map3qubit);
% Bent arrow from blue box to map3qubit
% \draw[arrow] (oracle.south) |- (map3qubit.north);

\end{tikzpicture}
%% \begin{tikzpicture}[
%%     box/.style={draw, rounded corners, fill=orange!20, align=center, font=\scriptsize, minimum width=3cm, minimum height=0.8cm, text width=2.8cm, inner sep=2pt},
%%     bigbox/.style={draw, rounded corners, fill=orange!20, align=center, font=\scriptsize, minimum width=3cm, minimum height=1.0cm, inner sep=2.8pt, text width=5.4cm},
%%     bluebox/.style={draw, rounded corners, fill=blue!20, align=center, font=\scriptsize, minimum width=3cm, minimum height=1.0cm, text width=2.8cm, inner sep=2pt},
%%     arrow/.style={-Stealth, thick, color=teal!80!black},
%%     bendarrow/.style={-Stealth, thick, color=teal!80!black}
%% ]

%% % Place nodes in a straight vertical line
%% \node[box] (decompose) at (0,0) {Decompose $f$ into ESOP expression};
%% \node[box, below=0.8cm of decompose] (minimize) {Minimize the number of $n \geq 4$ cubes in ESOP};
%% \node[box, below=0.8cm of minimize] (map3qubit) {Map the $n \leq 3$ cubes to the \textcolor{blue}{3 qubit} library};
%% \node[box, below=0.8cm of minimize, left=1cm of map3qubit] (synthesize) {Synthesize $n \geq 4$};
%% \node[box, below=0.8cm of synthesize, left=0.5cm of synthesize] (combine) {Combine $n \geq 4$ and $n \leq 3$};
%% \node[box, below=0.8cm of combine] (optimize) {Optimize for T-count and T-depth};


%% % Blue box left of "minimize"
%% \node[bluebox, left=0.5cm of decompose] (oracle) {Generate 3bit\\Boolean Oracle\\Library};

%% % Arrows (vertical)
%% \draw[arrow] (decompose) -- (minimize);
%% \draw[arrow] (minimize) -- (map3qubit);
%% \draw[arrow] (minimize) -- (synthesize);
%% \draw[arrow] (synthesize) -- (combine);
%% \draw[arrow] (map3qubit) -- (combine);
%% \draw[arrow] (combine) -- (synthesize);

%% % Bent arrow from blue box to map3qubit
%% \draw[bendarrow] (oracle.south) |- (map3qubit.west);

%% \end{tikzpicture}

  }
  \caption{The overall flow of the proposed method}
  \label{fig-flow}
\end{figure}

\subsection{Minimize the number of $n \geq 4$ cubes in ESOP}
\label{Pro:Minimize}

To minimize the number of $n \geq 4$ cubes in the ESOP, the proposal adapts EXORCISM-4~\cite{bib-exorcism}.
EXORCISM-4 is a heuristic method of minimizing the number of cubes in a given ESOP expression. Extended
discussion of its details is defered to the cited paper. 

We modify it for our purposes by changing
the cost function \texttt{CountCubesInExactPseudoKro} that EXORCISM-4 uses for its heuristic, which uses a plain sum
of cubes to assess the cost of the arrangement that its heuristic encounters. We introduce a cost function
that instead uses a weighted sum of the cubes, where each cube is weighted by its T-count (i.e. a 2 literal cube
has weight 0) for $n \leq 3$ and $2^n$ for $n \geq 4$ (its T-count when synthesized according to the method in
Sec.~\ref{Pro:n4}). This penalizes terms of $n \geq 4$, while terms of $n < 3$ are free.


\subsection{Synthesize $n \geq 4$ cubes}
\label{Pro:n4}

To synthesize cubes of $n \geq 4$ variables, we utilize the construction in Fig.~\ref{fig-phase-n4}. Here,
$U_F$ and $U_F^{\dagger}$ are $F$-controlled-NOT gates, meaning that they apply X to the target bit when
the Boolean function $F$ is true. We add an ancilla bit and use the relative-phase Boolean decomposition
method from~\cite{bib-clarino-lut} to enable the implementation of such $F$-controlled-NOT gates with a
reduced number of T-gates. We defer detailed discussion of that method to the cited paper. This resulting
phase oracle has T-count $2^{n}$, and T-depth $2^{n+1}$. Note that this means that larger cubes can
dominate the T-depth of a given circuit, so it is imperative that the minimum of them be found for
any given Boolean function.

\begin{figure}[t]
  \centering
  \scalebox{1.0} {
    \Qcircuit @C=1em @R=1em {
  \lstick{\ket{x_1}} & \qw               & \multigate{3}{U_F} & \qw           & \multigate{3}{U_F^{\dagger}} & \qw \\
  \lstick{\ket{x_2}} & \qw               & \ghost{U_F}        & \qw           & \ghost{U_F^{\dagger}}        & \qw \\
                     & \push{\vdots}     &                    & \push{\vdots} &                              & \push{\vdots} \\
  \lstick{\ket{x_n}} & \qw               & \ghost{U_F}       & \qw            & \ghost{U_F^{\dagger}}        & \qw \\
  \lstick{\ket{y}}   & \qw               & \targ\qwx[-1]     & \gate{Z}       & \targ\qwx[-1]                & \qw
}

  }
  \caption{Synthesize a phase oracle for a cube of $n \geq 4$}
  \label{fig-phase-n4}
  \vspace{-0.5cm}
\end{figure}

\subsection{Optimize for T-count and T-depth}
\label{Pro:Tpar}

To optimize for T-count and T-depth, we utilize the T-par algorithm from~\cite{bib-amy-matroid}. This is a
variant of the matroid partitioning algorithm that deals with optimization of Clifford+T circuits by heuristically
partitioning along the H-gates present in it. This deals with the circuits synthesized in Sec.~\ref{Pro:n4} as well
as the CNOT+T circuits the $n \leq 3$ processing step produces. We defer more detailed discussion to the cited
paper.





% DCTODO: Add Evaluation, Results, and Conclusion Sections

\bibliographystyle{IEEEtran}
\bibliography{ref}

\end{document}
