\documentclass[conference]{IEEEtran}

\usepackage{float}
\usepackage{cite}
\usepackage{caption}
\usepackage{subcaption}
\usepackage{amsmath,amssymb,amsfonts,amstext}
% \usepackage{algorithmic}
\usepackage[dvipdfmx]{graphicx}
\usepackage{textcomp}
\usepackage{xcolor}
\usepackage{colortbl}
\usepackage[T1]{fontenc}
\usepackage{mdframed}
\usepackage{algorithm, algpseudocode} % texlive-science
\usepackage[braket, qm]{qcircuit}
\usepackage{newtxtext}
\usepackage[normalem]{ulem}

\usepackage{nameref}
\usepackage[braket, qm]{qcircuit}
\usepackage{tikz}
%
\newcommand*\rtofstarg{\tikz[baseline=(char.base)]{
            \node[shape=circle,double,draw,inner sep=1pt,minimum width=16pt] (char) {S};}}
\newcommand*\rtofsdgtarg{\tikz[baseline=(char.base)]{
            \node[shape=circle,double,draw,inner sep=1pt] (char) {S$^{\dagger}$};}}
\newcommand*\rtoftarg{\tikz[baseline=(char.base)]{
            \node[shape=circle,draw,inner sep=-0.5pt,fill=none] (char) {$\oplus$};}}
\newcommand*\rtofsctrl{\tikz[baseline=(char.base)]{
            \node[shape=circle,double,draw,inner sep=1pt,fill=black, minimum width=3pt] (char) {};}}
\newcommand*\rtofsctrlo{\tikz[baseline=(char.base)]{
            \node[shape=circle,double,draw,inner sep=1pt,fill=white, minimum width=3pt] (char) {};}}
%\newcommand*\rtofsctrlo{\tikz[baseline=(char.base)]{
%            \node[shape=circle,double,draw,inner sep=1pt,fill=white, minimum width=3pt] (char) {};}}
\newcommand*\dashedwire{\tikz{
            \draw[dashed] (0,0) -- (0.5,0); }}

\newcommand*\ctrlrtofs[1]{\push{\rtofsctrl}\qwx[#1]\qw}
\newcommand*\ctrlortofs[1]{\push{\rtofsctrlo}\qwx[#1]\qw}
\newcommand*\targrtofs{\push{\rtofstarg}\qw}
\newcommand*\targrtof{\push{\rtoftarg}\qw}


\begin{document}

\title{Synthesis of Phase Oracles Using CNOT+T Circuits and ESOP Minimization}

\author{
    \IEEEauthorblockN{Your Name}
    \IEEEauthorblockA{Your Institution \\
    Email: your@email.com}
}

\maketitle

\begin{abstract}
Phase oracles are essential components of quantum algorithms such as Grover's algorithm. These circuits apply a phase of $-1$ to an input state when a particular Boolean function $f$ is true. For functions of fewer than 3 variables, these are well known to be exactly implementable using only the CNOT, T, and T$^\dagger$ gates. Amy~\cite{amy2018} established that this class of circuits can be provably optimized for T-count and T-depth. We identify a class of Boolean functions of 4 or more variables which can be synthesized using such a circuit, by way of a library of circuits that implement phase oracles of 3 or fewer variables. We then use this to synthesize Boolean functions of 4 or more variables, by minimizing the number of ESOP terms of more than 3 variables. We then synthesize the rest using a modification of a previously proposed method to synthesize arbitrary Boolean functions as single-target Boolean gates. We compare the results against Qiskit's \texttt{PhaseOracle} method.
\end{abstract}

\section{Introduction}
One of quantum computing's most significant results is the development of Grover's
algorithm~\cite{bib-grover1996fast}. This algorithm performs an unstructured search
optimally~\cite{bib-zaika-grov-opt}. Various applications have
been proposed for such an algorithm, including the acceleration of solving NP-hard
problems~\cite{bib-williams-grover-np}.

Grover's algorithm uses a phase oracle, a quantum circuit that applies a phase of
$-1$ when the Boolean function it implements is true, to increase the probability
of returning a fulfilling assignment.

Phase oracles are interesting because a restricted class of them can be implemented
using only phase gates such as the T, T$^\dagger$, and CNOT gates (hereafter
referred to as CNOT+T). In fact, phase oracles of 3 or fewer variables can all be
implemented exactly using this set of gates, and these circuits can be calculated
exactly using the Boolean Fourier transform. Amy et al.~\cite{amy-meet-in-middle}
demonstrates that CNOT+T circuits can be optimized for both T-count and
T-depth~\cite{bib-amy-matroid}.

However, in general, it is impossible to implement Boolean functions of 4 or more
variables using only this gate set~\cite{bib-amy-rm}. There will inevitably be a need to use
a Hadamard (H) gate. The addition of this gate makes optimal solutions non-unique,
and several methods to optimize for T-count and T-depth when including the H gate
are only heuristic ~\cite{amy-meet-in-middle,bib-amy-matroid,bib-amy-rm}.

Phase oracles are straightforward to synthesize from an Exclusive Sum
of Products (ESOP) expression~\cite{bib-phaseoracle}. Since each cube maps to
a Boolean function, it is easy to see that any such function expressible with
cubes of at most 3 variables can be implemented as a CNOT+T circuit. We utilize
this observation in our proposal.

In this work, we propose a solution that utilizes CNOT+T circuits to reduce the
T-count and T-depth for phase oracles of a given Boolean function of arbitrary
literal count.

Our contributions include:

\begin{itemize}
\item A precomputed library of circuits that implement all Boolean functions of
  3 variables or fewer.
\item A synthesis method that uses ESOP minimization to limit the number of
  product terms involving 4 or more variables.
\end{itemize}

We find that our solution... [DCTODO: complete with results]

The rest of this paper is structured as follows: First Sec~\ref{Pre}
introducs some preliminary knowledge. In addition to the basics of
qubits and quantum circuits, this includes the basics of the
mathematics of phase polynomials and their mapping to phase oracles
and CNOT+T circuits. Then Sec.~\ref{Mot} goes through some
motivational examples that inform the general idea of our proposal.
Sec.~\ref{Pro} details the proposal itself, including how to
generate a library of up to 3-qubit phase oracles, and how to use
it to generate phase oracles of 4 or more qubits. In Sec.~\ref{Exp},
we test our results against Qiskit PhaseOracle and analyze our
findings. Finally, Sec.~\ref{Conc} concludes with our findings and
proposes some future research.



\section{Preliminaries}

\subsection{Quantum Gates and Circuits}
CNOT+T circuits consist only of the CNOT, T, and T$^\dagger$ gates. Such circuits can be represented by a path-sum, which captures the algebraic structure of the circuit. Two circuits sharing the same path-sum are functionally equivalent~\cite{pathsum2018}. This property enables a resynthesis process where T gates can be merged.

One complication in this process is the treatment of negated variables. As demonstrated in Fig.~[DCTODO], certain T-gates can be merged more easily using the Boolean Fourier transform of a function rather than by explicitly negating its inputs. A library of such decompositions aids this combination.

\subsection{Phase Oracles}
A phase oracle applies a phase $(-1)$ to any input that satisfies a Boolean function $f$. A common method of synthesizing phase oracles is via an ESOP representation, expressing the function as an XOR of product terms (ANDs) of variables. Given the mapping $|x\rangle|y\rangle \mapsto (-1)^{x \oplus y}|x\rangle|y\rangle$, such circuits can be synthesized using known quantum subroutines (see Fig.~[DCTODO]).


\section{Proposal}

A sum-of-paths representation of a Boolean function can be derived from the Boolean Fourier decomposition~\cite{odonnell}. However, this decomposition forms a complete basis for functions with 3 variables or fewer when using the CNOT+T gate set. We demonstrate that composing such decompositions enables the exact implementation of a subset of $k$-variable Boolean functions.

We use a modified Boolean Fourier decomposition approach, based on~\cite{amy_duality}, to construct our 3-qubit-or-less library. Our synthesis is ancilla-free and schedules T gates following~\cite{amy2018} to minimize T-depth.

We then apply a modified form of the relative phase function synthesis described in~\cite{amy_relphase} to complete our circuit synthesis.

\section{Pseudo-Boolean Functions and the Boolean Fourier Transform}

One way to realize a Boolean AND is to implement what's called a {\it pseudo-Boolean}
representation~\cite{bib-barenco-elementary,bib-amy-cnot}, which uses a mix of
Boolean values, integer coefficients, and arithmetic to create a mapping $\{0,1\}^n \mapsto \{0,1\}$~\cite{bib-amy-rm}.

Before defining the concept mathematically, first observe Equation~\ref{eq-toff-bool} to see an example of a
pseudo-Boolean function. Here, real-valued coefficients are multiplied with Boolean functions of $x_a$, $x_b$,
and $y$, creating a real number, and those real numbers are then summed together using arithmetic operations
$+$ and $-$. Table~\ref{table-pseudo-toff} shows the values of $x_a \cdot x_b \cdot y$ with respect to
its components, where $T(f)$ is as defined in Equation~\ref{eq-toff-bool}. It can be proven by inspection of
Table~\ref{table-pseudo-toff} that the pseudo-Boolean expression in~\ref{eq-toff-bool} indeed implements
$x_a \cdot x_b \cdot y$. 
\begin{equation}
  \label{eq-toff-bool}
  \begin{aligned}
    &x_a \cdot x_b \cdot y &= &x_a + x_b + y \\\nonumber
    &&&+ 7(x_a \oplus x_b) + 7(x_a \oplus y) + 7(x_b \oplus y) \\\nonumber
    &&&+ (x_a \oplus x_b \oplus y)\\\nonumber
    &&= & (T_{x_a} + T_{x_b} + T_{y} + T_{x_a \oplus x_b} \\\nonumber
    &&&T_{x_a \oplus y} T_{x_b \oplus y} + T_{x_a \oplus x_b \oplus y})
  \end{aligned}
\end{equation}
\begin{table*}[t]
  \begin{minipage}{\textwidth}
    \begin{center}
      \scalebox{1.0} {
        \begin{tabular}{c|c|c|c|c|c|c|c|c|c|c}
          $x_a$ & $x_b$ & $y$ & $T_{x_a}$ & $T_{x_b}$ & $T_{y}$ & $T_{x_a \oplus x_b}$ & $T_{x_a \oplus y}$ & $T_{x_b \oplus y}$ & $T_{x_a \oplus x_b \oplus y}$ & $4x_a \cdot x_b \cdot y$\\\hline
          0     & 0     & 0   & 0         & 0         & 0       & 0                    & 0                  & 0                  & 0                             & 0            \\\hline
          0     & 0     & 1   & 0         & 0         & $1$   & 0                    & $7$             & $7$
                       & $1$                         & 0            \\\hline
          0     & 1     & 0   & 0         & $1$     & 0       & $7$               & $7$             & 0                  & $1$                         & 0            \\\hline
          0     & 1     & 1   & 0         & $1$     & $1$   & $7$               & 0                  & $7$             & 0                             & 0            \\\hline
          1     & 0     & 0   & $1$     & 0         & 0       & $7$               & 0                  & $7$             & $1$                         & 0            \\\hline
          1     & 0     & 1   & $1$     & 0         & $1$   & $7$               & $7$             & 0                  & 0                             & 0            \\\hline
          1     & 1     & 0   & $1$     & $1$     & 0       & 0                    & $7$             & $7$             & 0                             & 0            \\\hline
          1     & 1     & 1   & $1$     & $1$     & $1$   & 0                    & 0                  & 0                  & $1$                         & 4            \\\hline
        \end{tabular}
      }
      \caption{Truth table-like values of the pseudo-Boolean representation of $x_a \cdot x_b \cdot y$}
      \label{table-pseudo-toff}
    \end{center}
  \end{minipage}
  
\end{table*}

\begin{figure*}[t]
  \centering
  \scalebox{0.8} {
    \Qcircuit @C=0.5em @R=0.2em @!R { \\
                             &            &     &          &  & &                           &          &                                &          &                                &          &                             &          & \push{x_a}            &          &     & \\
          \lstick{\ket{x_a}} & \ctrl{3}   & \qw &          &  & & \qw                       & \ctrl{4} & \qw                            & \qw      & \qw                            & \ctrl{4} & \qw                         & \qw      & \gate{\mathrm{T}}     & \ctrl{2} & \qw & \\
                             &            &     &          &  & &                           &          &                                &          & \push{x_b}                     &          &                             &          & \push{x_a \oplus x_b} &          &     & \\
          \lstick{\ket{x_b}} & \ctrl{2}   & \qw & \push{=} &  & & \qw                       & \qw      & \qw                            & \ctrl{2} & \gate{\mathrm{T^{\dagger}}}    & \qw      & \qw                         & \ctrl{2} & \gate{\mathrm{T}}     & \targ    & \qw & \\
                             &            &     &          &  & & \push{y}                  &          & \push{x_a \oplus y}            &          & \push{x_a \oplus x_b \oplus y} &          & \push{x_b \oplus y}         &          &                       &          &     & \\
          \lstick{\ket{y}}   & \ctrl{-1}  & \qw &          &  & & \gate{\mathrm{T}}         & \targ    & \gate{\mathrm{T^{\dagger}}}    & \targ    & \gate{\mathrm{T}}              & \targ    & \gate{\mathrm{T^{\dagger}}} & \targ    & \qw                   & \qw      & \qw & \\
}

  }
  \caption{Implementing $x_a \cdot x_b \cdot y$ using CNOTs and T gates}
  \label{fig-toff-mark}
\end{figure*}


In general, a pseudo-Boolean function is a function such that
\begin{equation}
  \label{eq-pseudo-boolean}
  F(\mathbf{x}) = \sum_{\mathbf{k}} c_a \cdot f_k(\mathbf{x}), c_a \in \mathbb{R}, f_k : \{0,1\}^n \mapsto \{0,1\}
  F(\mathbf{x}) \in \{0,1\}
\end{equation}

In plain language, it is the arithmetic sum of the products of Boolean functions with real valued coefficients, such that
the values of the sum are in the set of binary values $\{0,1\}$~\cite{bib-barenco-elementary}. This
representation is also often referred to in the context of the phases of quantum circuits as the
{\it phase polynomial}~\cite{bib-amy-cnot}.

Phase polynomials and pseudo-Boolean functions important because phase polynomials can be calculated
using the {\it Boolean Fourier transform}, which is shown in Equation~\ref{eq-boolean-fourier}.
\begin{equation}
  \label{eq-boolean-fourier}
  \begin{aligned}
    f(\mathbf{x}) = \sum_{\mathbf{a} \neq 0} \hat{f}(\mathbf{a}) \cdot ( a_0 \cdot x_0 \oplus a_1 \cdot x_1 \cdots \oplus a_{n-1} \cdot x_{n-1}),\\\nonumber
    \hat{f}(\mathbf{a}) \in \mathbb{R} , a_0,a_1,\cdots a_{n-1} \in [0,1]
  \end{aligned}
\end{equation}

$\hat{f}(\mathbf{a})$ is a set of $2^{n}-1$ coefficients (one for each $\mathbf{a}$, except for the trivial
case $\mathbf{a} = 0$) associated with the pseudo-Boolean function, and
$( a_0 \cdot x_0 \oplus a_1 \cdot x_1 \cdots \oplus a_{n-1} \cdot x_{n-1}) $ are linear functions
that form a basis for the pseudo-Boolean function, and are maps $[0,1]^n \mapsto [0,1]$. $\hat{f}$ is referred to
as the {\it phase polynomial} of the quantum Boolean circuit.

Observe that a T-gate adds to the phase no matter which qubit it acts on.
For example, a T-gate acting on bit 0 creates the mapping $\ket{1}_0\ket{1}_1 \mapsto \ket{1}_0\ket{1}_1$ while
a T-gate acting on bit 1 also creates the same mapping. Therefore, $\hat{f}(\mathbf{a})$ sets the number of
phase gates, such as T-gates and T$^{\dagger}$ gates, that are used in the circuit. A Boolean function thus can
have a T-count up to $2^{n}-1$.

It is easy to see then that the $\hat{f}(\mathbf{a})$ that has the minimal number of odd entries for each
$\mathbf{a}$ also minimizes the T-count~\cite{bib-amy-rm}. Observe also, that the linear functions
$( a_0 \cdot x_0 \oplus a_1 \cdot x_1 \cdots \oplus a_{n-1} \cdot x_{n-1}) $ can be implemented
entirely by CNOT gates. This means that Equation~\ref{eq-boolean-fourier} can be realized entirely by
CNOT gates and phase gates, known as a {\it CNOT+T network}~\cite{bib-amy-cnot}. 

Optimizing the T-depth of such a CNOT+T network is done using the method in~\cite{bib-amy-matroid}.

\section{Phase Polynomial Calculation for the Three Qubit Case}
\label{Chap:Bool-pbool3q}

This section demonstrates how to calculate the phase polynomial for a three input Boolean function, using the
Boolean Fourier transform~\cite{bib-odonnell}. This section uses only a simplified version
of the Boolean Fourier transform introduced in~\cite{bib-odonnell}, but with notation taken from
\cite{bib-barenco-elementary} and~\cite{bib-amy-cnot}. The notation is chosen to more clearly demonstrate how
the equation may be implemented using quantum gates defined in the CNOT+T basis. 

First, the inner product to be used for the Boolean Fourier transform is defined below

\begin{equation}
  \label{eq-inner-prod}
  \langle f(\mathbf{x}) , g(\mathbf{x}) \rangle = \sum_{\mathbf{x}} \text{INNPROD}(f(\mathbf{x}),g(\mathbf{x})), \mathbf{x} \in \{0,1\}^n
\end{equation}
\begin{equation}  
  \text{INNPROD}(f(\mathbf{x}),g(\mathbf{x})) =
  \begin{cases}
    7 & \text{if $f(\mathbf{x})\neq g(\mathbf{x})$} \\
    1 & \text{if  $f(\mathbf{x})= g(\mathbf{x})$} 
  \end{cases}
\end{equation}

\cite{bib-odonnell} states that the {\it parity functions}, the set of exclusive sums of the input variables
$\chi(x_0,x_1,\cdots,x_n)$ shown in Equation~\ref{eq-bool-basis}, form an orthogonal basis with respect to the
inner product defined in Equation~\ref{eq-inner-prod}. In this equation, there is a factor of $\frac{1}{2^n}$ so
that $\langle f(\mathbf{x}), f(\mathbf{x}) \rangle = 1$, making the basis orthonormal.
\begin{equation}
  \label{eq-bool-basis}
  \begin{split}
  \chi(x_0,x_1,\cdots,x_n)_{\mathbf{a}} = a_0 \cdot x_0 \oplus a_1 \cdot x_1 \cdots \oplus a_{n-1} \cdot x_{n-1},\\
  a_0,a_1,\cdots a_{n-1} \in [0,1]
  \end{split}
\end{equation}

Because for each $\chi_{\mathbf{a}}$ and $\chi_{\mathbf{b}}$, there is
$\langle \chi_{\mathbf{a}}, \chi_{\mathbf{b}} \rangle = 0$, calculating $\hat{f}(\mathbf{a})$ for a specific
value of $\mathbf{a}$ simply involves taking the inner product of the full function $f(\mathbf{x})$ and the
basis state $\chi(\mathbf{x})_{\mathbf{a}}$ as shown in Equation~\ref{eq-fhat-prod}. This multiplication is
repeated for every value of $\mathbf{a}$ until all of $\hat{f}(\mathbf{a})$ is calculated.
\begin{equation}
  \label{eq-fhat-prod}
  \hat{f}(\mathbf{a}) = \langle f(\mathbf{x}), \chi(\mathbf{x})_{\mathbf{a}} \rangle
\end{equation}

\begin{example}
  As an example, let's try to derive the coefficients from Equation~\ref{eq-toff-bool}. Here
  $\mathbf{a} = (a_{x_a}, a_{x_b}, a_y)$.

  Because the trivial case ($\mathbf{a}= 000$) is not included, let's start with $\mathbf{a} = 100$ with
  $\chi_{100} = x_a$.
  Table~\ref{table-ex-100} enumerates its values in truth table form for convenience, as well as the
  value of each of the $(-1)^{f(\mathbf{x})} (-1)^{\chi_{001}(\mathbf{x})}$ terms in the inner product
  $\langle f(\mathbf{x}), \chi_{001} \rangle$

  \begin{table}[h]
    \begin{center}
      \begin{tabular}{c|c|c|c|c|c}
        \hline
        $x_a$ & $x_b$ & $y$ & $f(\mathbf{x})$ & $\chi_{100}$ & INNPROD($\chi_{100},f(\mathbf{x}$) \\\hline
        0 & 0 & 0 & 0 & 0 & 1\\\hline
        0 & 0 & 1 & 0 & 0 & 1\\\hline
        0 & 1 & 0 & 0 & 0 & 1\\\hline
        0 & 1 & 1 & 0 & 0 & 1\\\hline
        1 & 0 & 0 & 0 & 1 & 7\\\hline
        1 & 0 & 1 & 0 & 1 & 7\\\hline
        1 & 1 & 0 & 0 & 1 & 7\\\hline      
        1 & 1 & 1 & 1 & 1 & 1\\\hline
      \end{tabular}
      \caption{The value of $\chi_{100}$ and the inner product $\langle f(\mathbf{x}), \chi_{100} \rangle$}
      \label{table-ex-100}
    \end{center}
  \end{table}

  
  The terms in the right most column can then be summed just like in Equation~\ref{eq-inner-prod}:
  \begin{align}
      &\hat{f}(100) = &\langle f(\mathbf{x}), \chi_{100} \rangle = &( 1 + 1 + 1 + 1 - 1 - 1 - 1 + 1) \\\nonumber
      &&& = ( 2 ) = 1\nonumber\\\nonumber
  \end{align}
  This then gives:
  \begin{equation}
    f(\mathbf{x}) = x_a + \sum_{\mathbf{a} \neq 000,100} \hat{f}(\mathbf{a}) \cdot \chi_{\mathbf{a}}
  \end{equation}

  In the interest of conciseness, the rest of the calculation is omitted. But the reader can easily perform the
  rest to verify its equivalence with Equation~\ref{eq-toff-bool}.
\end{example}

The pseudo-Boolean function from Eq.~\ref{eq-toff-bool} is implemented as a circuit in Fig~\ref{fig-toff-mark}.


% DCTODO: Add Evaluation, Results, and Conclusion Sections

\bibliographystyle{IEEEtran}
\bibliography{ref}

\end{document}
