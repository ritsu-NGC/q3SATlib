\section{Experimental Results}
\label{Exp}
\subsection{Experiment Setup}
\label{Exp:Set}
We compare the ability of T-par to optimize circuits produced using our methodology against those of
Qiskit's \texttt{PhaseOracle} method. This method takes in an ESOP expression and applies the synthesis in
Sec.~\ref{Mot:Lib}. Since \texttt{PhaseOracle} only generates Control-Z gates,
we further decompose these using the methods described in Sec.~\ref{Pro}, and for cubes of $n \geq 4$,
the decomposition from Sec.~\ref{Pro:n4} is used. \texttt{PhaseOracle} synthesizes expressions in the same
manner as Ex.~\ref{ex-series} and so cannot take advantage of the sum of paths simplification that
the proposed method uses.

We randomly generate ESOP expressions to feed into both of our methods, and average all
of the differences in T-count and T-depth between the two. The main code is run in python, while running
T-par as a system call to a C++ binary and EXORCISM-4 through a Python-C++ API.

Here we propose two sets of measurements:

\begin{itemize}
\item (N3) : ESOP expressions are constrained to those of $n \leq 3$. This tests the ability of the
  precomputed 3bit library to enable simplification of pure CNOT+T circuits
\item (Scramble) : ESOP expressions are again constrained to $n \leq 3$, but these are then ``scrambled''
  by using a Davio expansion of the ESOP expression according to different variable orders until an ESOP expression
  that contains cubes of $n \geq 4$ is found. This tests the ability of the proposed method to find such a
  solution in its EXORCISM-4 step, by guaranteeing that there is such an optimal solution.
\end{itemize}

We test these for $n=4-12$ literals. The ESOP expressions are randomly generated according to two groups
based on the number of cubes $k$ in them.

\begin{itemize}
\item ($k=2^{n-1}$) This increases the chances that inserted subcircuits will have common literals that allow them
  to be simplified and have reduced T-count, while making it harder for T-par to partition any cubes $n \geq 4$
  for T-depth scheduling.
\item ($k=n$) This makes the arrangements sparser, allowing for better parallelization by T-par, but making
  it harder to have common literals that can combine to reduce T-count.
\end{itemize}

\subsection{Results}
\label{Exp:Res}

\begin{table}[t]
  \begin{center}
    \scalebox{1.0} {
      \begin{tabular}{c|c|c|c|c}\hline
                             & \multicolumn{2}{c|}{N3} & \multicolumn{2}{c}{Scramble}\\\hline
                             & $k=2^{n-1}$ & $k=n$    & $k=2^{n-1}$  & $k=n$        \\\hline
        Proposed             &             &          &             &               \\\hline
        \texttt{PhaseOracle} &             &          &             &               \\\hline
      \end{tabular}
    }
  \end{center}
  \caption{Experimental Results}
  \label{table-results}
\end{table}

\subsection{Analysis}
\label{Exp:An}
