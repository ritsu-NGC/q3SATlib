\begin{figure*}[t]
  \centering
  \begin{minipage}{\textwidth}
    \centering
    \scalebox{0.9} {
      \Qcircuit @C=0.5em @R=0.2em @!R {
\lstick{\ket{x_a}}     & \qw       & \gate{T}            & \targ     & \qw      & \ctrl{1} & \gate{T^\dagger}       & \targ     & \ctrl{2} & \qw       & \qw      & \qw       & \ctrl{1} & \qw & \qw      & \qw       & \gate{T}            & \targ     & \qw      & \ctrl{1} & \gate{T^\dagger}       & \targ     & \ctrl{2} & \qw       & \qw      & \qw       & \ctrl{1} & \qw      & \qw\\
\lstick{\ket{x_b}}     & \targ     & \gate{T^\dagger}    & \qw       & \ctrl{1} & \targ    & \gate{T^\dagger}       & \ctrl{-1} & \qw      & \targ     & \gate{T} & \targ     & \targ    & \qw & \qw      & \targ     & \gate{T^\dagger}    & \qw       & \ctrl{1} & \targ    & \gate{T^\dagger}       & \ctrl{-1} & \qw      & \targ     & \gate{T} & \targ     & \targ    & \qw      & \qw\\
\lstick{\ket{y}}       & \ctrl{-1} & \gate{T}            & \ctrl{-2} & \targ    & \qw      & \gate{T}               & \qw       & \targ    & \ctrl{-1} & \qw      & \ctrl{-1} & \qw      & \qw & \gate{X} & \ctrl{-1} & \gate{T}            & \ctrl{-2} & \targ    & \qw      & \gate{T}               & \qw       & \targ    & \ctrl{-1} & \qw      & \ctrl{-1} & \qw      & \gate{X} & \qw
}

    }
    \subcaption{Using negated input $x_a x_b \bar{y}$}
    \sublabel{fig-series}
    \scalebox{0.9} {
      \Qcircuit @C=0.5em @R=0.2em @!R {
\lstick{\ket{x_a}}     & \qw       & \gate{T}            & \targ     & \qw      & \ctrl{1} & \gate{T^\dagger}       & \targ     & \ctrl{2} & \qw       & \qw      & \qw       & \ctrl{1} & \qw & \qw       & \gate{T}           & \targ     & \qw      & \ctrl{1} & \gate{T}           & \targ     & \ctrl{2} & \qw       & \qw                & \qw       & \ctrl{1} & \qw      & \qw\\
\lstick{\ket{x_b}}     & \targ     & \gate{T^\dagger}    & \qw       & \ctrl{1} & \targ    & \gate{T^\dagger}       & \ctrl{-1} & \qw      & \targ     & \gate{T} & \targ     & \targ    & \qw & \targ     & \gate{T}           & \qw       & \ctrl{1} & \targ    & \gate{T^{\dagger}} & \ctrl{-1} & \qw      & \targ     & \gate{T^{\dagger}} & \targ     & \targ    & \qw      & \qw\\
\lstick{\ket{y}}       & \ctrl{-1} & \gate{T}            & \ctrl{-2} & \targ    & \qw      & \gate{T}               & \qw       & \targ    & \ctrl{-1} & \qw      & \ctrl{-1} & \qw      & \qw & \ctrl{-1} & \gate{T^{\dagger}} & \ctrl{-2} & \targ    & \qw      & \gate{T}           & \qw       & \targ    & \ctrl{-1} & \qw                & \ctrl{-1} & \qw      & \gate{X} & \qw
}

    }
    \subcaption{Using Boolean Fourier transform calculated $x_a x_b \bar{y}$}
    \sublabel{fig-series2}
  \end{minipage}
  \caption{Phase oracle for $x_a x_b y \oplus x_a x_b \bar{y}$}
  \label{fig-series-all}
  \vspace{-0.5cm}  
\end{figure*}

\section{Motivation}
\label{Mot}
\subsection{CNOT+T Circuit Optimization}
\label{Mot:CnotOpt}
CNOT+T circuits are notable because the T-gates in them can be freely rearranged to optimize T-depth.
Techniques such as matroid partitioning guarantee an optimal T-depth
for them~\cite{bib-amy-matroid}. Given enough ancilla, they can be rearranged to
arbitrarily small T-depth, down to a T-depth of 1 in some cases.

Additionally, there is a straightforward way to reduce the T-count for such circuits.
Note that because the phase is a scalar, the actions of phase gates on different
qubits all multiply together into the same scalar (essentially adding the powers
of $e^{i \pi}$). Therefore, the result of all phase gates in a circuit
can be computed as follows:

\begin{itemize}
\item For every phase gate, calculate the state driving it as a function of the
  input states, which will be a linear function of the same form as the
  $\chi_{\mathbf{k}}$ from the previous section.
\item Multiply the phase driven by the phase gate by this function.
\item Add the phases from all such similar functions.
\item Resynthesize the circuit from the resulting phase polynomial using
  the methods from Sec.~\ref{Pre}.
\end{itemize}

The resulting expression from this method is sometimes called a circuit's \emph{sum of paths}~\cite{bib-amy-cnot}.
The result is that any phase gates driven by the same function will merge together in the phase
polynomial. This means that T-gates can combine into an S-gate or Z-gate, reducing the total T-count.

\begin{example}
  \label{ex-series}
  We attempt to calculate the phase polynomial for Fig.~\ref{fig-series}. This circuit
  is the phase oracle for $x_a x_b y \oplus x_a x_b \bar{y}$ implemented using
  the method detailed in Sec~\ref{Pre:OracleEsop}. It inserts
  Fig.~\ref{fig-toff-mark-matroid} to implement $x_a x_b y$. The second term has a negated
  input, so it inserts X on the $y$ qubit after the first circuit to negate $y$, again
  inserts Fig.~\ref{fig-toff-mark-matroid} to implement the phase, and finally places another
  X on $y$ to return its value to the original.

  Since the first half of the circuit just implements $x_a x_b y$, we already know
  the phase polynomial for this from Eq.~\ref{eq-toff-bool}. The second term, we
  substitute $\bar{y}$ for $y$ in Eq.~\ref{eq-toff-bool}. We add the two phase
  polynomials together and get

  \begin{equation}
    \label{eq-series}
    \begin{aligned}
      &x_a \cdot x_b \cdot y \oplus x_a \cdot x_b \cdot \bar{y} = \frac{1}{2}x_a +
      \frac{1}{2}x_b +\frac{1}{2}(x_a \oplus x_b) \\
      & + \frac{1}{4}y -\frac{1}{4}(x_a \oplus y) - \frac{1}{4}(x_b \oplus y) +
      \frac{1}{4}(x_a \oplus x_b \oplus y)\\
      &\qquad -\frac{1}{4}(x_a \oplus y) - \frac{1}{4}(x_b \oplus y) +
      \frac{1}{4}(x_a \oplus x_b \oplus y)\\
      & + \frac{1}{4}\bar{y} -\frac{1}{4}(x_a \oplus \bar{y})
      - \frac{1}{4}(x_b \oplus \bar{y}) +
      \frac{1}{4}(x_a \oplus x_b \oplus \bar{y})\\
      &\qquad -\frac{1}{4}(x_a \oplus \bar{y}) - \frac{1}{4}(x_b \oplus \bar{y}) +
      \frac{1}{4}(x_a \oplus x_b \oplus \bar{y})
    \end{aligned}
  \end{equation}
  This results in a savings of 3 T-gates. 
\end{example}

Observe from Ex~\ref{ex-series} that the terms containing $\bar{y}$ didn't combine with
those containing $y$ that were otherwise identical. If $x_a x_b \bar{y}$ can be expressed
as a phase polynomial containing $y$ instead, these terms would combine. By taking the
Boolean Fourier transform of $x_a x_b \bar{y}$, we find that it can, in fact, be expressed
using the phase polynomial in Eq.~\ref{eq-toff-bool2}. We use this in the next example.

\begin{equation}
  \label{eq-toff-bool2}
  \begin{aligned}
    &x_a \cdot x_b \cdot \bar{y} = \frac{1}{4}x_a + \frac{1}{4}x_b - \frac{1}{4}y
    - \frac{1}{4}(x_a \oplus x_b) \\
    &\qquad +\frac{1}{4}(x_a \oplus y) + \frac{1}{4}(x_b \oplus y)
    - \frac{1}{4}(x_a \oplus x_b \oplus y)
  \end{aligned}
\end{equation}

\begin{example}
  From the new circuit in Fig.~\ref{fig-series2}, which integrates the Boolean Fourier
  transform calculated $x_a x_b \bar{y}$, the phase polynomial for it
  is now calculated by the addition of Eq.~\ref{eq-toff-bool} and Eq.~\ref{eq-toff-bool2}.
  Observe that all of the $y$ related terms now cancel, and we are left with
  $\frac{1}{2}x_a + \frac{1}{2}x_b +\frac{1}{2}(x_a \oplus x_b)$, which is the
  phase polynomial for a two bit Controlled-Z gate. This means that Fig.~\ref{fig-series}
  did not require any T-gates to implement at all.
\end{example}

A library of such Boolean function decompositions could enable easier sum of paths
optimization for a particular quantum circuit.

The reader may have already noticed that in this simplistic example, a simple Boolean
simplification may have dispensed with the need to even calculate the phase polynomial
in the first place. We utilize a similar observation in our next section.

\subsection{Optimizing T-depth in the Presence of Hadamard Gates}
\label{Mot:tdepth}


\subsection{Synthesizing a Phase Oracle from Standard Methods}
\label{Mot:std}

As a phase oracle implements a single Boolean function by multiplying -1 on certain inputs,
a phase oracle can be constructed as in~\figref{fig-single-target}. 
\subsection{Optimizing the Cubes in an ESOP Expression}
\label{Mot:Lib}

Observe that in Ex.~\ref{ex-esop-synth}, the cubes of two or fewer variables didn't incur
any T-count, while the one with four literals cost 15. It is thus beneficial to find
an equivalent ESOP expression for a circuit that has as few cubes of over 4 literals as
possible. This allows us to reduce the number of T-gates, as well as implement as much of
the circuit as possible using only CNOT+T, which allows the usage of tools such as
sum of paths simplification in Sec.~\ref{Mot:CnotOpt} and matroid partitioning from
~\cite{bib-amy-matroid} to optimize for T-count and T-depth.

\begin{example}
  \label{ex-esop-synth2}
  Let's revisit Fig.~\ref{fig-esop-synth}. Observe that
  $x_2x_3 \oplus x_3x_4 \oplus x_4x_5 \oplus x_2x_4 x_5x_1$ can be rewritten as
  $x_2x_3 \oplus x_3x_4 \oplus x_4x_5 \oplus x_5x_1 \oplus x_1x_2$. This means that
  Fig.~\ref{fig-esop-synth} can actually be implemented using Fig.~\ref{fig-esop-synth2}.
  This new circuit has a T-count of 0. In the next section, we will go over how to
  get as close as possible to something like this for a given Boolean function.

  \begin{figure}[t]
    \centering
    \scalebox{1.0}{
      \Qcircuit @C=1em @R=1.2em {
  \lstick{\ket{x_1}} & \ctrl{1} & \qw      & \qw      & \qw      & \qw      & \qw      & \ctrl{4} & \qw \\
  \lstick{\ket{x_2}} & \gate{Z} & \ctrl{1} & \qw      & \qw      & \qw      & \qw      & \qw      & \qw \\
  \lstick{\ket{x_3}} & \qw      & \gate{Z} & \ctrl{1} & \qw      & \qw      & \qw      & \qw      & \qw \\
  \lstick{\ket{x_4}} & \qw      & \qw      & \gate{Z} & \ctrl{1} & \qw      & \qw      & \qw      & \qw \\
  \lstick{\ket{x_5}} & \qw      & \qw      & \qw      & \gate{Z} & \gate{Z} & \qw      & \gate{Z} & \qw
}

    }
    \caption{Simplified Fig.~\ref{fig-esop-synth}}
    \label{fig-esop-synth2}
    \vspace{-0.5cm}    
  \end{figure}
\end{example}

