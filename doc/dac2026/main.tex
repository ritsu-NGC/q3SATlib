\documentclass[conference]{IEEEtran}

\usepackage{multirow}
\usepackage{float}
\usepackage{cite}
\usepackage{caption}
\usepackage{subcaption}
\usepackage{amsmath,amssymb,amsfonts,amstext}
% \usepackage{algorithmic}
\usepackage[dvipdfmx]{graphicx}
\usepackage{textcomp}
\usepackage{xcolor}
\usepackage{colortbl}
\usepackage[T1]{fontenc}
\usepackage{mdframed}
\usepackage{algorithm, algpseudocode} % texlive-science
\usepackage[braket, qm]{qcircuit}
\usepackage{newtxtext}
\usepackage[normalem]{ulem}
\usepackage{nameref}
\usepackage{tikz}
\usetikzlibrary{arrows.meta, shapes, positioning, calc, fit}

\newcommand*\rtofstarg{\tikz[baseline=(char.base)]{
            \node[shape=circle,double,draw,inner sep=1pt,minimum width=16pt] (char) {S};}}
\newcommand*\rtofsdgtarg{\tikz[baseline=(char.base)]{
            \node[shape=circle,double,draw,inner sep=1pt] (char) {S$^{\dagger}$};}}
\newcommand*\rtoftarg{\tikz[baseline=(char.base)]{
            \node[shape=circle,draw,inner sep=-0.5pt,fill=none] (char) {$\oplus$};}}
\newcommand*\rtofsctrl{\tikz[baseline=(char.base)]{
            \node[shape=circle,double,draw,inner sep=1pt,fill=black, minimum width=3pt] (char) {};}}
\newcommand*\rtofsctrlo{\tikz[baseline=(char.base)]{
            \node[shape=circle,double,draw,inner sep=1pt,fill=white, minimum width=3pt] (char) {};}}
%\newcommand*\rtofsctrlo{\tikz[baseline=(char.base)]{
%            \node[shape=circle,double,draw,inner sep=1pt,fill=white, minimum width=3pt] (char) {};}}
\newcommand*\dashedwire{\tikz{
            \draw[dashed] (0,0) -- (0.5,0); }}

\newcommand*\ctrlrtofs[1]{\push{\rtofsctrl}\qwx[#1]\qw}
\newcommand*\ctrlortofs[1]{\push{\rtofsctrlo}\qwx[#1]\qw}
\newcommand*\targrtofs{\push{\rtofstarg}\qw}
\newcommand*\targrtof{\push{\rtoftarg}\qw}


\makeatletter
\newcommand{\sublabel}[1]{\protected@edef\@currentlabel{\thefigure(\thesubfigure)}\label{#1}}
\makeatother
\captionsetup{subrefformat=parens}
\algrenewcommand\algorithmicindent{0.5em}
\newtheorem{definition}{Definition}
\newtheorem{example}{Example}
\newcommand{\thickhline}{%
    \noalign {\ifnum 0=`}\fi \hrule height 1pt
    \futurelet \reserved@a \@xhline
}
%\def\BibTeX{{\rm B\kern-.05em{\sc i\kern-.025em b}\kern-.08em
%    T\kern-.1667em\lower.7ex\hbox{E}\kern-.125emX}}
\newcommand{\figcaption}[1]{\def\@captype{figure}\caption{#1}}
\newcommand{\tblcaption}[1]{\def\@captype{table}\caption{#1}}

\newcommand{\Del}[1]{\textcolor{red}{\sout{#1}}}
\newcommand{\Add}[1]{\textcolor{blue}{\emph{#1}}}

\newcount\NGCcolor
\NGCcolor=1
\newcount\NGCcolortwo
\NGCcolortwo=1
%\NGCcolor=0
\ifnum\NGCcolortwo<1
  \newcommand{\blueoutt}[1]{\textcolor{blue}{#1}}  
\else
  \newcommand{\blueoutt}[1]{#1}
\fi
  
\ifnum\NGCcolor<1
  \newcommand{\redout}[1]{\textcolor{red}{#1}}
  \newcommand{\blueout}[1]{\textcolor{blue}{#1}}
  \newcommand{\greenout}[1]{\textcolor{green}{#1}}
  \newcommand{\cyannout}[1]{\textcolor{cyan}{#1}}
  \newcommand{\magentaout}[1]{\textcolor{magenta}{#1}}
  \newcommand{\yellowout}[1]{\textcolor{yellow}{\bf #1}}
  \newcommand{\algblueout}[1]{\leavevmode\color{blue}{#1}}
\else
  \newcommand{\redout}[1]{#1}
  \newcommand{\blueout}[1]{#1}
  \newcommand{\greenout}[1]{#1}
  \newcommand{\cyannout}[1]{#1}
  \newcommand{\magentaout}[1]{#1}
  \newcommand{\yellowout}[1]{#1}
  \newcommand{\algblueout}[1]{\leavevmode {#1}}
\fi


% refs for section
\newcommand{\secref}[1]{Sec.~\ref{#1}}
\newcommand{\figref}[1]{Fig.~\ref{#1}}
\newcommand{\tabref}[1]{Table~\ref{#1}}
\newcommand{\eqnref}[1]{(\ref{#1})}

\begin{document}

\title{Synthesis of Phase Oracles Using CNOT+T Circuits and ESOP Minimization}
\author{
    \IEEEauthorblockN{\phantom{David Clarino}}
    \IEEEauthorblockA{\phantom{Ritsumeikan University} \\
      \phantom{dclarino@fc.ritsumei.ac.jp}}
    \and 
    \IEEEauthorblockN{\phantom{Chitranshu Arya}}
    \IEEEauthorblockA{\phantom{Netaji Subhas} \\
      \phantom{University of Technology} \\
      \phantom{chitranshu.arya.ug22@nsut.ac.in}}
    \and
    \IEEEauthorblockN{\phantom{Shigeru Yamashita}}
    \IEEEauthorblockA{\phantom{Ritsumeikan University} \\
      \phantom{ger@cs.ritsumei.ac.jp}}
    \and
    \IEEEauthorblockN{\phantom{Zanhe Qi}}
    \IEEEauthorblockA{\phantom{Ritsumeikan University} \\ \phantom{goose@ngc.is.ritsumei.ac.jp}}
}
%% \author{
%%     \IEEEauthorblockN{David Clarino}
%%     \IEEEauthorblockA{Ritsumeikan University \\
%%       dclarino@fc.ritsumei.ac.jp}
%%     \and 
%%     \IEEEauthorblockN{Chitranshu Arya}
%%     \IEEEauthorblockA{Netaji Subhas \\
%%       University of Technology \\
%%       chitranshu.arya.ug22@nsut.ac.in}
%%     \and
%%     \IEEEauthorblockN{Shigeru Yamashita}
%%     \IEEEauthorblockA{Ritsumeikan University \\
%%       ger@cs.ritsumei.ac.jp}
%%     \and
%%     \IEEEauthorblockN{Zanhe Qi}
%%     \IEEEauthorblockA{Ritsumeikan University \\
%%       goose@ngc.is.ritsumei.ac.jp}
%% }

\maketitle

\begin{abstract}
  Phase oracles are essential components of quantum algorithms such as Grover's algorithm. These circuits apply a phase of $-1$ to an input state when a particular Boolean function $f$ is true. For functions of 3 or fewer variables, these are well known to be exactly implementable using only the CNOT, T, and T$^\dagger$ gates. Amy established that this class of circuits can be provably optimized for T-count and T-depth. We identify a class of Boolean functions of 4 or more variables which can be synthesized using such a circuit, by way of a library of circuits that implement phase oracles of 3 or fewer variables. We then use this to synthesize Boolean functions of 4 or more variables, by minimizing the number of ESOP terms of more than 3 variables and using a library of CNOT+T 3 variable circuits to synthesize cubes of 3 or fewer. This ensures as much of the circuit is implemented using only CNOT+T as possible and can thus be optimized by CNOT+T algorithms such as Tpar. We compare the results against Qiskit's \texttt{PhaseOracle} method and find T-par can optimize such circuits with on average 36.5\% lower T-count and T-depth, and as much as 65\% in some cases.
\end{abstract}

\section{Introduction}
One of quantum computing's most significant results is the development of Grover's
algorithm~\cite{bib-grover1996fast}. This algorithm performs an unstructured search
optimally~\cite{bib-zaika-grov-opt}. Various applications have
been proposed for such an algorithm, including the acceleration of solving NP-hard
problems~\cite{bib-williams-grover-np}.

Grover's algorithm uses a phase oracle, a quantum circuit that applies a phase of
$-1$ when the Boolean function it implements is true, to increase the probability
of returning a fulfilling assignment.

Phase oracles are interesting because a restricted class of them can be implemented
using only phase gates such as the T, T$^\dagger$, and CNOT gates (hereafter
referred to as CNOT+T). In fact, phase oracles of 3 or fewer variables can all be
implemented exactly using this set of gates, and these circuits can be calculated
exactly using the Boolean Fourier transform. Amy et al.~\cite{amy-meet-in-middle}
demonstrates that CNOT+T circuits can be optimized for both T-count and
T-depth~\cite{bib-amy-matroid}.

However, in general, it is impossible to implement Boolean functions of 4 or more
variables using only this gate set~\cite{bib-amy-rm}. There will inevitably be a need to use
a Hadamard (H) gate. The addition of this gate makes optimal solutions non-unique,
and several methods to optimize for T-count and T-depth when including the H gate
are only heuristic ~\cite{amy-meet-in-middle,bib-amy-matroid,bib-amy-rm}.

Phase oracles are straightforward to synthesize from an Exclusive Sum
of Products (ESOP) expression~\cite{bib-phaseoracle}. Since each cube maps to
a Boolean function, it is easy to see that any such function expressible with
cubes of at most 3 variables can be implemented as a CNOT+T circuit. We utilize
this observation in our proposal.

In this work, we propose a solution that utilizes CNOT+T circuits to reduce the
T-count and T-depth for phase oracles of a given Boolean function of arbitrary
literal count.

Our contributions include:

\begin{itemize}
\item A precomputed library of circuits that implement all Boolean functions of
  3 variables or fewer.
\item A synthesis method that uses ESOP minimization to limit the number of
  product terms involving 4 or more variables.
\end{itemize}

We find that our solution... [DCTODO: complete with results]

The rest of this paper is structured as follows: First Sec~\ref{Pre}
introducs some preliminary knowledge. In addition to the basics of
qubits and quantum circuits, this includes the basics of the
mathematics of phase polynomials and their mapping to phase oracles
and CNOT+T circuits. Then Sec.~\ref{Mot} goes through some
motivational examples that inform the general idea of our proposal.
Sec.~\ref{Pro} details the proposal itself, including how to
generate a library of up to 3-qubit phase oracles, and how to use
it to generate phase oracles of 4 or more qubits. In Sec.~\ref{Exp},
we test our results against Qiskit PhaseOracle and analyze our
findings. Finally, Sec.~\ref{Conc} concludes with our findings and
proposes some future research.



\section{Preliminaries}

\subsection{Quantum Gates and Circuits}
CNOT+T circuits consist only of the CNOT, T, and T$^\dagger$ gates. Such circuits can be represented by a path-sum, which captures the algebraic structure of the circuit. Two circuits sharing the same path-sum are functionally equivalent~\cite{pathsum2018}. This property enables a resynthesis process where T gates can be merged.

One complication in this process is the treatment of negated variables. As demonstrated in Fig.~[DCTODO], certain T-gates can be merged more easily using the Boolean Fourier transform of a function rather than by explicitly negating its inputs. A library of such decompositions aids this combination.

\subsection{Phase Oracles}
A phase oracle applies a phase $(-1)$ to any input that satisfies a Boolean function $f$. A common method of synthesizing phase oracles is via an ESOP representation, expressing the function as an XOR of product terms (ANDs) of variables. Given the mapping $|x\rangle|y\rangle \mapsto (-1)^{x \oplus y}|x\rangle|y\rangle$, such circuits can be synthesized using known quantum subroutines (see Fig.~[DCTODO]).


\begin{figure*}[t]
  \centering
  \begin{minipage}{\textwidth}
    \centering
    \scalebox{0.9} {
      \Qcircuit @C=0.5em @R=0.2em @!R {
\lstick{\ket{x_a}}     & \qw       & \gate{T}            & \targ     & \qw      & \ctrl{1} & \gate{T^\dagger}       & \targ     & \ctrl{2} & \qw       & \qw      & \qw       & \ctrl{1} & \qw & \qw      & \qw       & \gate{T}            & \targ     & \qw      & \ctrl{1} & \gate{T^\dagger}       & \targ     & \ctrl{2} & \qw       & \qw      & \qw       & \ctrl{1} & \qw      & \qw\\
\lstick{\ket{x_b}}     & \targ     & \gate{T^\dagger}    & \qw       & \ctrl{1} & \targ    & \gate{T^\dagger}       & \ctrl{-1} & \qw      & \targ     & \gate{T} & \targ     & \targ    & \qw & \qw      & \targ     & \gate{T^\dagger}    & \qw       & \ctrl{1} & \targ    & \gate{T^\dagger}       & \ctrl{-1} & \qw      & \targ     & \gate{T} & \targ     & \targ    & \qw      & \qw\\
\lstick{\ket{y}}       & \ctrl{-1} & \gate{T}            & \ctrl{-2} & \targ    & \qw      & \gate{T}               & \qw       & \targ    & \ctrl{-1} & \qw      & \ctrl{-1} & \qw      & \qw & \gate{X} & \ctrl{-1} & \gate{T}            & \ctrl{-2} & \targ    & \qw      & \gate{T}               & \qw       & \targ    & \ctrl{-1} & \qw      & \ctrl{-1} & \qw      & \gate{X} & \qw
}

    }
    \subcaption{Using negated input $x_a x_b \bar{y}$}
    \sublabel{fig-series}
    \scalebox{0.9} {
      \Qcircuit @C=0.5em @R=0.2em @!R {
\lstick{\ket{x_a}}     & \qw       & \gate{T}            & \targ     & \qw      & \ctrl{1} & \gate{T^\dagger}       & \targ     & \ctrl{2} & \qw       & \qw      & \qw       & \ctrl{1} & \qw & \qw       & \gate{T}           & \targ     & \qw      & \ctrl{1} & \gate{T}           & \targ     & \ctrl{2} & \qw       & \qw                & \qw       & \ctrl{1} & \qw      & \qw\\
\lstick{\ket{x_b}}     & \targ     & \gate{T^\dagger}    & \qw       & \ctrl{1} & \targ    & \gate{T^\dagger}       & \ctrl{-1} & \qw      & \targ     & \gate{T} & \targ     & \targ    & \qw & \targ     & \gate{T}           & \qw       & \ctrl{1} & \targ    & \gate{T^{\dagger}} & \ctrl{-1} & \qw      & \targ     & \gate{T^{\dagger}} & \targ     & \targ    & \qw      & \qw\\
\lstick{\ket{y}}       & \ctrl{-1} & \gate{T}            & \ctrl{-2} & \targ    & \qw      & \gate{T}               & \qw       & \targ    & \ctrl{-1} & \qw      & \ctrl{-1} & \qw      & \qw & \ctrl{-1} & \gate{T^{\dagger}} & \ctrl{-2} & \targ    & \qw      & \gate{T}           & \qw       & \targ    & \ctrl{-1} & \qw                & \ctrl{-1} & \qw      & \gate{X} & \qw
}

    }
    \subcaption{Using Boolean Fourier transform calculated $x_a x_b \bar{y}$}
    \sublabel{fig-series2}
  \end{minipage}
  \caption{Phase oracle for $x_a x_b y \oplus x_a x_b \bar{y}$}
  \label{fig-series-all}
\end{figure*}

\section{Motivation}
\label{Mot}
\subsection{CNOT+T Circuit Optimization}
\label{Mot:CnotOpt}
CNOT+T circuits are notable because matroid partitioning guarantees an optimal T-depth
for them.

Additionally, there is a straightforward way to reduce the T-count for such circuits.
Note that because the phase is a scalar, the actions of phase gates on different
qubits all multiply together into the same scalar (essentially adding the powers
of $e^{i \pi}$). Therefore, the result of all phase gates in a circuit
can be computed as follows:

\begin{itemize}
\item For every phase gate, calculate the state driving it as a function of the
  input states, which will be a linear function of the same form as the
  $\chi_{\mathbf{k}}$ from the previous section.
\item Multiply the phase driven by the phase gate by this function.
\item Add the phases from all such similar functions.
\item Resynthesize the circuit from the resulting phase polynomial using
  the methods from Sec.~\ref{Pre}.
\end{itemize}

The result (sometimes called a circuit's {\it sum of paths}~\cite{bib-amy-cnot})
is that any phase gates driven by the same function will merge
together in the phase polynomial. This means that T-gates
can combine into an S-gate or Z-gate, reducing the total T-count.

This is important because the matroid partitioning algorithm from~\cite{bib-amy-matroid}
that was used to sequence Fig.~\ref{fig-toff-mark-matroid} uses this sum of paths
calculation as a preprocessing step. 

\begin{example}
  \label{ex-series}
  We attempt to calculate the phase polynomial for Fig.~\ref{fig-series}. This circuit
  is the phase oracle for $x_a x_b y \oplus x_a x_b \bar{y}$ implemented using
  the method detailed in Sec~\ref{Pre:OracleEsop}. It inserts
  Fig.~\ref{fig-toff-mark-matroid} to implement $x_a x_b y$. The second term has a negated
  input, so it inserts X on the $y$ qubit after the first circuit to negate $y$, again
  insert Fig.~\ref{fig-toff-mark-matroid} to implement the phase, and finally place another
  X on $y$ to return its value to the original.

  Since the first half of the circuit just implements $x_a x_b y$, we already know
  the phase polynomial for this from Eq.~\ref{eq-toff-bool}. The second term, we
  substitute $\bar{y}$ for $y$ in Eq.~\ref{eq-toff-bool}. We add the two phase
  polynomials together and get

  \begin{equation}
    \label{eq-series}
    \begin{aligned}
      &x_a \cdot x_b \cdot y \oplus x_a \cdot x_b \cdot \bar{y} = \frac{1}{2}x_a +
      \frac{1}{2}x_b +\frac{1}{2}(x_a \oplus x_b) \\
      & + \frac{1}{4}y \qquad -\frac{1}{4}(x_a \oplus y) - \frac{1}{4}(x_b \oplus y) +
      \frac{1}{4}(x_a \oplus x_b \oplus y)\\
      &\qquad -\frac{1}{4}(x_a \oplus y) - \frac{1}{4}(x_b \oplus y) +
      \frac{1}{4}(x_a \oplus x_b \oplus y)\\
      & + \frac{1}{4}\bar{y} \qquad -\frac{1}{4}(x_a \oplus \bar{y})
      - \frac{1}{4}(x_b \oplus \bar{y}) +
      \frac{1}{4}(x_a \oplus x_b \oplus \bar{y})\\
      &\qquad -\frac{1}{4}(x_a \oplus \bar{y}) - \frac{1}{4}(x_b \oplus \bar{y}) +
      \frac{1}{4}(x_a \oplus x_b \oplus \bar{y})
    \end{aligned}
  \end{equation}
  This results in a savings of 3 T-gates. 
\end{example}

Observe from Ex~\ref{eq-series} that the terms containing $\bar{y}$ didn't combine with
those containing $y$ that were otherwise identical. If $x_a x_b \bar{y}$ can be expressed
as a phase polynomial containing $y$ instead, these terms would combine. By taking the
Boolean Fourier transform of $x_a x_b \bar{y}$, we find that it can, in fact, be expressed
using the phase polynomial in Eq.~\ref{eq-toff-bool2}. We use this in the next example.

\begin{equation}
  \label{eq-toff-bool2}
  \begin{aligned}
    &x_a \cdot x_b \cdot \bar{y} = \frac{1}{4}x_a + \frac{1}{4}x_b - \frac{1}{4}y
    - \frac{1}{4}(x_a \oplus x_b) \\
    &\qquad +\frac{1}{4}(x_a \oplus y) + \frac{1}{4}(x_b \oplus y)
    - \frac{1}{4}(x_a \oplus x_b \oplus y)
  \end{aligned}
\end{equation}

\begin{example}
  From the new circuit in Fig.~\ref{fig-series2}, which integrates the Boolean Fourier
  transform calculated $x_a x_b \bar{y}$, the phase polynomial for it
  is now calculated by the addition of Eq.~\ref{eq-toff-bool} and Eq.~\ref{eq-toff-bool2}.
  Observe that all of the $y$ related terms now cancel, and we are left with
  $\frac{1}{2}x_a + \frac{1}{2}x_b +\frac{1}{2}(x_a \oplus x_b)$, which is the
  phase polynomial for a two bit Controlled-Z gate. This means that Fig.~\ref{fig-series}
  did not require any T-gates to implement at all.
\end{example}

A library of such Boolean function decompositions could enable easier sum of paths
optimization for a particular quantum circuit.

The reader may have already noticed that in this simplistic example, a simple Boolean
simplification may have dispensed with the need to even calculate the phase polynomial
in the first place. We utilize a similar observation in our next section.


\subsection{Optimizing the Cubes in an ESOP Expression}
\label{Mot:Lib}

Observe that in Ex.~\ref{ex-esop-synth}, the cubes of two or fewer variables didn't incur
any T-count, while the one with four literals cost 15. It is thus beneficial to find
an equivalent ESOP expression for a circuit that has as few cubes of over 4 literals as
possible. This allows us to reduce the number of T-gates, as well as implement as much of
the circuit as possible using only CNOT+T, which allows the usage of tools such as
sum of paths simplification in Sec.~\ref{Mot:CnotOpt} and matroid partitioning from
~\cite{bib-amy-matroid} to optimize for T-count and T-depth.

\begin{example}
  \label{ex-esop-synth2}
  Let's revisit Fig.~\ref{fig-esop-synth}. Observe that
  $x_2x_3 \oplus x_3x_4 \oplus x_4x_5 \oplus x_2x_4 x_5x_1$ can be rewritten as
  $x_2x_3 \oplus x_3x_4 \oplus x_4x_5 \oplus x_5x_1 \oplus x_1x_2$. This means that
  Fig.~\ref{fig-esop-synth} can actually be implemented using Fig.~\ref{fig-esop-synth2}.
  This new circuit has a T-count of 0. In the next section, we will go over how to
  get as close as possible to something like this for a given Boolean function.

  \begin{figure}[t]
    \centering
    \scalebox{1.0}{
      \Qcircuit @C=1em @R=1.2em {
  \lstick{\ket{x_1}} & \ctrl{1} & \qw      & \qw      & \qw      & \qw      & \qw      & \ctrl{4} & \qw \\
  \lstick{\ket{x_2}} & \gate{Z} & \ctrl{1} & \qw      & \qw      & \qw      & \qw      & \qw      & \qw \\
  \lstick{\ket{x_3}} & \qw      & \gate{Z} & \ctrl{1} & \qw      & \qw      & \qw      & \qw      & \qw \\
  \lstick{\ket{x_4}} & \qw      & \qw      & \gate{Z} & \ctrl{1} & \qw      & \qw      & \qw      & \qw \\
  \lstick{\ket{x_5}} & \qw      & \qw      & \qw      & \gate{Z} & \gate{Z} & \qw      & \gate{Z} & \qw
}

    }
    \caption{Simplified Fig.~\ref{fig-esop-synth}}
    \label{fig-esop-synth2}
  \end{figure}
\end{example}



\section{Proposal}

A sum-of-paths representation of a Boolean function can be derived from the Boolean Fourier decomposition~\cite{odonnell}. However, this decomposition forms a complete basis for functions with 3 variables or fewer when using the CNOT+T gate set. We demonstrate that composing such decompositions enables the exact implementation of a subset of $k$-variable Boolean functions.

We use a modified Boolean Fourier decomposition approach, based on~\cite{amy_duality}, to construct our 3-qubit-or-less library. Our synthesis is ancilla-free and schedules T gates following~\cite{amy2018} to minimize T-depth.

We then apply a modified form of the relative phase function synthesis described in~\cite{amy_relphase} to complete our circuit synthesis.

\section{Pseudo-Boolean Functions and the Boolean Fourier Transform}

One way to realize a Boolean AND is to implement what's called a {\it pseudo-Boolean}
representation~\cite{bib-barenco-elementary,bib-amy-cnot}, which uses a mix of
Boolean values, integer coefficients, and arithmetic to create a mapping $\{0,1\}^n \mapsto \{0,1\}$~\cite{bib-amy-rm}.

Before defining the concept mathematically, first observe Equation~\ref{eq-toff-bool} to see an example of a
pseudo-Boolean function. Here, real-valued coefficients are multiplied with Boolean functions of $x_a$, $x_b$,
and $y$, creating a real number, and those real numbers are then summed together using arithmetic operations
$+$ and $-$. Table~\ref{table-pseudo-toff} shows the values of $x_a \cdot x_b \cdot y$ with respect to
its components, where $T(f)$ is as defined in Equation~\ref{eq-toff-bool}. It can be proven by inspection of
Table~\ref{table-pseudo-toff} that the pseudo-Boolean expression in~\ref{eq-toff-bool} indeed implements
$x_a \cdot x_b \cdot y$. 
\begin{equation}
  \label{eq-toff-bool}
  \begin{aligned}
    &x_a \cdot x_b \cdot y &= &x_a + x_b + y \\\nonumber
    &&&+ 7(x_a \oplus x_b) + 7(x_a \oplus y) + 7(x_b \oplus y) \\\nonumber
    &&&+ (x_a \oplus x_b \oplus y)\\\nonumber
    &&= & (T_{x_a} + T_{x_b} + T_{y} + T_{x_a \oplus x_b} \\\nonumber
    &&&T_{x_a \oplus y} T_{x_b \oplus y} + T_{x_a \oplus x_b \oplus y})
  \end{aligned}
\end{equation}
\begin{table*}[t]
  \begin{minipage}{\textwidth}
    \begin{center}
      \scalebox{1.0} {
        \begin{tabular}{c|c|c|c|c|c|c|c|c|c|c}
          $x_a$ & $x_b$ & $y$ & $T_{x_a}$ & $T_{x_b}$ & $T_{y}$ & $T_{x_a \oplus x_b}$ & $T_{x_a \oplus y}$ & $T_{x_b \oplus y}$ & $T_{x_a \oplus x_b \oplus y}$ & $4x_a \cdot x_b \cdot y$\\\hline
          0     & 0     & 0   & 0         & 0         & 0       & 0                    & 0                  & 0                  & 0                             & 0            \\\hline
          0     & 0     & 1   & 0         & 0         & $1$   & 0                    & $7$             & $7$
                       & $1$                         & 0            \\\hline
          0     & 1     & 0   & 0         & $1$     & 0       & $7$               & $7$             & 0                  & $1$                         & 0            \\\hline
          0     & 1     & 1   & 0         & $1$     & $1$   & $7$               & 0                  & $7$             & 0                             & 0            \\\hline
          1     & 0     & 0   & $1$     & 0         & 0       & $7$               & 0                  & $7$             & $1$                         & 0            \\\hline
          1     & 0     & 1   & $1$     & 0         & $1$   & $7$               & $7$             & 0                  & 0                             & 0            \\\hline
          1     & 1     & 0   & $1$     & $1$     & 0       & 0                    & $7$             & $7$             & 0                             & 0            \\\hline
          1     & 1     & 1   & $1$     & $1$     & $1$   & 0                    & 0                  & 0                  & $1$                         & 4            \\\hline
        \end{tabular}
      }
      \caption{Truth table-like values of the pseudo-Boolean representation of $x_a \cdot x_b \cdot y$}
      \label{table-pseudo-toff}
    \end{center}
  \end{minipage}
  
\end{table*}

\begin{figure*}[t]
  \centering
  \scalebox{0.8} {
    \Qcircuit @C=0.5em @R=0.2em @!R { \\
                             &            &     &          &  & &                           &          &                                &          &                                &          &                             &          & \push{x_a}            &          &     & \\
          \lstick{\ket{x_a}} & \ctrl{3}   & \qw &          &  & & \qw                       & \ctrl{4} & \qw                            & \qw      & \qw                            & \ctrl{4} & \qw                         & \qw      & \gate{\mathrm{T}}     & \ctrl{2} & \qw & \\
                             &            &     &          &  & &                           &          &                                &          & \push{x_b}                     &          &                             &          & \push{x_a \oplus x_b} &          &     & \\
          \lstick{\ket{x_b}} & \ctrl{2}   & \qw & \push{=} &  & & \qw                       & \qw      & \qw                            & \ctrl{2} & \gate{\mathrm{T^{\dagger}}}    & \qw      & \qw                         & \ctrl{2} & \gate{\mathrm{T}}     & \targ    & \qw & \\
                             &            &     &          &  & & \push{y}                  &          & \push{x_a \oplus y}            &          & \push{x_a \oplus x_b \oplus y} &          & \push{x_b \oplus y}         &          &                       &          &     & \\
          \lstick{\ket{y}}   & \ctrl{-1}  & \qw &          &  & & \gate{\mathrm{T}}         & \targ    & \gate{\mathrm{T^{\dagger}}}    & \targ    & \gate{\mathrm{T}}              & \targ    & \gate{\mathrm{T^{\dagger}}} & \targ    & \qw                   & \qw      & \qw & \\
}

  }
  \caption{Implementing $x_a \cdot x_b \cdot y$ using CNOTs and T gates}
  \label{fig-toff-mark}
\end{figure*}


In general, a pseudo-Boolean function is a function such that
\begin{equation}
  \label{eq-pseudo-boolean}
  F(\mathbf{x}) = \sum_{\mathbf{k}} c_a \cdot f_k(\mathbf{x}), c_a \in \mathbb{R}, f_k : \{0,1\}^n \mapsto \{0,1\}
  F(\mathbf{x}) \in \{0,1\}
\end{equation}

In plain language, it is the arithmetic sum of the products of Boolean functions with real valued coefficients, such that
the values of the sum are in the set of binary values $\{0,1\}$~\cite{bib-barenco-elementary}. This
representation is also often referred to in the context of the phases of quantum circuits as the
{\it phase polynomial}~\cite{bib-amy-cnot}.

Phase polynomials and pseudo-Boolean functions important because phase polynomials can be calculated
using the {\it Boolean Fourier transform}, which is shown in Equation~\ref{eq-boolean-fourier}.
\begin{equation}
  \label{eq-boolean-fourier}
  \begin{aligned}
    f(\mathbf{x}) = \sum_{\mathbf{a} \neq 0} \hat{f}(\mathbf{a}) \cdot ( a_0 \cdot x_0 \oplus a_1 \cdot x_1 \cdots \oplus a_{n-1} \cdot x_{n-1}),\\\nonumber
    \hat{f}(\mathbf{a}) \in \mathbb{R} , a_0,a_1,\cdots a_{n-1} \in [0,1]
  \end{aligned}
\end{equation}

$\hat{f}(\mathbf{a})$ is a set of $2^{n}-1$ coefficients (one for each $\mathbf{a}$, except for the trivial
case $\mathbf{a} = 0$) associated with the pseudo-Boolean function, and
$( a_0 \cdot x_0 \oplus a_1 \cdot x_1 \cdots \oplus a_{n-1} \cdot x_{n-1}) $ are linear functions
that form a basis for the pseudo-Boolean function, and are maps $[0,1]^n \mapsto [0,1]$. $\hat{f}$ is referred to
as the {\it phase polynomial} of the quantum Boolean circuit.

Observe that a T-gate adds to the phase no matter which qubit it acts on.
For example, a T-gate acting on bit 0 creates the mapping $\ket{1}_0\ket{1}_1 \mapsto \ket{1}_0\ket{1}_1$ while
a T-gate acting on bit 1 also creates the same mapping. Therefore, $\hat{f}(\mathbf{a})$ sets the number of
phase gates, such as T-gates and T$^{\dagger}$ gates, that are used in the circuit. A Boolean function thus can
have a T-count up to $2^{n}-1$.

It is easy to see then that the $\hat{f}(\mathbf{a})$ that has the minimal number of odd entries for each
$\mathbf{a}$ also minimizes the T-count~\cite{bib-amy-rm}. Observe also, that the linear functions
$( a_0 \cdot x_0 \oplus a_1 \cdot x_1 \cdots \oplus a_{n-1} \cdot x_{n-1}) $ can be implemented
entirely by CNOT gates. This means that Equation~\ref{eq-boolean-fourier} can be realized entirely by
CNOT gates and phase gates, known as a {\it CNOT+T network}~\cite{bib-amy-cnot}. 

Optimizing the T-depth of such a CNOT+T network is done using the method in~\cite{bib-amy-matroid}.

\section{Phase Polynomial Calculation for the Three Qubit Case}
\label{Chap:Bool-pbool3q}

This section demonstrates how to calculate the phase polynomial for a three input Boolean function, using the
Boolean Fourier transform~\cite{bib-odonnell}. This section uses only a simplified version
of the Boolean Fourier transform introduced in~\cite{bib-odonnell}, but with notation taken from
\cite{bib-barenco-elementary} and~\cite{bib-amy-cnot}. The notation is chosen to more clearly demonstrate how
the equation may be implemented using quantum gates defined in the CNOT+T basis. 

First, the inner product to be used for the Boolean Fourier transform is defined below

\begin{equation}
  \label{eq-inner-prod}
  \langle f(\mathbf{x}) , g(\mathbf{x}) \rangle = \sum_{\mathbf{x}} \text{INNPROD}(f(\mathbf{x}),g(\mathbf{x})), \mathbf{x} \in \{0,1\}^n
\end{equation}
\begin{equation}  
  \text{INNPROD}(f(\mathbf{x}),g(\mathbf{x})) =
  \begin{cases}
    7 & \text{if $f(\mathbf{x})\neq g(\mathbf{x})$} \\
    1 & \text{if  $f(\mathbf{x})= g(\mathbf{x})$} 
  \end{cases}
\end{equation}

\cite{bib-odonnell} states that the {\it parity functions}, the set of exclusive sums of the input variables
$\chi(x_0,x_1,\cdots,x_n)$ shown in Equation~\ref{eq-bool-basis}, form an orthogonal basis with respect to the
inner product defined in Equation~\ref{eq-inner-prod}. In this equation, there is a factor of $\frac{1}{2^n}$ so
that $\langle f(\mathbf{x}), f(\mathbf{x}) \rangle = 1$, making the basis orthonormal.
\begin{equation}
  \label{eq-bool-basis}
  \begin{split}
  \chi(x_0,x_1,\cdots,x_n)_{\mathbf{a}} = a_0 \cdot x_0 \oplus a_1 \cdot x_1 \cdots \oplus a_{n-1} \cdot x_{n-1},\\
  a_0,a_1,\cdots a_{n-1} \in [0,1]
  \end{split}
\end{equation}

Because for each $\chi_{\mathbf{a}}$ and $\chi_{\mathbf{b}}$, there is
$\langle \chi_{\mathbf{a}}, \chi_{\mathbf{b}} \rangle = 0$, calculating $\hat{f}(\mathbf{a})$ for a specific
value of $\mathbf{a}$ simply involves taking the inner product of the full function $f(\mathbf{x})$ and the
basis state $\chi(\mathbf{x})_{\mathbf{a}}$ as shown in Equation~\ref{eq-fhat-prod}. This multiplication is
repeated for every value of $\mathbf{a}$ until all of $\hat{f}(\mathbf{a})$ is calculated.
\begin{equation}
  \label{eq-fhat-prod}
  \hat{f}(\mathbf{a}) = \langle f(\mathbf{x}), \chi(\mathbf{x})_{\mathbf{a}} \rangle
\end{equation}

\begin{example}
  As an example, let's try to derive the coefficients from Equation~\ref{eq-toff-bool}. Here
  $\mathbf{a} = (a_{x_a}, a_{x_b}, a_y)$.

  Because the trivial case ($\mathbf{a}= 000$) is not included, let's start with $\mathbf{a} = 100$ with
  $\chi_{100} = x_a$.
  Table~\ref{table-ex-100} enumerates its values in truth table form for convenience, as well as the
  value of each of the $(-1)^{f(\mathbf{x})} (-1)^{\chi_{001}(\mathbf{x})}$ terms in the inner product
  $\langle f(\mathbf{x}), \chi_{001} \rangle$

  \begin{table}[h]
    \begin{center}
      \begin{tabular}{c|c|c|c|c|c}
        \hline
        $x_a$ & $x_b$ & $y$ & $f(\mathbf{x})$ & $\chi_{100}$ & INNPROD($\chi_{100},f(\mathbf{x}$) \\\hline
        0 & 0 & 0 & 0 & 0 & 1\\\hline
        0 & 0 & 1 & 0 & 0 & 1\\\hline
        0 & 1 & 0 & 0 & 0 & 1\\\hline
        0 & 1 & 1 & 0 & 0 & 1\\\hline
        1 & 0 & 0 & 0 & 1 & 7\\\hline
        1 & 0 & 1 & 0 & 1 & 7\\\hline
        1 & 1 & 0 & 0 & 1 & 7\\\hline      
        1 & 1 & 1 & 1 & 1 & 1\\\hline
      \end{tabular}
      \caption{The value of $\chi_{100}$ and the inner product $\langle f(\mathbf{x}), \chi_{100} \rangle$}
      \label{table-ex-100}
    \end{center}
  \end{table}

  
  The terms in the right most column can then be summed just like in Equation~\ref{eq-inner-prod}:
  \begin{align}
      &\hat{f}(100) = &\langle f(\mathbf{x}), \chi_{100} \rangle = &( 1 + 1 + 1 + 1 - 1 - 1 - 1 + 1) \\\nonumber
      &&& = ( 2 ) = 1\nonumber\\\nonumber
  \end{align}
  This then gives:
  \begin{equation}
    f(\mathbf{x}) = x_a + \sum_{\mathbf{a} \neq 000,100} \hat{f}(\mathbf{a}) \cdot \chi_{\mathbf{a}}
  \end{equation}

  In the interest of conciseness, the rest of the calculation is omitted. But the reader can easily perform the
  rest to verify its equivalence with Equation~\ref{eq-toff-bool}.
\end{example}

The pseudo-Boolean function from Eq.~\ref{eq-toff-bool} is implemented as a circuit in Fig~\ref{fig-toff-mark}.


\section{Experimental Results}
\label{Exp}
\subsection{Experiment Setup}
\label{Exp:Set}
We compare the ability of T-par to optimize circuits produced using our methodology ({\it Proposed}) against those
of Qiskit's \texttt{PhaseOracle} method. This method takes in an ESOP expression and applies the synthesis in
Sec.~\ref{Mot:Lib}. Since \texttt{PhaseOracle} only generates Control-Z gates,
we further decompose these using the methods described in Sec.~\ref{Pro}.  For cubes of $n \geq 4$,
the decomposition from Sec.~\ref{Pro:n4} is used. \texttt{PhaseOracle} synthesizes expressions in the same
manner as Ex.~\ref{ex-series} and so cannot take advantage of the sum of paths simplification that
the proposed method uses.

We randomly generate ESOP expressions to feed into both of our methods. The main code is run in python,
while running T-par as a system call to a C++ binary and EXORCISM-4 through a Python-C++ API.

Here we propose two kinds of ESOP Expressions to test:

\begin{itemize}
\item (N3) : Randomly generated ESOP expressions are constrained to those of $n \leq 3$. This tests the
  ability of the precomputed 3bit library to enable simplification of pure CNOT+T circuits through easier
  sum of paths calculations.
\item (N4) : Randomly generated ESOP cubes are constrained as follows :
  those with up to 2 literals (i.e. no T gates)  are generated with 2/5 probability,
  3 literals (has T-gates but CNOT+T) with 2/5 probability, and 4 or more literals with 1/5 probability.
  This tests the ability of the proposed method to find a simplified solution in its EXORCISM-4 step.
  It creates an expression which can already be mostly expressed as a CNOT+T network due to its cubes of
  3 or fewer literals in order to create a more fair comparison between Qiskit's method and the Proposed
  method.
\end{itemize}

We test these for $n=4-12$ literals. The ESOP expressions are randomly generated according to two groups
based on the number of cubes $k$ in them. These are further divided into two groups:

\begin{itemize}
\item ($k=2^{n-1}$) This randomizes the number of cubes between $2n$ and $2^(n-1)$.This increases the chances
  that inserted subcircuits will have common literals that allow them to be simplified and have reduced T-count,
  while making it harder for T-par to partition any cubes $n \geq 4$ for T-depth scheduling.
\item ($k=n$) This randomizes the number of cubes between $1$ and $2n-1$. This makes the arrangements sparser,
  allowing for better parallelization by T-par, but making it harder to have common literals that can combine
  to reduce the number of cubes.
\end{itemize}

For each combination of N3/N4 and $k=2^{n-1}$/$k=n$, we generated 100 random ESOP expressions and fed them
to both Qiskit and the Proposed method. We take the T-count of the circuit generated by Qiskit $tcnt_q$ and
the T-count of the circuit generated by Proposed $tcnt_p$ and take the percentage difference
$\Delta T-count\% = (tcnt_q - tcnt_p) / tcnt_q$ and take the arithmetic average of $\Delta T-count\%$ for
Avg $\Delta T-count\%$. We do the same for T-depth and get Avg $\Delta$T-depth\%. The results are
shown in Table~\ref{table-results}.
%\subsection{Results}
%\label{Exp:Res}

\begin{table}[t]
  \begin{center}
    \scalebox{1.2} {
      \begin{tabular}{c|c|c|c}\hline
                            &                   & $k=2^{n-1}$ & $k=n$    \\\hline
        \multirow{2}{*}{N3} & Avg $\Delta$T-count\% &  0.222      & 0.115    \\\cline{2-4}
                            & Avg $\Delta$T-depth\% &  0.228      & 0.199    \\\hline
        \multirow{2}{*}{N4} & Avg $\Delta$T-count\% &  0.655      & 0.440    \\\cline{2-4}
                            & Avg $\Delta$T-depth\% &  0.635      & 0.425    \\\hline

      \end{tabular}        
    }
  \end{center}
  \caption{Experimental Results}
  \label{table-results}
  \vspace{-0.5cm}
\end{table}

\subsection{Analysis}
The most significant result is \{$k=2^{n-1}$,N4\}, which reports reductions of over 60\% over
Qiskit's method. This is because EXORCISM-4 can find more cubes to simplify when they are denser
and share more literals.

Next is \{$k=n$,N4\}. Despite the sparsity, EXORCISM-4 was still able to find cubes to simplify
and that contributes to the reduction. This is notable because the EXORCISM-4 produced ESOP
expressions had a lower average number of literals in each cube.

\{$k=2^{n-1}$,N3\} showed a more moderate but significant advantage. This is because EXORCISM-4
can find fewer advantages over the standard case since it is already in a form that allows
the T-par algorithm to work optimally. However, there is still a significant advantage because
the 3-bit library of functions allows a more efficient sum of paths calculation to combine
T-gates into Clifford phase gates.

Finally \{$k=n$,N3\} shows the least amount of advantage because of its sparsity. This is most
notable in T-depth, because T-par can already efficiently schedule the circuit produced by Qiskit.
This means that neither 

Despite the clear advantages conferred by the Proposed method, there remain a few corner
cases where the Qiskit circuit outperformed the one generated by the Proposed method, such as
in the case for the expression $(x_3 \land x_7) \oplus (\lnot x_6 \land x_4) \oplus (x_3 \land \lnot x_7 \land \lnot x_6) \oplus (\lnot x_6 \land x_5 \land \lnot x_3) \oplus (x_2 \land x_4) \oplus (\lnot x_8 \land \lnot x_3) \oplus (\lnot x_7 \land x_3) \oplus (x_1 \land \lnot x_8) \oplus (\lnot x_1 \land \lnot x_8) \oplus (x_2 \land x_4 \land x_3) \oplus (x_3 \land x_5) \oplus (\lnot x_7 \land x_6)$.
Evaluating this will require much more material than can be reasonably fit into this paper,
and we will leave it as a topic for future research.

\label{Exp:An}



\section{Conclusion}
We introduced a method that synthesizes a quantum circuit from an ESOP expression, in such a way that
it can be optimized by CNOT+T optimization methods such as Tpar. We find that the T-count
and T-depth produced by such an optimization method is less than the action of the same optimization
method on Qiskit's built in operations, by an average of 36.5\%. However, we find corner cases where
the method underperforms the default Qiskit behavior. Further investigation into this will be a topic of
future research.
\label{Conc}


% DCTODO: Add Evaluation, Results, and Conclusion Sections

\bibliographystyle{unsrt}
\bibliography{ref}

\end{document}
