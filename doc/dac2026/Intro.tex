\section{Introduction}
One of quantum computing's most significant results is the development of Grover's
algorithm~\cite{bib-grover1996fast}. This algorithm performs an unstructured search
optimally~\cite{bib-zaika-grov-opt}. Various applications have
been proposed for such an algorithm, including the acceleration of solving NP-hard
problems~\cite{bib-williams-grover-np}.

Grover's algorithm uses a phase oracle, a quantum circuit that applies a phase of
$-1$ when the Boolean function it implements is true, to mark search hits in the
database.

In fault-tolerant quantum computing, there is a need to optimize these circuits for
the number of T-gates and T$^{\dagger}$-gates, known as the {\it T-count}, as well
as the longest such chain of gates, known as the {\it T-depth} DCTODOrefs(amy).
A restricted class of circuits using only phase gates such as the T, T$^\dagger$,
as well as CNOT gates (hereafter referred to as {\it CNOT+T}), is well known to be
efficiently optimizable for T-count and
T-depth~\cite{amy-meet-in-middle,bib-amy-matroid,bib-selinger-tdepthon}. In fact,
phase oracles of 3 or fewer variables can all be implemented exactly using this set
of gates~\cite{bib-amy-cnot}, and these circuits can be calculated exactly using the
Boolean Fourier transform.

However, the vast majority of Boolean functions of 4 or more variables cannot be
implemented using only this gate set~\cite{bib-amy-rm}. There will inevitably be a need to use
a Hadamard (H) gate. The addition of this gate adds rotations that CNOT+T optimization
methods cannot account for. Many existing methods to optimize for T-count and T-depth try to
account for the H-gate in the general case by introducing heuristic methods to partition around
it~\cite{amy-meet-in-middle,bib-amy-matroid,bib-amy-rm} or gadgetization to eliminate it
entirelyDCTODOheyfron. The T-depth of methods which partition according to H-gates, such
as~\cite{bib-amy-matroid} thus relates directly to how well it can partition these H-gates.

While phase oracle generation methods in general have not been explored extensively, there
exist oracle generation methods that focus on driving a Boolean function on a single qubit's
state while minimizing the T-count~\cite{bib-meuli-mult}. It is straightforward to implement
a phase oracle from one of these circuits by generating an oracle and using it to drive a
Z-gate on an ancilla qubit, and then applying the oracle again to return the state. However,
as we will see later, these types of constructions provide an obstacle to optimizing their
T-depth, owing to their use of Hadamard gates.

In this work, we demonstrate that Exclusive Sum of Products (ESOP) generation of phase oracles,
most prominently implemented in Qiskit~\cite{bib-phaseoracle}, produce circuits that are more
amenable to such optimization techniques. We additionally propose several extensions to it
that make optimization easier, by expressing as much of the circuit as CNOT+T circuits as
possible.

Our contributions include:

\begin{itemize}
\item A precomputed library of circuits that implement all Boolean functions of
  3 variables or fewer.
\item A synthesis method that uses ESOP minimization to limit the number of
  product terms involving 4 or more variables.
\end{itemize}

We find that our solution can reduce T-count and T-depth by an average of 36.5\%
and over 65\% in some cases. We examine those cases and which ones the method
is best suited for.DCTODO

The rest of this paper is structured as follows: First Sec~\ref{Pre}
introduces some preliminary knowledge. In addition to the basics of
qubits and quantum circuits, this includes the basics of the
mathematics of phase polynomials and their mapping to phase oracles
and CNOT+T circuits. Then Sec.~\ref{Mot} goes through some
motivational examples that inform the general idea of our proposal.
Sec.~\ref{Pro} details the proposal itself, including how to
generate a library of up to 3-qubit phase oracles, and how to use
it to generate phase oracles of 4 or more qubits. In Sec.~\ref{Exp},
we test our results against Qiskit PhaseOracle and analyze our
findings. Finally, Sec.~\ref{Conc} concludes with our findings and
proposes some future research.

