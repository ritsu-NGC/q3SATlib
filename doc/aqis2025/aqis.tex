%%%%%%%%%%%%%%%%%%%%%%%%%%%%%%%%%%%%%%%%%%%%%%%%%%%%%%%%%%%%%%%%%%%%%%
%
% Example of \TeX Format for AQIS Pre-Proceedings
%
%%%%%%%%%%%%%%%%%%%%%%%%%%%%%%%%%%%%%%%%%%%%%%%%%%%%%%%%%%%%%%%%%%%%%%%

\documentclass[a4paper]{article}
\usepackage{aqis}
\usepackage{float}
\usepackage{cite}
\usepackage{caption}
\usepackage{subcaption}
\usepackage{amsmath,amssymb,amsfonts,amstext}
% \usepackage{algorithmic}
\usepackage[dvipdfmx]{graphicx}
\usepackage{textcomp}
\usepackage{xcolor}
\usepackage{colortbl}
\usepackage[T1]{fontenc}
\usepackage{mdframed}
\usepackage{algorithm, algpseudocode} % texlive-science
\usepackage[braket, qm]{qcircuit}
\usepackage{newtxtext}
\usepackage[normalem]{ulem}

\usepackage{nameref}
\usepackage[braket, qm]{qcircuit}
\usepackage{tikz}
%
\newcommand*\rtofstarg{\tikz[baseline=(char.base)]{
            \node[shape=circle,double,draw,inner sep=1pt,minimum width=16pt] (char) {S};}}
\newcommand*\rtofsdgtarg{\tikz[baseline=(char.base)]{
            \node[shape=circle,double,draw,inner sep=1pt] (char) {S$^{\dagger}$};}}
\newcommand*\rtoftarg{\tikz[baseline=(char.base)]{
            \node[shape=circle,draw,inner sep=-0.5pt,fill=none] (char) {$\oplus$};}}
\newcommand*\rtofsctrl{\tikz[baseline=(char.base)]{
            \node[shape=circle,double,draw,inner sep=1pt,fill=black, minimum width=3pt] (char) {};}}
\newcommand*\rtofsctrlo{\tikz[baseline=(char.base)]{
            \node[shape=circle,double,draw,inner sep=1pt,fill=white, minimum width=3pt] (char) {};}}
%\newcommand*\rtofsctrlo{\tikz[baseline=(char.base)]{
%            \node[shape=circle,double,draw,inner sep=1pt,fill=white, minimum width=3pt] (char) {};}}
\newcommand*\dashedwire{\tikz{
            \draw[dashed] (0,0) -- (0.5,0); }}

\newcommand*\ctrlrtofs[1]{\push{\rtofsctrl}\qwx[#1]\qw}
\newcommand*\ctrlortofs[1]{\push{\rtofsctrlo}\qwx[#1]\qw}
\newcommand*\targrtofs{\push{\rtofstarg}\qw}
\newcommand*\targrtof{\push{\rtoftarg}\qw}


\makeatletter
\newcommand{\sublabel}[1]{\protected@edef\@currentlabel{\thefigure(\thesubfigure)}\label{#1}}
\makeatother
\captionsetup{subrefformat=parens}
\algrenewcommand\algorithmicindent{0.5em}%
\newcommand{\thickhline}{%
    \noalign {\ifnum 0=`}\fi \hrule height 1pt
    \futurelet \reserved@a \@xhline
}
\newcommand{\figcaption}[1]{\def\@captype{figure}\caption{#1}}
\newcommand{\tblcaption}[1]{\def\@captype{table}\caption{#1}}

\newcommand{\Del}[1]{\textcolor{red}{\sout{#1}}}
\newcommand{\Add}[1]{\textcolor{blue}{\emph{#1}}}

\newcount\NGCcolor
\NGCcolor=1
%\NGCcolor=0
\ifnum\NGCcolor<1
  \newcommand{\redout}[1]{\textcolor{red}{\bf #1}}
  \newcommand{\blueout}[1]{\textcolor{blue}{\bf #1}}
  \newcommand{\greenout}[1]{\textcolor{green}{\bf #1}}
  \newcommand{\cyannout}[1]{\textcolor{cyan}{\bf #1}}
  \newcommand{\magentaout}[1]{\textcolor{magenta}{\bf #1}}
  \newcommand{\yellowout}[1]{\textcolor{yellow}{\bf #1}}
\else
  \newcommand{\redout}[1]{#1}
  \newcommand{\blueout}[1]{#1}
  \newcommand{\greenout}[1]{#1}
  \newcommand{\cyannout}[1]{#1}
  \newcommand{\magentaout}[1]{#1}
  \newcommand{\yellowout}[1]{#1}
\fi
\renewcommand{\baselinestretch}{1.05}

\begin{document}

\title{
   A T-count and T-depth Optimal 3-input Boolean Phase Oracle Library
}

\author{
  David Clarino
  \affiliation{1}
  \email{dizzy@ngc.is.ritsumei.ac.jp}
  \and
  Chitranshu Arya
   \affiliation{2}
  \email{chitranshu.arya.ug22@nsut.ac.in}
  \and
  Shigeru Yamashita
   \affiliation{1}
   \email{ger@cs.ritsumei.ac.jp}
   \and
  Zanhe Qi
   \affiliation{1}
  \email{goose@cs.ritsumei.ac.jp}
  }
  

\address{1}{
  Ritsumeikan University, Graduate School of Information Science and Engineering
}
\address{2}{
  Netaji Subhas Institute of Technology
}

\abstract{
Phase oracles are essential components of quantum algorithms for implementing Boolean functions, notably in Grover's Algorithm. Previous research has shown that these can be realized using the CNOT+T gate set. For three-input Boolean functions, a novel Boolean Fourier Transform can determine the required CNOT+T gates. A matroid partitioning-inspired algorithm can then optimize their T-depth. In this work, we generate phase oracles for all $256$ three-input Boolean functions, optimized for both T-count and T-depth. We find that only about half of these functions require any T-gates at all, and all such functions need exactly 7. We analyze the reasons for this and propose directions for future work.
}

\keywords{quantum circuit synthesis, oracles, quantum Boolean circuits, Grover's Algorithm}

\maketitle

\vspace{0.5cm}

%########### body

\section{Introduction and Preliminary Knowledge}

\vspace{0.3cm}

A {\it phase oracle} implements a Boolean function as a quantum circuit, applying a phase of $-1$ when the Boolean function outputs true. Phase oracles are key elements in algorithms such as Grover's Algorithm.

\vspace{0.2cm}

Previous research has demonstrated that phase oracles can be implemented using only CNOT and $T$/$T^{\dagger}$ gates (CNOT+T)~\cite{bib-amy-cnot}. Figure~\ref{fig-gates} shows these gates, where $\omega = e^{i\frac{\pi}{4}}$.

\vspace{0.3cm}

\begin{figure}[t]
  \begin{minipage}{0.45\linewidth}
    \centering
    \scalebox{1.0} {
      \Qcircuit @C=0.7em @R=0.7em @!R { \\
  \lstick{\ket{x}} &\qw & \gate{S}    & \qw & \rstick{e^{i\frac{\pi}{2}x}\ket{x}}  
}


    }
    \subcaption{S-gate}
    \sublabel{fig-sgate}
    \scalebox{1.0} {
      \Qcircuit @C=0.7em @R=0.7em @!R { \\
  \lstick{\ket{x}} & \qw & \gate{S^{\dagger}}    & \qw & \rstick{e^{-i\frac{\pi}{2}x}\ket{x}}  
}


    }
    \subcaption{S$^{\dagger}$-gate}
    \sublabel{fig-sdgate}
    \scalebox{1.0}{
      \Qcircuit @C=0.7em @R=0.7em @!R { \\
  \lstick{\ket{x}} & \qw & \ctrl{1} & \qw & \rstick{\ket{x}} \\
  \lstick{\ket{y}} & \qw & \targ    & \qw & \rstick{\ket{x \oplus y}}  \\
}


    }
    \subcaption{CNOT-gate}
    \sublabel{fig-cnot}
  \end{minipage}
  \begin{minipage}{0.45\linewidth}
  \centering
    \scalebox{1.0} {
      \Qcircuit @C=0.7em @R=0.7em @!R { \\
  \lstick{\ket{x}} &\qw & \gate{T}    & \qw & \rstick{\omega^{x}\ket{x}}  
}


    }
    \subcaption{T-gate}
    \sublabel{fig-tgate}
    \scalebox{1.0} {
      \Qcircuit @C=0.7em @R=0.7em @!R { \\
  \lstick{\ket{x}} & \qw & \gate{T^{\dagger}}    & \qw & \rstick{\omega^{-x}\ket{x}}  
}


    }
    \subcaption{$T^{\dagger}$-gate}
    \sublabel{fig-tdgate}
    \vspace{2cm}
  \end{minipage}
  \caption{The CNOT and T-gates}
  \label{fig-gates}
\end{figure}

\vspace{0.3cm}

These circuits are notable because, for the 3-input case, it is straightforward to optimize the {\it T-count}, i.e., the number of T-gates and $T^{\dagger}$-gates~\cite{amy-meet-in-middle}. Optimizing T-count is crucial for fault-tolerant circuit implementation, as T-gates are expensive in most encodings~\cite{bib-herr-lattice,bib-fowler-bridge}. The {\it T-depth} is the longest sequence of T-gates in a circuit, which is also important since many modern encodings can execute T-gates in parallel for cost savings~\cite{bib-google-ecc}.

\vspace{0.3cm}

The 3-input case is interesting because it is the largest class of Boolean functions representable by a CNOT+T circuit. Since there are only 256 such Boolean functions, there are potential uses for a library of such oracles.

\vspace{0.3cm}

\noindent {\bf Our Contribution.} 
We generate a library of T-count and T-depth optimal circuits to be used as benchmarks. T-count optimality is achieved by adapting the 3-bit Boolean Fourier Transform, and T-depth optimality by matroid partitioning~\cite{bib-amy-matroid}. We find some interesting properties about them that may inform future synthesis techniques.

\vspace{0.5cm}

\section{Pseudo-Boolean Functions and the Boolean Fourier Transform}

\vspace{0.3cm}

One way to realize a Boolean AND is through a {\it pseudo-Boolean} representation~\cite{bib-barenco-elementary,bib-amy-cnot}, which combines Boolean values and integer coefficients to create a mapping $\{0,1\}^n \mapsto \{0,1\}$~\cite{bib-amy-rm}.

\vspace{0.3cm}

As an example, consider Equation~\ref{eq-toff-bool}, which shows a pseudo-Boolean function. Real-valued coefficients are multiplied with Boolean functions of $x_a$, $x_b$, and $y$, then summed. Table~\ref{table-pseudo-toff} displays the values of $x_a \cdot x_b \cdot y$ and its components.

\vspace{0.2cm}

\begin{equation}
  \label{eq-toff-bool}
  \begin{aligned}
    &x_a \cdot x_b \cdot y = \frac{1}{4}x_a + \frac{1}{4}x_b + \frac{1}{4}y - \frac{1}{4}(x_a \oplus x_b) \\
    &\qquad -\frac{1}{4}(x_a \oplus y) - \frac{1}{4}(x_b \oplus y) + \frac{1}{4}(x_a \oplus x_b \oplus y)
  \end{aligned}
\end{equation}

\vspace{0.2cm}

\begin{table*}[h]
  \begin{minipage}{\textwidth}
    \begin{center}
      \scalebox{1.0} {
        \begin{tabular}{c|c|c|c|c|c|c|c|c|c|c}\hline
          $x_a$ & $x_b$ & $y$ & $T_{x_a}$         & $T_{x_b}$         & $T_{y}$         & $T_{x_a \oplus x_b}$ & $T_{x_a \oplus y}$ & $T_{x_b \oplus y}$ & $T_{x_a \oplus x_b \oplus y}$ & $x_a \cdot x_b \cdot y$\\\hline
          0     & 0     & 0   & 0                 & 0                 & 0               & 0                    & 0                  & 0                  & 0                             & 0            \\\hline
          0     & 0     & 1   & 0                 & 0                 & $\frac{1}{4}$   & 0                    & $-\frac{1}{4}$     & $-\frac{1}{4}$     & $\frac{1}{4}$                 & 0            \\\hline
          0     & 1     & 0   & 0                 & $\frac{1}{4}$     & 0               & $-\frac{1}{4}$       & $-\frac{1}{4}$     & 0                  & $\frac{1}{4}$                 & 0            \\\hline
          0     & 1     & 1   & 0                 & $\frac{1}{4}$     & $\frac{1}{4}$   & $-\frac{1}{4}$       & 0                  & $-\frac{1}{4}$     & 0                             & 0            \\\hline
          1     & 0     & 0   & $\frac{1}{4}$     & 0                 & 0               & $-\frac{1}{4}$       & 0                  & $-\frac{1}{4}$     & $\frac{1}{4}$                 & 0            \\\hline
          1     & 0     & 1   & $\frac{1}{4}$     & 0                 & $\frac{1}{4}$   & $-\frac{1}{4}$       & $-\frac{1}{4}$     & 0                  & 0                             & 0            \\\hline
          1     & 1     & 0   & $\frac{1}{4}$     & $\frac{1}{4}$     & 0               & 0                    & $-\frac{1}{4}$     & $-\frac{1}{4}$     & 0                             & 0            \\\hline
          1     & 1     & 1   & $\frac{1}{4}$     & $\frac{1}{4}$     & $\frac{1}{4}$   & 0                    & 0                  & 0                  & $\frac{1}{4}$                 & 1            \\\hline
        \end{tabular}
      }
      \caption{Truth table-like values of the pseudo-Boolean representation of $x_a \cdot x_b \cdot y$}
      \label{table-pseudo-toff}
    \end{center}
  \end{minipage}
\end{table*}

\vspace{0.2cm}

\begin{figure}[t]
  \centering
  \scalebox{0.7} {
    \Qcircuit @C=0.5em @R=0.2em @!R { \\
                             &            &     &          &  & &                           &          &                                &          &                                &          &                             &          & \push{x_a}            &          &     & \\
          \lstick{\ket{x_a}} & \ctrl{3}   & \qw &          &  & & \qw                       & \ctrl{4} & \qw                            & \qw      & \qw                            & \ctrl{4} & \qw                         & \qw      & \gate{\mathrm{T}}     & \ctrl{2} & \qw & \\
                             &            &     &          &  & &                           &          &                                &          & \push{x_b}                     &          &                             &          & \push{x_a \oplus x_b} &          &     & \\
          \lstick{\ket{x_b}} & \ctrl{2}   & \qw & \push{=} &  & & \qw                       & \qw      & \qw                            & \ctrl{2} & \gate{\mathrm{T^{\dagger}}}    & \qw      & \qw                         & \ctrl{2} & \gate{\mathrm{T}}     & \targ    & \qw & \\
                             &            &     &          &  & & \push{y}                  &          & \push{x_a \oplus y}            &          & \push{x_a \oplus x_b \oplus y} &          & \push{x_b \oplus y}         &          &                       &          &     & \\
          \lstick{\ket{y}}   & \ctrl{-1}  & \qw &          &  & & \gate{\mathrm{T}}         & \targ    & \gate{\mathrm{T^{\dagger}}}    & \targ    & \gate{\mathrm{T}}              & \targ    & \gate{\mathrm{T^{\dagger}}} & \targ    & \qw                   & \qw      & \qw & \\
}

  }
  \caption{Implementing $x_a \cdot x_b \cdot y$ using CNOT and T gates}
  \label{fig-toff-mark}
\end{figure}

\vspace{0.3cm}

In general, a pseudo-Boolean function is defined as
\begin{equation}
  \label{eq-pseudo-boolean}
  F(\mathbf{x}) = \sum_{\mathbf{k}} c_k \cdot f_k(\mathbf{x}), \quad c_k \in \mathbb{R},\ f_k : \{0,1\}^n \to \{0,1\}
\end{equation}
In plain terms, it is a sum of products of Boolean functions with real coefficients, resulting in binary values~\cite{bib-barenco-elementary}. This representation is also known as a {\it phase polynomial}~\cite{bib-amy-cnot}.

\vspace{0.5cm}

Phase polynomials are important because they can be calculated using the {\it Boolean Fourier transform}:
\begin{equation}
  \begin{aligned}
    \label{eq-boolean-fourier}
    f(\mathbf{x}) = \sum_{\mathbf{k} \neq 0} \hat{f}(\mathbf{k}) \cdot ( k_0 x_0 \oplus k_1 x_1 \oplus \cdots \oplus k_{n-1} x_{n-1}), \\\nonumber
    \hat{f}(\mathbf{k}) \in \mathbb{R} \quad \mathbf{k} \in \{0,1\}^{n}
  \end{aligned}
\end{equation}

\vspace{0.3cm}

The coefficients $\hat{f}(\mathbf{k})$ (for all nontrivial $\mathbf{k}$) specify the phase polynomial, and the linear functions $( k_0 x_0 \oplus k_1 x_1 \oplus \cdots \oplus k_{n-1} x_{n-1})$ form a basis.

Such a phase polynomial can be realized using a CNOT+T network, as in Fig.~\ref{fig-toff-mark}. Minimizing the number of odd multiples of $\frac{\pi}{4}$ in $\hat{f}(\mathbf{k})$ also minimizes the T-count~\cite{bib-amy-rm}. Optimizing T-depth is done as in~\cite{bib-amy-matroid}.

\vspace{0.4cm}

\section{Phase Polynomial Calculation for the Three Qubit Case}
\label{Chap:Bool-pbool3q}

\vspace{0.3cm}

This section shows how to calculate the phase polynomial for a three-input Boolean function, using the Boolean Fourier transform~\cite{bib-odonnell}. The notation is chosen to clarify how the equation may be implemented using quantum gates in the CNOT+T basis. 

\vspace{0.3cm}

The inner product for the Boolean Fourier transform is
\begin{equation}
  \label{eq-inner-prod}
  \langle f(\mathbf{x}) , g(\mathbf{x}) \rangle = \frac{1}{2^n} \sum_{\mathbf{x}} (-1)^{f(\mathbf{x})} (-1)^{g(\mathbf{x})}, \quad \mathbf{x} \in \{0,1\}^n
\end{equation}

The {\it parity functions}, i.e., exclusive sums of input variables $\chi(x_0,\ldots,x_n)$ as in Eq.~\ref{eq-bool-basis}, form an orthogonal basis with respect to this product:
\begin{equation}
  \label{eq-bool-basis}
  \chi(x_0,\ldots,x_n)_{\mathbf{k}} = k_0 x_0 \oplus k_1 x_1 \oplus \cdots \oplus k_{n-1} x_{n-1},\quad k_i \in \{0,1\}
\end{equation}

\vspace{0.2cm}

Since $\langle \chi_{\mathbf{k}}, \chi_{\mathbf{b}} \rangle = 0$ for $\mathbf{k} \neq \mathbf{b}$, the coefficient $\hat{f}(\mathbf{k})$ is computed as
\begin{equation}
  \label{eq-fhat-prod}
  \hat{f}(\mathbf{k}) = \langle f(\mathbf{x}), \chi(\mathbf{x})_{\mathbf{k}} \rangle
\end{equation}

\vspace{0.2cm}

\begin{example}
As an example, let's derive the coefficients for Eq.~\ref{eq-toff-bool} with $\mathbf{k} = (k_{x_a}, k_{x_b}, k_y)$.
For $\mathbf{k} = 100$, $\chi_{100} = x_a$. Table~\ref{table-ex-100} enumerates the values and the term $(-1)^{f(\mathbf{x})} (-1)^{\chi_{100}(\mathbf{x})}$ for the inner product.
\begin{table}[h]
  \begin{center}
    \begin{tabular}{c|c|c|c|c|c}
      \hline
      $x_a$ & $x_b$ & $y$ & $f(\mathbf{x})$ & $\chi_{100}$ & $(-1)^{\chi_{100}(\mathbf{x})} (-1)^{f(\mathbf{x})}$ \\\hline
      0 & 0 & 0 & 0 & 0 & 1\\\hline
      0 & 0 & 1 & 0 & 0 & 1\\\hline
      0 & 1 & 0 & 0 & 0 & 1\\\hline
      0 & 1 & 1 & 0 & 0 & 1\\\hline
      1 & 0 & 0 & 0 & 1 & -1\\\hline
      1 & 0 & 1 & 0 & 1 & -1\\\hline
      1 & 1 & 0 & 0 & 1 & -1\\\hline      
      1 & 1 & 1 & 1 & 1 & 1\\\hline
    \end{tabular}
    \caption{The value of $\chi_{100}$ and the inner product $\langle f(\mathbf{x}), \chi_{100} \rangle$}
    \label{table-ex-100}
  \end{center}
\end{table}

Summing the rightmost column, as in Eq.~\ref{eq-inner-prod}:
\begin{align}
    &\hat{f}(100) = \langle f(\mathbf{x}), \chi_{100} \rangle = \frac{1}{8} ( 1 + 1 + 1 + 1 - 1 - 1 - 1 + 1 ) = \frac{1}{4}
\end{align}
Thus,
\begin{equation}
  f(\mathbf{x}) = \frac{1}{4}x_a + \sum_{\mathbf{k} \neq 000,100} \hat{f}(\mathbf{k}) \cdot \chi_{\mathbf{k}}
\end{equation}
The rest of the calculation is omitted for brevity, but readers can verify that it results in Eq.~\ref{eq-toff-bool}.
\end{example}

\vspace{0.4cm}

\section{Analysis}

\vspace{0.3cm}

On analysis of the T-count and T-depth of the generated library, perhaps surprisingly, we found
that every circuit fell into two categories: {\it even-weighted} (even number of fulfilling
assignments) functions that had 0 T-count and 0 T-depth, and {\it odd-weighted} (odd number
of fulfilling assignments) functions, with a T-count of 7 and T-depth of 3. We offer an
explanation below.

\vspace{0.2cm}

\subsection{Even-Weight Functions}

\vspace{0.2cm}

Observe from Eq.~\ref{eq-inner-prod} that when $n=2$ i.e. for Boolean functions
of two variables, the coefficients are at most $1/2$, meaning all such functions
can be implemented using only CNOT and S gates, with 0 T-count. One such circuit
is displayed in Fig.~\ref{fig:phase_x1x2}.

\vspace{0.2cm}

\begin{figure}[h]
  \centering
  \[
  \Qcircuit @C=1em @R=1em {
    \lstick{\ket{x_1}} & \gate{S} & \ctrl{1} & \qw              & \ctrl{1} & \qw  \\
    \lstick{\ket{x_2}} & \gate{S} & \targ    & \gate{S^\dagger} & \targ    & \qw
  }
  \]
  \caption{Circuit that implements a phase of $x_1 \cdot x_2$.}
  \label{fig:phase_x1x2}
\end{figure}

\vspace{0.3cm}

Functions $f$ with an even number of satisfying assignments can be expressed as:
\[
f(x) = g(x_i, x_j \oplus x_k)
\quad \text{or} \quad 
f(x) = g(x_i \oplus x_j, x_k)
\]
for some $i,j,k$, where $g$ is a two-variable Boolean function. This can be confirmed
quickly from calculating all such 3 variable functions. They are {\it affine} with
two variable functions up to an XOR with one of the inputs or the output.

\vspace{0.3cm}

\subsection{Odd-Weight Functions}

\vspace{0.2cm}

For a function $f$ such that $f(x) = 1$ for exactly one input (e.g., $f(x) = x_0 x_1 x_2$), the $T$-count is known to be 7.

\vspace{0.2cm}

Any function with an odd number of true assignments can be decomposed as:
\[
f = f_{\text{even}} \oplus f_{\text{single}}
\]
where $f_{\text{even}}$ has an even number of satisfying assignments, and $f_{\text{single}}$ has exactly one. Thus, the
$T$-count of an odd-weighted such $f$ is 7.

\vspace{0.4cm}

\section{Conclusion}
\label{Sec:concl}

\vspace{0.3cm}

We produced T-count and T-depth optimal 3-qubit phase oracles to implement 3-bit Boolean functions
such as those required in Grover's Algorithm. Upon analyzing the produced library, we realized that
even-weighted functions required no T-gates to implement, while odd-weighted functions required at least 7.
Future work involves using this library to construct arbitrary $n$-bit phase oracles.

\vspace{0.2cm}

It informs some interesting future avenues of inquiry.
\begin{itemize}
  \item Can functions with even-weight truth tables in general be implemented with the same cost as some $f'$ of $n-1$ variables?
  \item Can functions of the form
  \[
  f(x_0,\ldots,x_n) = g(x_{i_1} \oplus x_{i_2} \oplus \cdots, x_{j_1} \oplus x_{j_2} \oplus \cdots, x_{k_1} \oplus \cdots)
  \]
  where $g$ depends on $\leq 3$ variables, be synthesized exactly with only CNOT+T gates? If so, can we use this fact to optimize
  synthesis of circuits in general by integrating usage of the library produced in this work?
\end{itemize}

\vspace{0.3cm}

While the usages of 3 variable phase oracles themselves may be limited, there could be further applications for
phase oracles of 4 or more variables by composing them. This will be interesting future research.

\vspace{0.5cm}

\section*{Acknowledgments}                                                     
This work was supported by JSPS KAKENHI Grant Numbers 24K22298 and JST, CREST Grant Number JPMJCR24I4, Japan.

\vspace{0.5cm}

\bibliographystyle{unsrt}
\bibliography{ref}

\end{document}
% end of file
